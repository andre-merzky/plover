 
  Several SAGA packages share the notion of name spaces and
  operations on these name spaces.  In order to increase
  consistency in the API, these packages share the same
  API paradigms.  This section describes those paradigms, and
  these classes which operate on arbitrary hierarchical
  name spaces, such as used in physical, virtual, and logical file
  systems, and in information systems.
 
  The API is inspired by the POSIX standard, which defines tools
  and calls to handle the name space of physical files
  and directories.  The methods listed for the interfaces
  have POSIX-like syntax and semantics.
 
  While POSIX has an iterative interface to directory listing
  (i.e. opendir, telldir, seekdir, readdir), the corresponding
  part of the interface included here deviates significantly
  from the POSIX version: it has fewer calls, with a different
  syntax, but identical semantics.
 
  Please note that 'stat'-like API calls are \I{not} covered
  here -- they are rather meaningless on a name space per se,
  but belong to the specific implementations, e.g.  physical
  files, which inherit the |namespace| classes.
 
  \subsubsection{Definitions}
 
   The Grid File System Working Group in OGF has defined a
   Resource Name\-space Service (RNS~\cite{rns}).  The SAGA Core
   API specification follows the definition of a name space from
   that document.
 
   \paragraph{Directory:}
 
     A 'Directory' represents what~\cite{rns} defines as 'Virtual
     Directory':
 
     \up
     \begin{quote}
 
       \small \it
 
       ``A virtual directory is an RNS entry that is represented
       as a non-leaf node in the hierarchical name space tree.
       When rendered by a name space service client, a virtual
       directory functions similar to that of a standard
       filesystem directory or registry key. It is considered
       virtual because it does not have any corresponding
       representation outside of the name space. A virtual
       directory, therefore, is purely a name space entity that
       functions in much the same way as a conventional
       filesystem directory or registry key by maintaining a
       list of subentries, which thereby demonstrate a
       hierarchical relationship. There are no restrictions
       regarding the layout of the name space tree; both virtual
       directories and junctions can be nested within nested
       virtual directories recursively.
 
       A virtual directory may be considered analogous to a
       collection, category, or context -- to the extent that
       these terms are used in most directory, registry, or
       catalogue contexts. Virtual directories do not have any
       time or space existence outside of the name space and
       strictly serve to facilitate hierarchy.  Name space
       hierarchies offer categorization or grouping of entries,
       by presenting the illusion of compartments, which may
       contain sub-compartments as well as junctions.''
     
     \end{quote}
 
 
   \paragraph{Directory Entry:}
 
     A \I{directory entry} or \I{entry} represent
     what~\cite{rns} defines as 'Junction'.  Note that any type
     of junction defined there could be used:
 
     \up
     \begin{quote}
 
       \small \it
 
       ``A junction is an RNS entry that interconnects a
       reference to an existing resource into the hierarchical
       name space. Junctions represent a name-to-resource mapping
       that is composed of a human oriented index key or `name'
       that maps to an endpoint reference. The endpoint
       reference may refer to any addressable resource, which
       includes other name space entries, as well as names or
       unique identifiers to be resolved by other resolution
       service, as well as definitive target consumable
       resource. All compliant RNS implementations MUST embody
       the target information of a name space junction within a
       valid WS-Addressing [\dots] Endpoint Reference (EPR).''
 
     \end{quote}
 
   \paragraph{Pathnames:}
 
     A \I{pathname} as accepted by this specification MUST be
     either formatted as URLs or MUST follow the specification
     of entry names as described
     in~\cite{rns}, Section 1.2.2.1 ``Entry Name Restrictions''
     (formatting changed):
 
     \begin{quote}
 
       \small \it
 
      ``Entry names are composed of a simple string of human
      readable characters. Since certain characters serve
      special purposes both within the name space service and
      within a number of systems that may use this service, this
      section describes the mandatory restrictions for all entry
      names:\\
 
      Names MUST NOT...

        \begin{shortlist}
         \item Contain any of the following characters: 
               \verb'/ : ; * ? " < > |'
         \item Contain any non-readable characters, such as the
               carriage return (ANSI 13) or line feed (ANSI 10) 
               or tab (ANSI 9)
         \item Be greater than 255 characters in length
               (Unicode)\\
        \end{shortlist}
 
       Names SHOULD...
          
        \begin{shortlist}
         \item Accommodate Unicode characters
         \item Be easily readable by a human user, suggesting 
               less than 32 characters per name\\
        \end{shortlist}
 
      Names MAY...
 
        \begin{shortlist}
         \item Contain space (ANSI 32) characters\\
        \end{shortlist}
 
      Notice these restrictions apply to entry names and are not
      describing paths. Paths are constructed of one or more
      entry names separated by the forward slash character (|/|)''.
 
     \end{quote}
 
     Note that, in fact, pathnames as specified above \I{are}
     syntactically valid URLs, and this specification is
     therefore only referring to URLs.  Both, SAGA
     implementations and SAGA usage SHOULD, however, strive for
     compliance with~\cite{rns}.  An exception is the use of
     relative pathnames which, in SAGA, can contain wildcards
     (see below).
 
     All method arguments which are named |name|, |source| or
     |target| are considered pathnames.  These pathnames can
     always be relative pathnames (i.e. they can be relative to
     the current working directory (|cwd|) of the object
     instance the operation is performed upon, e.g. when they
     start with |'./'| or |'../'|).  

     \XAdd[4]{Note that relative path elements are not always
     resolvable during URL construction.  Instead, resolution
     may be delayed until the URL is being used, and further may
     need to be performed differently on each use of the URL,
     depending on the context of usage:}


     \shift |saga::url u0 ("ftp://localhost/tmp/data/test.txt");|\\
     \shift |saga::url u1 ("gridftp://localhost/tmp/data/test.txt");|\\
     \shift |saga::url u2 ("../test.txt");|

     \shift |saga::file f0 (u0);|\\
     \shift |saga::file f1 (u1);|

     \shift |f0.move (u2);  // resolve u2 relative to u0|\\
     \shift |f1.move (u2);  // resolve u2 relative to u1|

     \XMark[9]
 
     Note that the comments from Section~\ref{ssec:urlprob},
     apply here.  In particular, an implementation MAY throw an
     |IncorrectURL| exception if it is unable to handle a given
     URL, e.g. because of its scheme.
 
 
   \paragraph{Current Working Directory (cwd)}
 
     Every |saga::ns_entry| instance has an associate current
     working directory (cwd), which forms the implicit base for
     all operations on relative pathnames.  For
     |saga::ns_directory| instances, that cwd can be changed
     with the |change_dir| method.  Otherwise, cwd only changes
     if the entry itself is |move()|'d.
 
 
   \paragraph{Links:}
 
     \I{Links} in this specification are considered \I{symbolic
     links}, i.e. they can break if the entry they point to is
     removed.  An implementation MAY support links, as not all
     backends can support links, and others might support links
     only in specific circumstances (e.g. if entry and link live
     on the same file system).
 
     The 'Dereference' flag allows methods to operate on the
     link target instead of the link -- only one level of
     reference is resolved though.  The |read_link()| method
     does also resolve only one link level, and returns a URL
     pointing to the link target.
 
     At the moment, \cite{rns} does not have a notion
     of symbolic links.  However, an RNS 'junction' which is
     associated with another RNS junction can be regarded as
     a symbolic link.
 
 
  \paragraph{Wildcards:}
 
    The API supports wildcards for a
    number of calls, as listed below, and thereby follows the POSIX
    standard~\cite{posix1,posix2,posix3} for shell
    wildcards.  Available wildcard patterns are:
 
    \begin{tabbing}
    xxxxxxx \= xxxxxxx  \= \kill
            \> |*     | \> : matches any string\\
            \> |?     | \> : matches a single character\\
            \> |[abc] | \> : matches any  of a set of characters\\
            \> |[a-z] | \> : matches any  of a range of characters\\
            \> |[!abc]| \> : matches none of a range of characters\\
            \> |[!a-z]| \> : matches none of a range of characters\\
            \> |{a,bc}| \> : matches any  of a set of strings\\
    \end{tabbing}
 
    See the POSIX standard~\cite{posix1,posix2,posix3} for
    more details.  In the SAGA API, wildcards are allowed
    in all pathnames where they can be used in the respective
    shell commands, as:
 
      \shift |copy   *.txt dir|\\
      \shift |move   *.txt dir|\\
      \shift |link   *.txt dir|\\
      \shift |ls     *.txt|\\
      \shift |remove *.txt|\\
 
 
    Note that only those methods MUST support
    wildcards for which this is explicitly specified here.  Other
    methods MUST NOT support wildcards, as this would not be
    meaningful.  Flags MUST be applied to all elements of a wildcard
    expansion, even if that raises an exception for any reason.
 
    For the use of wildcards, separate calls are provided
    which accept strings instead of URLs.  The reason for this is that
    RFC~3986~\cite{rfc-3986}, which defines the syntax of URLs,
    explicitly forbids most POSIX wildcard characters as part of a
    URL.  Also, we feel that wildcards make most sense in relative
    pathnames (i.e. relative to a working directory). Strings in these
    separate calls thus MUST be relative paths, and thus MUST only
    contain URL path elements, whereby the path element MUST NOT start
    with an \texttt{'/'}.  Apart from that, the semantics of the
    wildcard-enabled string method versions of the calls are identical
    to the semantics of their respective URL counterparts.  If the
    method encounters any error condition on any one of the expanded
    URLs, an exception is thrown, and the state of the other (valid or
    invalid) expanded URL targets remains undefined.
 
 
 
  \paragraph{Opening and Closing Name Space Entries:}
 
   If a |ns_entry| object instance gets created, it is also
   opened.  Hence, the semantics and all notes of the
   respective |open()| call also apply to the
   constructor.  The same holds for all classes that inherit
   |ns_entry|.
 
   In accordance with Section~\ref{ssec:garbage}, the
   |saga::ns_entry| class has a |close()| method, which allows
   to enforce a timely release of used (local and remote)
   resources.  After a |ns_entry| instance was closed, all
   method calls on that instance (apart from the
   |DESTRUCTOR|) MUST throw an |IncorrectState| exception.
   A destruction of an entry implies the respective |close()|
   semantics.  The same holds for all classes that inherit
   |ns_entry|.
 
   If an entry gets successfully opened without specifying
   'Lock' as open flag, its state may get corrupted if some
   other backend operation removes or moves the opened entity,
   or changes its state.  In that case, any subsequent operation
   on the object instance can fail unexpectedly.  An
   |IncorrectState| exception describing the type of state
   change SHOULD be thrown if such a state change is detected
   and causes an operation to fail. Otherwise, the normal
   exception indicating the type of error which occurred
   SHOULD be thrown.  The |IncorrectState|
   exception is thus listed on most method calls below,
   but not individually motivated unless it is also used in any
   other semantic context.
 
 
 \subsubsection{Specification}
 
 \begin{myspec}
  package saga.namespace
  {
    enum flags
    {
      None            =    0,
      Overwrite       =    1,
      Recursive       =    2,
      Dereference     =    4,
      Create          =    8,
      Exclusive       =   16,
      Lock            =   32,
      CreateParents   =   64,
!     //                 128,    reserved for Truncate
!     //                 256,    reserved for Append
!     Read            =  512,
!     Write           = 1024,
!     ReadWrite       = 1536  // Read | Write
    }
 
 
    class ns_entry : implements   saga::object,
                     implements   saga::async
                     implements   saga::permissions
                  // from object  saga::error_handler
    {
      CONSTRUCTOR        (in  session        s,
                          in  url            name,
                          in  int            flags   = None,
                          out ns_entry       obj     );
      DESTRUCTOR         (in  ns_entry       obj     );
 
      // basic properties
      get_url            (out url            url     );
      get_cwd            (out url            cwd     );
      get_name           (out url            name    );
 
      // navigation/query methods
      is_dir             (out boolean        test    );
      is_entry           (out boolean        test    );
      is_link            (out boolean        test    );
      read_link          (out url            link    );
+     get_mtime          (out int            time    );
 
      // management methods
      copy               (in  url            target,
                          in  int            flags = None);
      link               (in  url            target,
                          in  int            flags = None);
      move               (in  url            target,
                          in  int            flags = None);
      remove             (in  int            flags = None);
      close              (in  float          timeout = 0.0);
 
 
      // permissions with flags
      permissions_allow  (in  string         id,
                          in  permission     perm, 
                          in  int            flags = None);
      permissions_deny   (in  string         id,
                          in  permission     perm, 
                          in  int            flags = None);
    }
 
 
    class ns_directory : extends        saga::ns_entry
                         // from ns_entry  saga::object
                         // from ns_entry  saga::async
                         // from ns_entry  saga::permissions
                         // from object    saga::error_handler
    {
      CONSTRUCTOR        (in  session        s,
                          in  url            name,
                          in  int            flags = None,
                          out ns_directory   obj     );
      DESTRUCTOR         (in  ns_directory   obj     );
 
      // navigation/query methods
      change_dir         (in  url            dir     );
      list               (in  string         name_pattern = ".",
                          in  int            flags   = None, 
                          out array<url>     names   );
      find               (in  string         name_pattern,
                          in  int            flags = Recursive,
                          out array<url>     names   );
      exists             (in  url            name,
                          out boolean        exists  );
      is_dir             (in  url            name,
                          out boolean        test    );
      is_entry           (in  url            name,
                          out boolean        test    );
      is_link            (in  url            name,
                          out boolean        test    );
      read_link          (in  url            name,
                          out url            link    );
+     get_mtime          (in  url            name,
+                         out int            time    );
 
      // manage entries by number
      get_num_entries    (out int            num     );
      get_entry          (in  int            entry,
                          out url            name    );
 
      // management methods
      copy               (in  url            source,
                          in  url            target,
                          in  int            flags = None);
      link               (in  url            source,
                          in  url            target,
                          in  int            flags = None);
      move               (in  url            source,
                          in  url            target,
                          in  int            flags = None);
      remove             (in  url            target,
                          in  int            flags = None);
      make_dir           (in  url            target,
                          in  int            flags = None);
 
      // management methods - wildcard versions
      copy               (in  string         source,
                          in  url            target,
                          in  int            flags = None);
      link               (in  string         source,
                          in  url            target,
                          in  int            flags = None);
      move               (in  string         source,
                          in  url            target,
                          in  int            flags = None);
      remove             (in  string         target,
                          in  int            flags = None);
 
      // factory methods
      open               (in  url            name,
!                         in  int            flags = Read,
                          out ns_entry       entry   );
      open_dir            (in  url           name,
!                         in  int            flags = Read,
                          out ns_directory   dir     );
 
 
      // permissions with flags
      permissions_allow  (in  url            target,
                          in  string         id,
                          in  int            perm, 
                          in  int            flags = None);
      permissions_deny   (in  url            target,
                          in  string         id,
                          in  int            perm, 
                          in  int            flags = None);
 
      // permissions with flags - wildcard versions
      permissions_allow  (in  string         target,
                          in  string         id,
                          in  int            perm, 
                          in  int            flags = None);
      permissions_deny   (in  string         target,
                          in  string         id,
                          in  int            perm, 
                          in  int            flags = None);
    }
  }
 \end{myspec}
 
 
 \subsubsection{Specification Details}
 
  \subsubsection*{Enum \T{flags}}
 
  The \T{flags} describe the properties of several
  operations on namespace entries.  Packages which inherit from
  the namespace package use the same flag semantics unless
  specified otherwise, but will, in general, add additional flags
  to some operations.
  
    |None|\\[0.3mm]
    \begin{tabular}{cp{110mm}}
      ~~ & indicates the absence of flags, and thus also
           implies that the default flags for an operation do 
           not apply, either.
    \end{tabular}
 
    |Overwrite|\\[0.3mm]
    \begin{tabular}{cp{110mm}}
      ~~ & enforces an operation which creates a new
           namespace entry to continue even if the target entry 
           does already exist -- if that flag is not given, an 
           \T{'AlreadyExists'} exception would result from 
           such an operation.
    \end{tabular}
 
    |Recursive|\\[0.3mm]
    \begin{tabular}{cp{110mm}}
      ~~ & enforces an operation to apply recursively on a
           directory tree -- if that flag is not given, the same
           operation would only apply to the given directory,
           and not to its children.
    \end{tabular}
 
    |Dereference|\\[0.3mm]
    \begin{tabular}{cp{110mm}}
      ~~ & enforces an operation to apply not to the entry
           pointed to by the target name, but to the link target
           of that entry -- if that flag is not given, the same
           operation would apply to the entry directly, and its
           link target stays unaffected.
    \end{tabular}
 
    |Create|\\[0.3mm]
    \begin{tabular}{cp{110mm}}
      ~~ & allows a namespace entry to be created while 
           opening it, if it does not already exist -- if that 
           flag is not given, the same open operation would 
           cause a \T{'DoesNotExist'} exception.  If the entry 
           exists, the flag is ignored.  \XAddn{This flag implies
           the \T{'Write'} flag.}
    \end{tabular}
 
    |Exclusive|\\[0.3mm]
    \begin{tabular}{cp{110mm}}
      ~~ & implies a modification to the meaning of the
           \T{Create} flag: if the entry already exists, the
           \T{Create} flag is is no longer silently ignored, but
           causes an \T{'AlreadyExists'} exception.
    \end{tabular}
 
    |Lock|\\[0.3mm]
    \begin{tabular}{cp{110mm}}
      ~~ & enforces a lock on the name space entry when
            it is opened.  Locks are advisory in SAGA, semantic 
            details for locking are defined in the description 
            of the \T{open()} call.
    \end{tabular}
 
    |CreateParents|\\[0.3em]
    \begin{tabular}{cp{110mm}}
      ~~ & An operation which would create a name space 
           entry would normally fail if any path element in the
           targets name does not yet exist.  If this flag is
           given, such an operation would not fail, but would
           imply that the missing path elements are created on
           the fly.  \XAddn{This flag implies the \T{'Create'} 
           flag.}
    \end{tabular}
 
    |Read|\\[0.3mm]
    \begin{tabular}{cp{110mm}}
      ~~ & \XAddn{The entry or directory is opened for reading 
           -- that does not imply the ability to write to the 
           entry or directory.}
    \end{tabular}
    \XMark[2]
 
    |Write|\\[0.3mm]
    \begin{tabular}{cp{110mm}}
      ~~ & \XAddn{The entry or directory is opened for writing 
           -- that does not imply the ability to read from the 
           entry or directory.}
    \end{tabular}
    \XMark[2]
 
    |ReadWrite|\\[0.3mm]
    \begin{tabular}{cp{110mm}}
      ~~ & \XAddn{The entry or directory is opened for reading 
           and writing.}
    \end{tabular}
    \XMark
 
 
  \subsubsection*{Class \T{ns\_entry}}
 
    |ns_entry| defines methods which serve the inspection of the
    entry itself, methods which allows to manage the entry (e.g.
    to copy, move, or remove it),  and methods to manipulate the
    entry's access control lists.
 
    In general, multiple such URLs might be valid to identify an
    entry:
 
    \up\up
    \begin{myspec}
       ftp://ftp.host.net/pub/data/test.txt
      http://www.host.net/ftp/data/test.txt
      http://www.host.net/ftp/data/./test.txt
      http://www.host.net/ftp/data/../data/test.txt
    \end{myspec}
    \up
 
 
    Any valid URL can be returned on |get_url()|, but it SHOULD
    not contain |'..'| or |'.'| \XAddn{as non-leading} path
    \XRep{elements}{components}, i.e.  \XSpelln{SHOULD} have
    a normalized path element.  The URL returned on |get_url()|
    should serve as base for the return values on |get_cwd()|
    and |get_name()|: In general it should hold \XAdd{that}:
    
    \shift\T{get\_url() == get\_cwd() + '/' + get\_name()}
    \XMark

    
    \XAdd{Note that \T{get\_cwd()} behaves like the unix command
    \T{'dirname'}.}

 
 \begin{myspec}
    - CONSTRUCTOR
      Purpose:  create the object
      Format:   CONSTRUCTOR          (in  session   s, 
                                      in  url       name,
!                                     in  int       flags = Read,
                                      out ns_entry obj)
      Inputs:   s:                    session handle
                name:                 initial working dir
                flags:                open mode
      InOuts:   -
      Outputs:  obj:                  the newly created object
      PreCond:  -
      PostCond: - the entry is opened.
                - 'Owner' of target is the id of the context
                  use to perform the operation, if the entry 
                  gets created.
      Perms:    Exec  for parent directory.
                Write for parent directory if Create is set.
                Write for name if Write is set.
                Read  for name if Read  is set.
      Throws:   NotImplemented
                IncorrectURL
                BadParameter
                DoesNotExist
                AlreadyExists
                PermissionDenied
                AuthorizationFailed
                AuthenticationFailed
                Timeout
                NoSuccess
!     Notes:    - the default flag set is 'Read'.
                - the constructor performs an open of the 
                  entry - all notes to the respective open 
                  call (on namespace_directory) apply.
 
 
    - DESTRUCTOR
      Purpose:  destroy the object
      Format:   DESTRUCTOR           (in  ns_entry     obj)
      Inputs:   obj:                  the object to destroy
      InOuts:   -
      Outputs:  -
      PreCond:  -
      PostCond: - the entry is closed.
      Perms:    -
      Throws:   - 
      Notes:    - if the instance was not closed before, the 
                  destructor performs a close() on the instance,
                  and all notes to close() apply.
 
 
 
    Methods for inspecting ns_entry:
    --------------------------------
 
    - get_url
      Purpose:  obtain the complete url pointing to the entry
      Format:   get_url            (out  url       url);
      Inputs:   -
      InOuts:   -
      Outputs:  url                 url pointing to the entry
      PreCond:  -
      PostCond: -
      Perms:    -
      Throws:   NotImplemented
                IncorrectState
                Timeout
                NoSuccess
      Notes:    -
 
 
 
    - get_cwd
      Purpose:  obtain the current working directory for the
                entry
      Format:   get_cwd            (out url       cwd);
      Inputs:   -
      InOuts:   -
      Outputs:  cwd                 current working directory
      PreCond:  -
      PostCond: -
      Perms:    -
      Throws:   NotImplemented
                IncorrectState
                Timeout
                NoSuccess
      Notes:    -
                - returns the directory part of the url path
                  element.
 
 
 
    - get_name
      Purpose:  obtain the name part of the url path element
      Format:   get_name           (out  url       name);
      Inputs:   -
      InOuts:   -
      Outputs:  name                last part of path element
      PreCond:  -
      PostCond: -
      Perms:    -
      Throws:   NotImplemented
                IncorrectState
                Timeout
                NoSuccess
      Notes:    -
 
 
 
    - is_dir
      Purpose:  tests the entry for being a directory
      Format:   is_dir             (out boolean test);
      Inputs:   -
      InOuts:   -
      Outputs:  test:               boolean indicating if entry
                                    is a directory
      PreCond:  -
      PostCond: -
      Perms:    Query
                Query for parent directory.
      Throws:   NotImplemented
                IncorrectState
                PermissionDenied
                AuthorizationFailed
                AuthenticationFailed
                Timeout
                NoSuccess
      Notes:    - returns true if entry is a directory, false
                  otherwise
                - similar to 'test -d' as defined by POSIX.
 
 
    - is_entry
      Purpose:  tests the entry for being an ns_entry
      Format:   is_entry           (out boolean test);
      Inputs:   -
      InOuts:   -
      Outputs:  test:               boolean indicating if entry
                                    is an ns_entry
      PreCond:  -
      PostCond: -
      Perms:    Query
                Query for parent directory.
      Throws:   NotImplemented
                IncorrectState
                PermissionDenied
                AuthorizationFailed
                AuthenticationFailed
                Timeout
                NoSuccess
      Notes:    - the method returns false if the entry is a
                  link or a directory (although an ns_directory
                  IS_A ns_entry, false is returned on a test on 
                  an ns_directory) - otherwise true is returned.
                - similar to 'test -f' as defined by POSIX.
 
 
    - is_link
      Purpose:  tests the entry for being a link
      Format:   is_link            (out boolean test);
      Inputs:   -
      InOuts:   -
      Outputs:  test:               boolean indicating if
                                    entry is a link
      PreCond:  -
      PostCond: -
      Perms:    Query
                Query for parent directory.
      Throws:   NotImplemented
                IncorrectState
                PermissionDenied
                AuthorizationFailed
                AuthenticationFailed
                Timeout
                NoSuccess
      Notes:    - returns true if the entry is a link, false
                  otherwise
!               - similar to libc's 'readlink' as defined by
+                 POSIX, but with only one level of redirection
+                 resolved.
 
 
    - read_link 
      Purpose:  returns the name of the link target
      Format:   read_link          (out url       link);
      Inputs:   -
      InOuts:   -
      Outputs:  link:               resolved name
      PreCond:  -
      PostCond: -
      Perms:    Query
                Query for parent directory.
      Throws:   NotImplemented
                IncorrectState
                PermissionDenied
                AuthorizationFailed
                AuthenticationFailed
                Timeout
                NoSuccess
      Notes:    - the returned name MUST be sufficient to
                  access the link target entry
                - resolves one link level only
                - if the entry instance this method is called
                  upon does not point to a link, an
                  'IncorrectState' exception is thrown.
!               - similar to libc's 'readlink' as defined by
+                 POSIX, but with only one level of redirection
+                 resolved.
 
+
+   - get_mtime
+     Purpose:  returns the last modification time
+     Format:   get_time           (out int time);
+     Inputs:   -
+     InOuts:   -
+     Outputs:  time:               time of last modification
+     PreCond:  -
+     PostCond: -
+     Perms:    Query
+               Query for parent directory.
+     Throws:   NotImplemented
+               IncorrectState
+               PermissionDenied
+               AuthorizationFailed
+               AuthenticationFailed
+               Timeout
+               NoSuccess
+     Notes:    - the returned number represents the time of
+                 last modification in seconds since epoch
+                 (01.01.1970)
+               - similar to the 'st_mtimespec' element of the
+                 stat strucure used in the POSIX stat() call.
+
 
    Methods for managing the name space entry:
    -------------------------------------------
 
    - copy
      Purpose:  copy the entry to another part of the name space
      Format:   copy               (in  url       target,
                                    in  int       flags = None);
      Inputs:   target:             name to copy to
                flags:              flags defining the operation
                                    modus
      InOuts:   -
      Outputs:  -
      PreCond:  -
      PostCond: - an identical copy exists at target.
                - 'Owner' of target is the id of the context
                  use to perform the operation, if target gets
                  created.
      Perms:    Query 
                Exec  for parent directory.
                Query for target.
                Query for target's parent directory.
                Exec  for target's parent directory.
                Write for target
                      if  target does exist.
                Write for target's parent directory 
                      if  target does not exist.
      Throws:   NotImplemented
                IncorrectURL
                BadParameter
                DoesNotExist
                AlreadyExists
                IncorrectState
                PermissionDenied
                AuthorizationFailed
                AuthenticationFailed
                Timeout
                NoSuccess
      Notes:    - if the target is a directory, the source entry
                  is copied into that directory
                - a 'BadParameter' exception is thrown if the 
                  source is a directory and the 'Recursive' flag 
                  is not set.
                - a 'BadParameter' exception is thrown if the 
                  source is not a directory and the 'Recursive' 
                  flag is set.
                - if the target lies in a non-existing part of
                  the name space, a 'DoesNotExist' exception is
                  thrown, unless the 'CreateParents' flag is
                  given - then that part of the name space must
                  be created.
                - if the target already exists, it will be
                  overwritten if the 'Overwrite' flag is set,
                  otherwise it is an 'AlreadyExists' exception.
                - if a directory is to be copied recursively,
                  but the target exists and is not a directory,
                  and not a link to a directory, an
                  'AlreadyExists' exception is thrown even if
                  the 'Overwrite' flag is set. 
                - if the instance points at an symbolic link, 
                  the source is deeply dereferenced before copy.
                  If derefencing is impossible (e.g. on a broken 
                  link), an 'IncorrectState' exception is thrown.
                - other flags are not allowed, and cause a
                  'BadParameter' exception.
                - the default flags are 'None' (0).
                - similar to 'cp' as defined by POSIX.
 
 
    - link
      Purpose:  create a symbolic link from the target entry to
                the source entry ( this entry) so that any reference 
                to the target refers to the source entry
      Format:   link               (in  url       target,
                                    in  int       flags = None);
      Inputs:   target:             name to link to
                flags:              flags defining the operation
                                    modus
      InOuts:   -
      Outputs:  -
      PreCond:  -
      PostCond: - a symbolic link to the entry exists at target.
                - 'Owner' of target is the id of the context
                  use to perform the operation if target gets
                  created.
      Perms:    Query 
                Exec  for parent directory.
                Query for target.
                Query for target's parent directory.
                Exec  for target's parent directory.
                Write for target
                      if  target does exist.
                Write for target's parent directory 
                      if  target does not exist.
      Throws:   NotImplemented
                IncorrectURL
                BadParameter
                DoesNotExist
                AlreadyExists
                IncorrectState
                PermissionDenied
                AuthorizationFailed
                AuthenticationFailed
                Timeout
                NoSuccess
      Notes:    - if the target is a directory, the source entry
                  is linked into that directory
                - if the source is a directory, and the
                  'Recursive' flag is set, the source directory
                  is recursively linked to the target (which must
                  be a directory as well - otherwise a
                  'BadParameter' exception is thrown).  The
                  method then behaves similar to lndir.  If the
                  'Recursive' flag is not set, the source entry
                  itself is linked.  
                - a 'BadParameter' exception is thrown if the 
                  source is not a directory and the 'Recursive' 
                  flag is set.
                - if the target lies in a non-existing part of
                  the name space, a 'DoesNotExist' exception is
                  thrown, unless the 'CreateParents' flag is
                  given - then that part of the name space must
                  be created.
                - if the target already exists, it will be
                  overwritten if the 'Overwrite' flag is set,
                  otherwise it is an 'AlreadyExists' exception.
                - if a directory is to be moved, but the target
                  exists and is not a directory, and not a link 
                  to a directory, an 'AlreadyExists' exception 
                  is thrown even if the 'Overwrite' flag is set. 
                - if the instance points at an symbolic link, 
                  the source is not dereferenced before linking, 
                  unless the 'Dereference' flag is given.  If
                  derefencing is impossible (e.g. on a broken 
                  link), an 'IncorrectState' exception is thrown.
                - other flags are not allowed, and cause a
                  'BadParameter' exception.
                - the default flags are 'None' (0).
                - similar to 'ln' as defined by POSIX.
 
 
    - move
      Purpose:  rename source to target, or move source to
                target if target is a directory.
      Format:   move               (in  url       target,
                                    in  int       flags = None);
      Inputs:   target:             name to move to
                flags:              flags defining the operation
                                    modus
      InOuts:   -
      Outputs:  -
      PreCond:  -
!     PostCond: - the entry exists at the target url.
!               - no entry exists at the original url.
+               - the object instance is not closed.
                - 'Owner' of target is the id of the context
                  use to perform the operation, if target gets
                  created.
      Perms:    Query 
                Write 
                Exec  for parent directory.
                Write for parent directory.
                Query for target.
                Exec  for target's parent directory.
                Write for target
                      if  target does exist.
                Write for target's parent directory 
                      if  target does not exist.
      Throws:   NotImplemented
                IncorrectURL
                BadParameter
                DoesNotExist
                AlreadyExists
                IncorrectState
                PermissionDenied
                AuthorizationFailed
                AuthenticationFailed
                Timeout
                NoSuccess
      Notes:    - if the target is a directory, the source entry
                  is moved into that directory.
                - a 'BadParameter' exception is thrown if the 
                  source is a directory and the 'Recursive' flag 
                  is not set.
                - a 'BadParameter' exception is thrown if the 
                  source is not a directory and the 'Recursive' 
                  flag is set.
                - if the target lies in a non-existing part of
                  the name space, a 'DoesNotExist' exception is
                  thrown, unless the 'CreateParents' flag is
!                 given - then that part of the name space MUST
                  be created.
                - if the target already exists, it will be
                  overwritten if the 'Overwrite' flag is set,
                  otherwise it is an 'AlreadyExists' exception.
                - if the instance points at an symbolic link, 
                  the source is not dereferenced before moving,
                  unless the 'Dereference' flag is given.
                  If derefencing is impossible (e.g. on a broken 
                  link), an 'IncorrectState' exception is thrown.
                - other flags are not allowed, and cause a
                  'BadParameter' exception.
                - the default flags are 'None' (0).
                - similar to 'mv' as defined by POSIX.
 
 
    - remove
      Purpose:  removes this entry, and closes it
      Format:   remove             (in  int  flags = None);
      Inputs:   target:             entry to be removed
      InOuts:   -
      Outputs:  -
      PreCond:  -
      PostCond: - the original entry is closed and removed.
      Perms:    Query 
                Write 
                Exec  for parent directory.
                Write for parent directory.
      Throws:   NotImplemented
                BadParameter
                IncorrectState
                PermissionDenied
                AuthorizationFailed
                AuthenticationFailed
                Timeout
                NoSuccess
      Notes:    - a 'BadParameter' exception is thrown if the 
!                 entry is a non-empty directory and the 
!                 'Recursive' flag is not set.
+               - if the 'Recursive' flag is defined, the target
+                 is recursively removed if it is a directory;
+                 otherwise this flag is ignored.
+               - if the 'Dereference' flag is specified, the
+                 method applies to the link target of target.
+                 The flag causes a 'BadParameter' exception if
+                 target is not a link.
                - a 'BadParameter' exception is thrown if the 
!                 entry is not a directory and the 'Recursive' 
                  flag is set.
-               - the entry will not be dereferenced unless the
-                 'Dereference' flag is given.  If derefencing is
-                 impossible (e.g. on a broken link), an
-                 'IncorrectState' exception is thrown.
                - other flags are not allowed, and cause a
                  'BadParameter' exception.
                - the default flags are 'None' (0).
                - if the instance was not closed before, this
                  call performs a close() on the instance, and 
                  all notes to close() apply.
                - similar to 'rm' as defined by POSIX.
 
 
    - close
      Purpose:  closes the object
      Format:   close              (in  float timeout = 0.0);
      Inputs:   timeout             seconds to wait
      InOuts:   -
      Outputs:  -
      PreCond:  -
      PostCond: - the entry instance is closed.
      Perms:    -
      Throws:   NotImplemented
-               IncorrectState
                NoSuccess
      Notes:    - any subsequent method call on the object
                  MUST raise an 'IncorrectState' exception
                  (apart from DESTRUCTOR and close()).
                - close() can be called multiple times, with no
                  side effects.
                - if close() is implicitly called in the
                  DESTRUCTOR, it will never throw an exception.
                - for resource deallocation semantics, see 
                  Section 2.
                - for timeout semantics, see Section 2.
 
 
    // overload permissions because of namespace specific flags
 
    - permissions_allow
      Purpose:  enable a permission 
      Format:   permissions_allow    (in  string  id,
                                      in  int     perm, 
                                      in  int     flags = None);
      Inputs:   id:                   id to set permission for
                perm:                 permission to enable
                flags:                mode of operation
      InOuts:   -
      Outputs:  -
      PreCond:  -
      PostCond: - the permissions are enabled.
      Perms:    Owner
      Throws:   NotImplemented
                BadParameter
                IncorrectState
                PermissionDenied
                AuthorizationFailed
                AuthenticationFailed
                Timeout
                NoSuccess
      Notes:    - all notes to permissions_allow from the
                  saga::permissions interface apply.
                - allowed flags are: 'Recursive', 'Dereference'.
                  All other flags cause a 'BadParameter'
                  exception.
                - specifying 'Recursive' for a non-directory
                  causes a 'BadParameter' exception.
 
 
    - permissions_deny
      Purpose:  disable a permission flag
      Format:   permissions_deny     (in  string     id,
                                      in  int        perm,
                                      in  int        flags);
      Inputs:   id:                   id to set permission for
                perm:                 permission to disable
                flags:                mode of operation
      InOuts:   -
      Outputs:  -
      PreCond:  -
      PostCond: - the permissions are disabled.
      Perms:    Owner
      Throws:   NotImplemented
                BadParameter
                IncorrectState
                PermissionDenied
                AuthorizationFailed
                AuthenticationFailed
                Timeout
                NoSuccess
      Notes:    - all notes to permissions_deny from the
                  saga::permissions interface apply.
                - allowed flags are: 'Recursive', 'Dereference'.
                  All other flags cause a 'BadParameter'
                  exception.
                - specifying 'Recursive' for a non-directory
                  causes a 'BadParameter' exception.
 \end{myspec}
 
 
  \subsubsection*{Class \T{ns\_directory}}
 
    |ns_directory| inherits all navigation and manipulation
    methods from |ns_entry|, but adds some more methods to these
    sets: instead of |dir.copy (target)| they allow, for
    example, to do |dir.copy (source, target)|.  Other methods
    added allow to change the cwd of the instance (which changes
    the values returned by the |get_name()|, |get_cwd()| and
    |get_url()| inspection methods), and others allow to open
    new |ns_entry| and |ns_directory| instances (|open()| and
    |open_dir()|).
 
    For all methods which have the same name as in the
    |ns_entry| class, the descriptions and semantics defined in
    |ns_entry| apply, unless noted here otherwise.
 
 
 \begin{myspec}
    - CONSTRUCTOR
      Purpose:  create the object
      Format:   CONSTRUCTOR          (in  session   s,
                                      in  url       name,
!                                     in  int       flags = Read,
                                      out ns_directory obj)
      Inputs:   name:                 initial working dir
                flags:                open mode
                s:                    session handle for
                                      object creation
      InOuts:   -
      Outputs:  obj:                  the newly created object
      PreCond:  -
      PostCond: - the directory is opened.
                - 'Owner' of target is the id of the context
                  use to perform the operation, if the
                  directory gets created.
      Perms:    Exec  for parent directory.
                Write for parent directory if Create is set.
                Write for name if Write is set.
                Read  for name if Read  is set.
      Throws:   NotImplemented
                IncorrectURL
                BadParameter
                DoesNotExist
                PermissionDenied
                AuthorizationFailed
                AuthenticationFailed
                Timeout
                NoSuccess
      Notes:    - the semantics of the inherited constructors
                  apply
                - the constructor performs an open of the 
                  entry - all notes to the respective open 
                  call apply.
!               - the default flag set is 'Read'.
 
 
    - DESTRUCTOR
      Purpose:  destroy the object
      Format:   DESTRUCTOR           (in  ns_directory obj)
      Inputs:   obj:                  the object to destroy
      InOuts:   -
      Outputs:  -
      PreCond:  -
      PostCond: - the directory is closed.
      Perms:    -
      Throws:   - 
      Notes:    - the semantics of the inherited destructors
                  apply
 
 
    Methods for navigation in the name space hierarchy:
    ---------------------------------------------------
 
    - change_dir
      Purpose:  change the working directory
      Format:   change_dir         (in  url        dir);
      Inputs:   dir:                directory to change to
      InOuts:   -
      Outputs:  -
      PreCond:  -
      PostCond: - dir is the directory the instance represents.
      Perms:    Exec for dir.
      Throws:   NotImplemented
                IncorrectURL
                BadParameter
                DoesNotExist
                IncorrectState
                PermissionDenied
                AuthorizationFailed
                AuthenticationFailed
                Timeout
                NoSuccess
      Notes:    - if 'dir' can be parsed as URL, but contains an
                  invalid directory name, a 'BadParameter'
                  exception is thrown.
                - if 'dir' does not exist, a 'DoesNotExist'
                  exception is thrown.
                - similar to the 'cd' command in the POSIX 
                  shell.
 
 
    - list
      Purpose:  list entries in this directory
      Format:   list               (in  string name_pattern = ".",
                                    in  int    flags        = None
                                    out array<url> names);
      Inputs:   flags:              flags defining the operation
                                    modus
                name_pattern:       name or pattern to list
      InOuts:   -
      Outputs:  names:              array of names matching the
                                    name_pattern
      PreCond:  -
      PostCond: -
      Perms:    Query for entries specified by name_pattern.
                Exec  for parent directories of these entries.
                Query for parent directories of these entries.
                Read  for directories specified by name_pattern.
                Exec  for directories specified by name_pattern.
                Exec  for parent directories of these directories.
                Query for parent directories of these directories.
      Throws:   NotImplemented
                IncorrectURL
                BadParameter
                IncorrectState
                PermissionDenied
                AuthorizationFailed
                AuthenticationFailed
                Timeout
                NoSuccess
      Notes:    - if name_pattern is not given (i.e. is an empty 
                  string), all entries in the current working 
                  directory are listed.
                - if name_pattern is given and points to a 
                  directory, the contents of that directory 
                  are listed.
                - the name_pattern follows the standard POSIX 
                  shell wildcard specification, as described 
                  above.
                - list does not follow symbolically linked
                  directories, unless the 'Dereference' flag
                  is specified - otherwise list lists symbolic 
                  link entries with a matching name.
                - if the 'DeReference' flag is set, list 
                  returns the name of link targets, not of the
                  link entry itself.
                - the default flags are 'None' (0).
                - other flags are not allowed, and cause a
                  'BadParameter' exception.
                - if the name_pattern cannot be parsed, a
                  'BadParameter' exception with a descriptive
                  error message is thrown.
                - if the name_pattern does not match any entry,
                  an empty list is returned, but no exception is
                  raised.
                - similar to 'ls' as defined by POSIX.
 
 
    - find
      Purpose:  find entries in the current directory and below
      Format:   find               (in  string  name_pattern,
                                    in  int     flags = Recursive,
                                    out array<url> names);
      Inputs:   name_pattern:       pattern for names of
                                    entries to be found
                flags:              flags defining the operation
                                    modus
      InOuts:   -
      Outputs:  names:              array of names matching the
                                    name_pattern
      PreCond:  -
      PostCond: -
      Perms:    Read  for cwd.
                Query for entries specified by name_pattern.
                Exec  for parent directories of these entries.
                Query for parent directories of these entries.
                Read  for directories specified by name_pattern.
                Exec  for directories specified by name_pattern.
                Exec  for parent directories of these directories.
                Query for parent directories of these directories.
      Throws:   NotImplemented
                BadParameter
                IncorrectState
                PermissionDenied
                AuthorizationFailed
                AuthenticationFailed
                Timeout
                NoSuccess
      Notes:    - find operates recursively below the current
                  working directory if the 'Recursive' flag is
                  specified (default)
                - find does not follow symbolically linked
                  directories, unless the 'Dereference' flag
                  is specified - otherwise find lists symbolic 
                  link entries with a matching name.
                - the default flags are 'Recursive' (1).
                - other flags are not allowed, and cause a
                  'BadParameter' exception.
                - the name_pattern follows the standard POSIX 
                  shell wildcard specification, as described 
                  above.
                - the matching entries returned are path names 
                  relative to cwd.
                - similar to 'find' as defined by POSIX, but
                  limited to the -name option.
 
 
    - exists
      Purpose:  returns true if entry exists, false otherwise
      Format:   exists             (in  url       name,
                                    out boolean  exists);
      Inputs:   name:               name to be tested for
                                    existence
      InOuts:   -
      Outputs:  exists:             boolean indicating existence
                                    of name
      PreCond:  -
      PostCond: -
      Perms:    Query for name.
                Exec  for name's parent directory.
                Read  for name's parent directory.
      Throws:   NotImplemented
                IncorrectURL
                BadParameter
                IncorrectState
                PermissionDenied
                AuthorizationFailed
                AuthenticationFailed
                Timeout
                NoSuccess
      Notes:    - if 'name' can be parsed as URL, but contains 
                  an invalid entry name, an 'BadParameter'
                  exception is thrown.
                - note that no exception is thrown if the entry
                  does not exist - the method just returns
                  'false' in this case.
                - similar to 'test -e' as defined by POSIX.
 
 
    - is_dir
      Purpose:  tests name for being a directory
      Format:   is_dir             (in  url       name,
                                    out boolean test);
      Inputs:   name:               name to be tested
      InOuts:   -
      Outputs:  test:               boolean indicating if name
                                    is a directory
      PreCond:  -
      PostCond: -
      Perms:    Query for name.
                Exec  for name's parent directory.
                Read  for name's parent directory.
      Throws:   NotImplemented
                IncorrectURL
                BadParameter
                DoesNotExist
                IncorrectState
                PermissionDenied
                AuthorizationFailed
                AuthenticationFailed
                Timeout
                NoSuccess
      Notes:    - returns true if the instance represents
                  a directory entry, false otherwise
!               - all notes to the ns_entry::is_dir() method 
                  apply.
                - if 'name' can be parsed as URL, but contains 
                  an invalid entry name, an 'BadParameter'
                  exception is thrown.
                - if 'name' is a valid entry name but the entry
                  does not exist, a 'DoesNotExist' exception is
                  thrown.
                - similar to 'test -d' as defined by POSIX.
 
 
    - is_entry
      Purpose:  tests name for being an ns_entry
      Format:   is_entry           (in  url       name,
                                    out boolean   test);
      Inputs:   name:               name to be tested
      InOuts:   -
      Outputs:  test:               boolean indicating if name
                                    is a non-directory entry
      PreCond:  -
      PostCond: -
      Perms:    Query for name.
                Exec  for name's parent directory.
                Read  for name's parent directory.
      Throws:   NotImplemented
                IncorrectURL
                BadParameter
                DoesNotExist
                IncorrectState
                PermissionDenied
                AuthorizationFailed
                AuthenticationFailed
                Timeout
                NoSuccess
!     Notes:    - all notes to the ns_entry::is_entry() method 
                  apply.
                - if 'name' can be parsed as URL, but contains 
                  an invalid entry name, a 'BadParameter'
                  exception is thrown.
                - if 'name' is a valid entry name but the entry
                  does not exist, a 'DoesNotExist' exception is
                  thrown.
                - similar to 'test -f' as defined by POSIX.
 
 
    - is_link
      Purpose:  tests name for being a symbolic link
      Format:   is_link            (in  url       name,
                                    out boolean   test);
      Inputs:   name:               name to be tested
      InOuts:   -
      Outputs:  test:               boolean indicating if name
                                    is a link
      PreCond:  -
      PostCond: -
      Perms:    Query for name.
                Exec  for name's parent directory.
                Read  for name's parent directory.
      Throws:   NotImplemented
                IncorrectURL
                BadParameter
                IncorrectState
                DoesNotExist
                PermissionDenied
                AuthorizationFailed
                AuthenticationFailed
                Timeout
                NoSuccess
!     Notes:    - all notes to the ns_entry::is_link() method 
                  apply.
                - if 'name' can be parsed as URL, but contains 
                  an invalid entry name, a 'BadParameter'
                  exception is thrown.
                - if 'name' is a valid entry name but the entry
                  does not exist, a 'DoesNotExist' exception is
                  thrown.
                - similar to 'test -L' as defined by POSIX.
 
 
    - read_link 
      Purpose:  returns the name of the link target
      Format:   read_link          (in  url       name,
                                    out url       link);
      Inputs:   name:               name to be resolved
      InOuts:   -
      Outputs:  link:               resolved name
      PreCond:  -
      PostCond: -
      Perms:    Query for name.
                Exec  for name's parent directory.
                Read  for name's parent directory.
      Throws:   NotImplemented
                IncorrectURL
                BadParameter
                DoesNotExist
                IncorrectState
                PermissionDenied
                AuthorizationFailed
                AuthenticationFailed
                Timeout
                NoSuccess
      Notes:    - all notes to ns_entry::read_link() apply
                - if 'name' can be parsed as URL, but contains 
                  an invalid entry name, a 'BadParameter'
                  exception is thrown.
                - if 'name' does not exist, a 'DoesNotExist'
                  exception is thrown.
 
+
+   - get_mtime
+     Purpose:  returns the last modification time
+     Format:   get_time           (in  url       name,
+                                   out int time);
+     Inputs:   name:               name to be checked
+     InOuts:   -
+     Outputs:  time:               time of last modification
+     PreCond:  -
+     PostCond: -
+     Perms:    Query for name.
+               Query for name's parent directory.
+     Throws:   NotImplemented
+               IncorrectURL
+               BadParameter
+               DoesNotExist
+               IncorrectState
+               PermissionDenied
+               AuthorizationFailed
+               AuthenticationFailed
+               Timeout
+               NoSuccess
+     Notes:    - all notes to ns_entry::get_mtime() apply.
+               - if 'name' can be parsed as URL, but contains 
+                 an invalid entry name, a 'BadParameter'
+                 exception is thrown.
+               - if 'name' does not exist, a 'DoesNotExist'
+                 exception is thrown.
+
 
 
    Iterate over large directories:
    -------------------------------
 
    - get_num_entries
      Purpose:  gives the number of entries in the directory
      Format:   get_num_entries    (out int   num);
      Inputs:   -
      InOuts:   -
      Outputs:  num:                number of entries in the
                                    directory
      PreCond:  -
      PostCond: -
      Perms:    Query for cwd.
                Exec  for cwd.
                Read  for cwd.
      Throws:   NotImplemented
                IncorrectState
                PermissionDenied
                AuthorizationFailed
                AuthenticationFailed
                Timeout
                NoSuccess
      Notes:    - at the time of using the result of this call,
                  the actual number of entries may already have
                  changed (no locking is implied)
                - vaguely similar to 'opendir'/'readdir' (2) as
                  defined by POSIX.
 
 
    - get_entry 
      Purpose:  gives the name of an entry in the directory
                based upon the enumeration defined by
                get_num_entries
      Format:   get_entry          (in  int       entry,
                                    out url       name);
      Inputs:   entry:              index of entry to get
      InOuts:   -
      Outputs:  name:               name of entry at index
      PreCond:  -
      PostCond: -
      Perms:    Query for cwd.
                Exec  for cwd.
                Read  for cwd.
      Throws:   NotImplemented
                IncorrectState
                DoesNotExist
                PermissionDenied
                AuthorizationFailed
                AuthenticationFailed
                Timeout
                NoSuccess
      Notes:    - '0' is the first entry
                - there is no sort order implied by the
                  enumeration, however an underlying
                  implementation MAY choose to sort the entries
                - subsequent calls to get_entry and/or
                  get_num_entries may return inconsistent data,
                  i.e. no locking or state tracking is implied.
                  In particular, an index may be invalid - a
                  'DoesNotExist' exception is then thrown (not a
                  'BadParameter' exception).
                - vaguely similar to 'opendir'/'readdir' (2) as
                  defined by POSIX.
 
 
    Management of name space entries:
    ---------------------------------
 
    - copy
      Purpose:  copy the entry to another part of the name space
      Format:   copy               (in  url        source,
                                    in  url        target,
                                    in  int     flags = None);
      Inputs:   source:             name to copy
                target:             name to copy to
                flags:              flags defining the operation
                                    modus
      InOuts:   -
      Outputs:  -
      PreCond:  -
      PostCond: - an identical copy of source exists at target.
                - 'Owner' of target is the id of the context
                  used to perform the operation if target gets
                  created.
      Perms:    Query for source.
                Exec  for source's parent directory.
                Query for target.
                Query for target's parent directory.
                Exec  for target's parent directory.
                Write for target
                      if  target does exist.
                Write for target's parent directory 
                      if  target does not exist.
      Throws:   NotImplemented
                IncorrectURL
                BadParameter
                AlreadyExists
                DoesNotExist
                IncorrectState
                PermissionDenied
                AuthorizationFailed
                AuthenticationFailed
                Timeout
                NoSuccess
      Notes:    - all notes to the ns_entry::copy() method
                  apply.
                - the default flags are 'None' (0).
!               - if 'source' or 'target' can be parsed as URL, 
                  but contain an invalid entry name, a 
                  'BadParameter' exception is thrown.
!               - if 'source' or 'target' are valid entry names 
                  but the entry does not exist, a 'DoesNotExist'
                  exception is thrown.
 
 
    - link
      Purpose:  create a symbolic link from the target entry to
                the source entry so that any reference to the
                target refers to the source entry
      Format:   link               (in  url        source,
                                    in  url        target,
                                    in  int     flags = None);
      Inputs:   source:             name to link
                target:             name to link to
                flags:              flags defining the operation
                                    modus
      InOuts:   -
      Outputs:  -
      PreCond:  -
      PostCond: - a symbolic link to source exists at target.
                - 'Owner' of target is the id of the context
                  used to perform the operation if target gets
                  created.
      Perms:    Query for source.
                Exec  for source's parent directory.
                Query for target.
                Query for target's parent directory.
                Exec  for target's parent directory.
                Write for target
                      if  target does exist.
                Write for target's parent directory 
                      if  target does not exist.
      Throws:   NotImplemented
                IncorrectURL
                BadParameter
                AlreadyExists
                DoesNotExist
                IncorrectState
                PermissionDenied
                AuthorizationFailed
                AuthenticationFailed
                Timeout
                NoSuccess
      Notes:    - all notes to the ns_entry::link() method
                  apply.
-               - if the 'Recursive' flag is defined, the source
-                 is recursively linked if it is a directory;
-                 otherwise this flag is ignored.
-               - if the 'Dereference' flag is specified, the
-                 method applies to the link target of source.
-                 The flag causes a 'BadParameter' exception if
-                 source is not a link.
-               - if the the target already exists, the 
-                 'Overwrite' flag must be specified, otherwise
-                 an 'AlreadyExists' exception is thrown.
                - the default flags are 'None' (0).
                - other flags are not allowed on this method, 
                  and cause a 'BadParameter' exception.
                - if 'source' can be parsed as URL, but contains 
                  an invalid entry name, a 'BadParameter'
                  exception is thrown.
                - if 'source' is a valid entry name but the entry
                  does not exist, a 'DoesNotExist' exception is
                  thrown.
 
 
    - move
      Purpose:  rename source to target, or move source to
                target if target is a directory.
      Format:   move               (in  url        source,
                                    in  url        target,
                                    in  int     flags = None);
      Inputs:   source:             name to move
                target:             name to move to
                flags:              flags defining the operation
                                    modus
      InOuts:   -
      Outputs:  -
      PreCond:  -
!     PostCond: - the entry exists at the target url.
!               - no entry exists at the original url.
+               - the object instance is not closed.
                - 'Owner' of target is the id of the context
                  used to perform the operation if target gets
                  created.
      Perms:    Query for source.
                Write for source.
                Exec  for source's parent directory.
                Write for source's parent directory.
                Query for target.
                Exec  for target's parent directory.
                Write for target
                      if  target does exist.
                Write for target's parent directory 
                      if  target does not exist.
      Throws:   NotImplemented
                IncorrectURL
                BadParameter
                AlreadyExists
                DoesNotExist
                IncorrectState
                PermissionDenied
                AuthorizationFailed
                AuthenticationFailed
                Timeout
                NoSuccess
      Notes:    - all notes to the ns_entry::move() method
                  apply.
 -              - if the 'Recursive' flag is defined, the source
 -                is recursively copied if it is a directory;
 -                otherwise this flag is ignored.
 -              - if the 'Dereference' flag is specified, the
 -                method applies to the link target of source.
 -                The flag causes a 'BadParameter' exception if
 -                source is not a link.
 -              - if the the target already exists, the 
 -                'Overwrite' flag must be specified, otherwise
 -                an 'AlreadyExists' exception is thrown.
                - the default flags are 'None' (0).
                - other flags are not allowed on this method, 
                  and cause a 'BadParameter' exception.
                - if 'source' can be parsed as URL, but contains 
                  an invalid entry name, a 'BadParameter'
                  exception is thrown.
                - if 'source' is a valid entry name but the entry
                  does not exist, a 'DoesNotExist' exception is
                  thrown.
                - moving any parent or the current directory 
                  (e.g.  '.', '..' etc.) is not allowed, and 
                  throws a 'BadParameter' exception
 
 
    - remove
      Purpose:  removes the entry
      Format:   remove             (in  url        target,
                                    in  int     flags = None);
      Inputs:   target:             entry to be removed
      InOuts:   -
      Outputs:  -
      PreCond:  -
!     PostCond: - target is removed.
!               - target is closed if it refers to the cwd.
!     Perms:    Query for target.
!               Write for target.
!               Exec  for target's parent directory.
!               Write for target's parent directory.
      Throws:   NotImplemented
                IncorrectURL
                BadParameter
                AlreadyExists
                DoesNotExist
                IncorrectState
                PermissionDenied
                AuthorizationFailed
                AuthenticationFailed
                Timeout
                NoSuccess
      Notes:    - all notes to the ns_entry::remove() method
                  apply.
-               - if the 'Recursive' flag is defined, the target
-                 is recursively removed if it is a directory;
-                 otherwise this flag is ignored.
-               - if the 'Dereference' flag is specified, the
-                 method applies to the link target of target.
-                 The flag causes a 'BadParameter' exception if
-                 target is not a link.
                - the default flags are 'None' (0).
                - other flags are not allowed on this method, 
                  and cause a 'BadParameter' exception.
!               - if 'target' can be parsed as URL, but contains 
                  an invalid entry name, a 'BadParameter'
                  exception is thrown.
!               - if 'target' is a valid entry name but the entry
                  does not exist, a 'DoesNotExist' exception is
                  thrown.
+               - if the instance was not closed before, this
+                 call performs a close() on the instance, and 
+                 all notes to close() apply.
                - removing any parent or the current directory 
                  (e.g.  '.', '..' etc.) is not allowed, and 
                  throws a 'BadParameter' exception
 
 
    - make_dir
      Purpose:  creates a new directory
      Format:   make_dir           (in  url        target,
                                    in  int     flags = None);
      Inputs:   target:             directory to create
      InOuts:   -
      Outputs:  -
      PreCond:  -
      PostCond: - 'Owner' of target is the id of the context
                  used to perform the operation if target gets
                  created.
      Perms:    Exec  for target's parent directory.
                Write for target's parent directory.
                Write for target if Write is set.
                Read  for target if Read  is set.
      Throws:   NotImplemented
                IncorrectURL
                BadParameter
                AlreadyExists
                DoesNotExist
                IncorrectState
                PermissionDenied
                AuthorizationFailed
                AuthenticationFailed
                Timeout
                NoSuccess
      Notes:    - if the parent directory or directories do not
                  exist, the 'CreateParents' flag must be set 
                  or a 'DoesNotExist' exception is thrown.  
                  If set, the parent directories are created as 
                  well.
                - an 'AlreadyExists' exception is thrown if the
                  directory already exists and the 'Exclusive' 
                  flag is given.
                - the default flags are 'None' (0).
                - other flags are not allowed on this method, 
                  and cause a 'BadParameter' exception.
                - if 'target' can be parsed as URL, but contains 
                  an invalid entry name, a 'BadParameter'
                  exception is thrown.
                - similar to 'mkdir' (2) as defined by POSIX.
 
 
    - open_dir
      Purpose:  creates a new ns_directory instance
      Format:   open_dir           (in  url       name,
!                                   in  int       flags = Read,
                                    out ns_directory dir);
      Inputs:   name:               directory to open
                flags:              flags defining the operation
                                    modus
      InOuts:   -
      Outputs:  dir:                opened directory instance
      PreCond:  -
      PostCond: - the session of the returned instance is that of
                  the calling instance.
                - 'Owner' of name is the id of the context
                  used to perform the operation if name gets
                  created.
                - the namespace directory is created if it
                  does not yet exist, and the Create is set.
      Perms:    Exec  for name's parent directory.
                Write for name's parent directory if Create is set.
                Write for name if Write is set.
                Read  for name if Read  is set.
      Throws:   NotImplemented
                IncorrectURL
                BadParameter
                AlreadyExists
                DoesNotExist
                IncorrectState
                PermissionDenied
                AuthorizationFailed
                AuthenticationFailed
                Timeout
                NoSuccess
      Notes:    - the cwd of the new dir object instance is set
                  to 'name'
                - a 'DoesNotExist' exception is thrown if 'name'
                  does not exist and the 'Create' flag is not
                  given.
                - a 'AlreadyExist' exception is thrown if 'name'
                  does exist and the 'Create' flag and the
                 'Exclusive' flag are given.
                - no exception is thrown if 'name' does exist and 
                  the 'Create' flag is given, and the 'Exclusive' 
                  flag is not given.
                - if the 'Create' flag is given, all notes to the
                  ns_directory::make_dir() method apply.
!               - the default flag set is 'Read'.
!               - the flags 'Overwrite', 'Recursive' and
!                 'Dereference' are not allowed on this method, 
!                 and cause a 'BadParameter' exception.
                - 'name' is always deeply dereferenced, however,
                  the cwd is still set to 'name', and not to the
                  value of the link target.
                - parent directories are created on the fly if
                  the 'CreateParents' and 'Create' flag are both
                  given, if they don't exist.
                - if 'name' can be parsed as URL, but contains 
                  an invalid directory name, a 'BadParameter'
                  exception is thrown.
 
 
    - open
      Purpose:  creates a new ns_entry instance
      Format:   open               (in  url       name,
!                                   in  int       flags = Read,
                                    out ns_entry  entry);
      Inputs:   name:               entry
                flags:              flags defining the operation
                                    modus
      InOuts:   -
      Outputs:  entry:              opened entry instance
      PreCond:  -
      PostCond: - the session of the returned instance is that
                  of the calling instance.
                - 'Owner' of name is the id of the context
                  used to perform the operation if name gets
                  created.
                - the namespace entry is created if it does not 
                  yet exist, and the CREATE flag is specified.
      Perms:    Exec  for name's parent directory.
                Write for name's parent directory if Create is set.
                Write for name if Write is set.
                Read  for name if Read  is set.
      Throws:   NotImplemented
                IncorrectURL
                BadParameter
                AlreadyExists
                DoesNotExist
                IncorrectState
                PermissionDenied
                AuthorizationFailed
                AuthenticationFailed
                Timeout
                NoSuccess
      Notes:    - a 'BadParameter' exception is thrown if 'name'
!                 is an invalid entry name.
                - a 'DoesNotExist' exception is thrown if 'name'
                  does not exist, and the 'Create' flag is not
                  given.
                - a 'AlreadyExists' exception is thrown if 'name'
                  does exist, and the 'Create' and 'Exclusive' 
                  flags are given.
                - 'name' is always deeply dereferenced, the cwd,
                  however, is not changed to the link targets
                  cwd.
                - parent directories are created on the fly if
                  the 'CreateParents' and 'Create' flag are both
                  given, if they don't exist.
                - the entry is locked on open if the 'Lock' flag
                  is given.  If the entry is already in a locked
                  state, the open will fail and a descriptive
                  error will be issued.  If a entry is opened in
                  locked mode, any other open on that entry MUST
                  fail with a 'NoSuccess' exception if the 'Lock'
                  flag is given.  Note that a entry can be opened
                  in unlocked mode, and then in locked mode,
                  without an error getting raised.  The
                  application programmer must take precautions
                  to avoid such situations.  The lock will get
                  removed on destruction of the entry object, and
                  also on close.  If an implementation does not
                  support locking, a descriptive 'BadParameter'
                  exception MUST get thrown if the 'Lock' flag 
                  is given.  Read-locks and Write-locks are not
                  distinguished.
!               - the default flag set is 'Read'.
!               - the flags 'Recursive' and 'Dereference' are 
!                 not allowed on this method, and cause a 
!                 'BadParameter' exception.
                - similar to 'open' (2) as defined by POSIX.
 
 
    Management of name space entries - wildcard versions:
    -----------------------------------------------------
 
    - copy
      Purpose:  copy the entry to another part of the name space
      Format:   copy               (in  string     source,
                                    in  url        target,
                                    in  int        flags = None);
      Notes:    - the syntax and semantics of this call is 
                  identical to its URL based counterpart.  
                - the 'source' string can contain wildcards, as
                  described above.
                - on error conditions on any of the expanded 
                  list of source entries, the respective error 
                  described in the URL version of the method is 
                  thrown - the state of the operations on the 
                  other elements of the expanded entry list is 
                  undefined.
                - if source expands to multiple entries, then the
                  target URL MUST specify a directory -
                  otherwise a 'BadParameter' exception is thrown.
 
 
    - link
      Purpose:  create a symbolic link from the target entry to
                the source entry so that any reference to the
                target refers to the source entry
      Format:   link               (in  string     source,
                                    in  url        target,
                                    in  int        flags = None);
      Notes:    - the syntax and semantics of this call is 
                  identical to its URL based counterpart.  
                - the 'source' string can contain wildcards, as
                  described above.
                - on error conditions on any of the expanded 
                  list of source entries, the respective error 
                  described in the URL version of the method is 
                  thrown - the state of the operations on the 
                  other elements of the expanded entry list is 
                  undefined.
                - if source expands to multiple entries, then the
                  target URL MUST specify a directory -
                  otherwise a 'BadParameter' exception is thrown.
 
    - move
      Purpose:  moves sources to a target directory.
      Format:   move               (in  string     source,
                                    in  url        target,
                                    in  int        flags = None);
      Notes:    - the syntax and semantics of this call is 
                  identical to its URL based counterpart.  
                - the 'source' string can contain wildcards, as
                  described above.
                - on error conditions on any of the expanded 
                  list of source entries, the respective error 
                  described in the URL version of the method is 
                  thrown - the state of the operations on the 
                  other elements of the expanded entry list is 
                  undefined.
                - if source expands to multiple entries, then the
                  target URL MUST specify a directory -
                  otherwise a 'BadParameter' exception is thrown.
 
    - remove
      Purpose:  removes entries
      Format:   remove             (in  string  target,
                                    in  int     flags = None);
      Notes:    - the syntax and semantics of this call is 
                  identical to its URL based counterpart.  
!               - the 'target' string can contain wildcards, as
                  described above.
                - on error conditions on any of the expanded 
!                 list of target entries, the respective error 
                  described in the URL version of the method is 
                  thrown - the state of the operations on the 
                  other elements of the expanded entry list is 
                  undefined.
 
 
    // overload permissions because of namespace specific flags
 
    - permissions_allow
      Purpose:  enable a permission 
      Format:   permissions_allow    (in  url       target, 
                                      in  string    id,
                                      in  int       perm, 
                                      in  int       flags = None);
      Inputs:   target:               entry to set permissions for
                id:                   id to set permission for
                perm:                 permission to enable
                flags:                mode of operation
      InOuts:   -
      Outputs:  -
      PreCond:  -
      PostCond: - the permissions are enabled.
      Perms:    Owner of target
      Throws:   NotImplemented
                IncorrectURL
                BadParameter
                IncorrectState
                PermissionDenied
                AuthorizationFailed
                AuthenticationFailed
                Timeout
                NoSuccess
      Notes:    - all notes to permissions_allow from the
                  saga::permissions interface apply.
                - allowed flags are: 'Recursive', 'Dereference'.
                  All other flags cause a 'BadParameter'
                  exception.
                - specifying 'Recursive' for a non-directory
                  causes a 'BadParameter' exception.
 
 
    - permissions_deny
      Purpose:  disable a permission flag
      Format:   permissions_deny     (in  url       target, 
                                      in  string    id,
                                      in  int       perm,
                                      in  int       flags = None);
      Inputs:   target:               entry to set permissions for
                id:                   id to set permission for
                perm:                 permission to disable
                flags:                mode of operation
      InOuts:   -
      Outputs:  -
      PreCond:  -
      PostCond: - the permissions are disabled.
      Perms:    Owner of target
      Throws:   NotImplemented
                IncorrectURL
                BadParameter
                IncorrectState
                PermissionDenied
                AuthorizationFailed
                AuthenticationFailed
                Timeout
                NoSuccess
      Notes:    - all notes to permissions_deny from the
                  saga::permissions interface apply.
                - allowed flags are: 'Recursive', 'Dereference'.
                  All other flags cause a 'BadParameter'
                  exception.
                - specifying 'Recursive' for a non-directory
                  causes a 'BadParameter' exception.
 
 
    // permissions calls - wildcard versions
 
    - permissions_allow
      Purpose:  enable a permission 
      Format:   permissions_allow    (in  string  target, 
                                      in  string  id,
                                      in  int     perm, 
                                      in  int     flags = None);
      Notes:    - the syntax and semantics of this call is 
                  identical to its URL based counterpart.  
                - the 'source' string can contain wildcards, as
                  described above.
                - on error conditions on any of the expanded 
                  list of source entries, the respective error 
                  described in the URL version of the method is 
                  thrown - the state of the operations on the 
                  other elements of the expanded entry list is 
                  undefined.
 
 
    - permissions_deny
      Purpose:  disable a permission flag
      Format:   permissions_deny     (in  string  target, 
                                      in  string  id,
                                      in  int     perm,
                                      in  int     flags = None);
      Notes:    - the syntax and semantics of this call is 
                  identical to its URL based counterpart.  
                - the 'source' string can contain wildcards, as
                  described above.
                - on error conditions on any of the expanded 
                  list of source entries, the respective error 
                  described in the URL version of the method is 
                  thrown - the state of the operations on the 
                  other elements of the expanded entry list is 
                  undefined.
 \end{myspec}
 
 
 \subsubsection{Examples:}
 
 \begin{mycode}
  More examples are given in the File and Logical_File packages.
 
  Example: provide recursive directory listing for a given
           directory
 
  Note:    - check for '.' and '..' recursion are left as an
             exercise to the reader.
           - string operations and printf statements are
             obviously simplified.
 
  +-------------------------------------------------------------+
   // c++ example
   std::string indent (int indent)
   {
     std::string s = "  ";
 
     for (int i = 0; i < indent; i++, s += "  ");
 
     return (s);
   }
 
   void list_dir (saga::url url,
                  int       indent = 0)
   {
     try
     {
       // create directory and iterate over entries
       saga::ns_dir dir (url);
 
       printf ("\n%s ---> %s\n", indent (indent), url.get_url ());
 
       for ( int i = 0; i < dir.get_num_entries (); i++ )
       {
         char   type = '?';
         string info = "";
 
         // get name of next entry
         saga::url name = dir.get_entry (i);
 
         // get type and other info
         if ( dir.is_link (name) )
         {
           // check where link points to
           if (dir.exists(dir.read_link (name))){info=" ---> ";}
           else                                 {info=" -|-> ";}
           info += dir.read_link (name);
           type  = 'l';
         }
         else if (dir.is_entry(name)){ type = 'f';            }
         else if (dir.is_dir  (name)){ type = 'd'; info = "/";}
 
         printf ("%s > %3d - %s - %s%s\n",
                 indent (indent), i + 1,
                 type, name.get_cstr (), info);
 
         // recursion on directories
         if ( dir.is_dir (name) )
         {
           list_dir (name, indent++);
         }
       }
 
       printf ("\n%s <--- %s\n", indent (indent), url.get_url ());
     }
 
     // catch all errors - see elsewhere for better examples
     // of error handling in SAGA
     catch ( const saga::exception & e )
     {
       std::cerr << "Oops! SAGA exception: "
                 << e.get_message () 
                 << std::endl;
     }
 
     return;
   }
 \end{mycode}
 
 
 
 
 
