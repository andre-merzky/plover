  
  Operations performed in highly heterogeneous distributed
  environments may take a long time to complete, and it is thus
  desirable to have the ability to perform operations in an
  asynchronous manner.  The SAGA task model as described here,
  provides this ability to all other SAGA classes.  As such, the
  package is orthogonal to the rest of the SAGA API.
  
  % Jha: do we mean independent? Is this package independent of
  % the other packages or are the other packages independent of
  % this package? If so, we should say so explicitly and not as
  % said currently?
  % 
  % AM: Orthogonal means really that: think of two dimensions
  % for the API, X and Y.  Extending the API in X means adding
  % new classes, extending the API in Y means adding new async
  % models or calls.  Any changes in the task model apply
  % immediately to all classes: these dimensions are orthogonal.
  % We use that term frequently to describe architecture
  % properties, to describe what a property means for other
  % elements: if it is 'orthogonal', it means its implicitly
  % available immediately.  I am not sure how widely used the
  % term is, but I have seen it more often than not.
  
  \myfig{task_states}{\footnotesize The SAGA task state model
  (See figure~\ref{fig:states} for a legend).}
  
  In order to understand the SAGA task model it is \I{not}
  sufficient to read the specification of the |saga::task| and
  |saga::task_container| classes below, but it is also
  imperative to understand how task instances get created.  This
  is actually not covered in the SIDL specification sections in
  this document, but documented in prose below, with references
  to Figure~\ref{fig:task_states}.  Note that the task state
  model is closely modeled after the BES state model~\cite{bes},
  which is in particular relevant to the (similar) job state
  model as described in Section~\ref{ssec:job}.
 
  \subsubsection*{Tasks versus Jobs}
 
   In SAGA, tasks should not be confused with jobs!
   Jobs represent remotely running applications/executables,
   which are usually managed by a job manager.  Tasks on the
   other hand represent asynchronous operations.  Thus, any
   asynchronous method call in SAGA results in a task.
  
   Tasks and jobs have, however, several commonalities,
   the most important one is \I{state}: both can be newly
   created (in \T{New} state), can be currently making progress
   (in \T{Running} state), or can be finished in some way (in
   \T{Done}, \T{Failed} or \T{Canceled} state).  Additionally,
   jobs can be suspended and resumed (they have a \T{Suspended}
   state).
  
   Mostly for this reason, and to simplify the
   management of both tasks and jobs in SAGA, the \T{saga::job}
   class inherits the \T{saga::task} class.
 
  \subsubsection*{Tasks versus Threads}
 
   Tasks and threads are another potential pair to
   confuse: in many APIs and programming languages, tasks and
   asynchronous operations are implemented by threading.  In
   SAGA, however, tasks have a semantically richer meaning.  In
   particular, threads always imply that the state management
   for the asynchronous operation lies within the application
   hosting the thread.  SAGA tasks, however, imply no such
   restriction.
   
   For example, a SAGA task to copy a remote file
   could be implemented by using the Globus Reliable File
   Transfer Service (RFT,~\cite{rft}): the asynchronous method
   invocation in SAGA would then start the remote operation on
   the RFT service.  All management of the operation progress is
   in the service - no threading at all is required on the
   application side.  Even more: the application could finish,
   and after restart could reconnect to the RFT service, and
   recreate the task, as the complete state is still available
   on the RFT service - that is basically impossible with
   threads.  Well, it is also not possible in SAGA right now,
   but for very different reasons, and it is expected that
   future versions and extensions of SAGA add this and other
   options to the notion of tasks.
 
   Implementors of SAGA are warned \B{not} to rely
   solely on threading while implementing \T{saga::task}, but to
   exploit middleware support for server side asynchronous
   operations wherever possible.
 
 
  \subsubsection*{Task Model Description}
 
  The SAGA task model operates as follows:
 
 \begin{itemize}
   
   \item A SAGA object is said to \I{implement the SAGA task
   model} if, (a) it inherits the |saga::async| interface, and
   (b) all methods on that object are implemented in three
   different versions, which are called \I{synchronous},
   \I{asynchronous}, and \I{task} version.
 
   \item The \I{synchronous} version of a SAGA call corresponds
   to the normal method call specified in the SAGA
   specification.  The first |out| parameter specified (if any)
   is used as return value.
 
   \item The \I{asynchronous} version of a SAGA
   call has the same signature, but 
   returns a
   |saga::task| instance.  That returned task is in |Running|
   state and represents the asynchronous operation: it can be
   queried for state, and can be canceled.
 
   \item The \I{task} version of the SAGA call is very similar to
   the asynchronous version; the only difference is that the
   returned task instance is in the |New| state, and must
   be |run()| to get into the |Running| state.
 
   \item For symmetry, a language binding MAY add a second
   flavour of the synchronous call, which has the same signature
   as the asynchronous and task version, but the
   returned task is in a final state (i.e., |run()| \I{and}
   |wait()| have been called on that task before returning).
   \footnote{Note that state transitions for this type of method
   call are not shown in the state diagram -- the diagram would
   essentially have 'Done' as an initial
   and final state.}
 
   \item The first \T{out} parameter, which is the
   return value in the synchronous method version, is, in the
   task and asynchronous version, accessed by calling
   \T{task.get\_result <return\_type> (void);}, which is a
   templetized member method.  That call implies a call to
   \T{wait()}.  For language bindings where templetized member
   functions are not available, a language specific mechanism
   MUST be found, which MAY use type casting.
 
   \item Other |out| and all |inout| parameter for
   asynchronous operations are passed by reference to the
   initial function call, and MUST NOT be accessed before the
   corresponding task enters the \T{Done} state.  In all
   other states, no assumption can be made about the contents of
   these parameters.  They are guaranteed to not be
   accessed or changed by the implementation when the task
   enters any final state.
   
   \item |in| parameters are passed by value, and are assumed to
   be constant.  They can be accessed and changed again as soon
   as the task instance is created.
 
   \item The original object instance, from which the task was
   created, can be retrieved from a task by calling
   \T{get\_object <object\_type> (void);}, again a templetized
   member method, on the task.  The same comments as above apply
   to that templetized method.
 
 \end{itemize}
 
 
 \subsubsection*{Asynchronous Object Construction}
 
  The task model as described above focuses on
  asynchronous invocation of object methods.  It does not
  explicitly cover asynchronous object construction or
  destruction though.  That is important, however, as many
  constructors, such as for example for \T{saga::file}, imply a
  remote operation during construction or destruction (here
  \T{open()/close()}).
  
  How asynchronous constructors and destructors are
  provided is up to the specific language bindings.  Procedural
  bindings, such as expected for C, SHOULD integrate
  asynchronous versions for the respective method calls to keep
  these mechanisms in sync with the task model presented above.
  Object oriented language bindings MAY either introduce an
  asynchronous factory pattern, or introduce delayed
  construction/destruction by explicitly using asynchronous
  \T{init()} and \T{close()}, or MAY introduce some other
  mechanism which most natively allows to asynchronously create
  SAGA objects.
 
 
 \subsubsection*{Tasks and Error Handling}
 
  Errors arising from synchronous method invocations on SAGA
  objects are, in general, flagged by exceptions, and can be
  inspected using the \T{error\_hand\-ler} interface that all
  SAGA objects implement.  For asynchronous operations, this
  mechanism would break, as the |error_handler| interface allows
  in general only inspection of the \I{last} method call
  -- but the order of execution is undefined for asynchronous
  operations.  Additionally, exceptions from asynchronous
  operations would be difficult to catch, as they would
  presumably be thrown outside of an exception
  protection block.
 
  For this reason, errors on asynchronous operations (i.e.
  tasks) are handled as follows:
 
  \begin{description}
   
   \item[Error Handler:] The |saga::task| class implements the
   |saga::error_handler| interface, which allows inspection of
   an error thrown by an asynchronous operation.  Errors MUST
   NOT be reported unless the task enters a final state.
     
   \item[Exceptions:] The task instance MUST catch all SAGA
   exceptions and, if possible, all other exceptions thrown by
   the asynchronous operation.  If an exception is caught by the
   task instance, the task state MUST be changed to |Failed|
   immediately.  Such exceptions are to be re-thrown by the task
   when the |rethrow()| method is called.
 
  \end{description}
 
  This specification assumes that tasks are, in general, created
  and maintained in the API implementation, and not in the
  backend.  However, for those cases where task states are
  maintained in the middleware backend, several methods on
  |task|s and |task_containers| MAY throw a |Timeout| or
  |NoSuccess| exception, if that backend is not available
   -- these exceptions can be directly delivered to the
  application.  It is, however, not allowed to throw an
  |AuthorizationFailed|, |AuthenticationFailed| or
  |PermissionDenied| exception, as this specification
  assumes that the creator of the task can always inspect and
  control that task  -- these exceptions MUST be
  caught, and MUST be made available via \T{rethrow()}.  Later
  versions of this API MAY change that, for example when they
  introduce persistent tasks which can survive the lifetime of a
  SAGA application.
 
 \subsubsection{Example Rendering in C++}
 
  Below is an example of how the SAGA task model might be
  rendered in C++ (this example is not normative).  Note that
  template-tags are used here to distinguish the three
  task-returning method calls.
 
 \begin{mycode}
  // c++ example
 
  // SAGA specification:
  //  read        (inout array<byte>             buffer,
  //               in    int                     len_in = -1,
  //               out   int                     len_out);
 
  // create a saga file
  saga::file f (url);
 
  // synchronous version
  ssize_t len_out = f.read (size_t   len_in, 
                            char   * buffer);
                             
 
  // alternative synchronous version
  saga::task t1   = f.read <saga::task::Sync>
                           (size_t    len_in,
                            char    * buffer);
 
  // asynchronous version
  saga::task t2   = f.read <saga::task::ASync>
                           (size_t    len_in, 
                            char    * buffer);
 
  // task version
  saga::task t3   = f.read <saga::task::Task>
                           (size_t    len_in,
                            char    * buffer);
 
  // t1 is in Done or Failed state
  // t2 is in Running state
  // t3 is in New state
 
  // get results
  ssize_t len_out_1 = t1.get_result <ssize_t> ();
  ssize_t len_out_2 = t2.get_result <ssize_t> ();
  ssize_t len_out_3 = t3.get_result <ssize_t> ();
 
  // all tasks are in a final state now, 
  // as get_result() implies a wait().
 
  // obtain the original file object, three 
  // times the same actually
  saga::file f1 = t1.get_object <saga::file> ();
  saga::file f2 = t2.get_object <saga::file> ();
  saga::file f3 = t3.get_object <saga::file> ();
 \end{mycode}
 
  A C language binding of this package might choose to use flags
  to distinguish between these calls;
  equivalently the C binding might
  use different method names, for it is up to the language
  bindings to define the mechanism that is native -- or as close
  as possible -- to the language to distinguish these calls.
 
  For additional notes on resource management and task lifetime,
  see the introduction Section~\ref{ssec:lifetime} of this
  document.
 
 \subsubsection{Specification}
 
 \begin{myspec}
  package saga.core
  {
    enum state
    {
      New       =  1,
      Running   =  2,
      Done      =  3,
      Canceled  =  4,
      Failed    =  5
    }
 
 
    enum wait_mode
    {
      All       =  0,
      Any       =  1
    }
 
 
    interface async
    {
      // this interface is empty on purpose, and is used only
      // for tagging of SAGA classes which implement the SAGA
      // task model.
    }
 
 
    class task : implements   saga::object
                 implements   saga::monitorable
              // from object  saga::error_handler
    {
      // no constructor
      DESTRUCTOR       (in  task            obj);
 
      // state management
      run               (void);
      cancel            (in  float           timeout =  0.0);
      wait              (in  float           timeout = -1.0,
                         out boolean         finished);
 
      // inspection
      get_state         (out state           state);
      get_result <type> (out type            result);
      get_object <type> (out type            object);
 
      // error handling
      rethrow           (void);
 
      // Metric:
      //   name:  task.state
      //   desc:  fires on task state change, and
      //          has the literal value of the task
      //          state enum.
      //   mode:  ReadOnly
      //   unit:  1
      //   type:  Enum
      //   value: 0
    }
 
 
    class task_container : implements   saga::object
                           implements   saga::monitorable
                        // from object  saga::error_handler
    {
      CONSTRUCTOR      (out task_container  obj);
      DESTRUCTOR       (in  task_container  obj);
 
      // task management
      add               (in  task            t);
      remove            (in  task            t);
 
      // state management
      run               (void);
      cancel            (in  float           timeout =  0.0);
      wait              (in  wait_mode       mode    =  All,
                         in  float           timeout = -1.0,
                         out task            finished);
 
      // inspection
      size              (out int             n);
      list_tasks        (out array<task>     tasks);
      get_task          (in  string          id, 
                         out task            t);
      get_tasks         (out array<task>     tasks);
      get_states        (out array<state>    states);
 
      // Metric:
      //   name:  task_container.state
      //   desc:  fires on state changes of any task in
      //          the container, and has the value of that
      //          task's object id.
      //   mode:  ReadOnly
      //   unit:  1
      //   type:  String
      //   value: -
    }
  }
 \end{myspec}
 
 
 \subsubsection{Specification Details}
 
 % Jha: ``It is initial'' is it redundant? Or am I missing
 % something.
 % 
 % AM: Initial states are something special - see figure 1.  We
 % probably should include a reference here.  Better yet, the state
 % description should be moved to the top, to the state diagram (where
 % that reference is, and where Initial States are marked)
 % (TODO).
 
  \subsubsection*{Enum \T{state}}
 
  A task can be in one of several possible states (see
  Fig.~\ref{fig:task_states}):
  
 
    |New|\\[1.5mm]
    \begin{tabular}{cp{110mm}}
      ~~ & This state identifies a newly constructed task
           instance which has not yet run.  This state corresponds to
           the BES state 'Pending'.   This state is initial.
    \end{tabular}
 
    |Running|\\[1.5mm]
    \begin{tabular}{cp{110mm}}
       ~~ & The \T{run()} method has been invoked on the task, either
            explicitly or implicitly.  This state corresponds to the
            BES state 'Running'.  This state is initial.  
    \end{tabular}
 
    |Done|\\[1.5mm]
    \begin{tabular}{cp{110mm}}
       ~~ & The synchronous or asynchronous operation has finished
            successfully.  It corresponds to the BES state 'Finished'.
            This state is final.  
    \end{tabular}
 
    |Canceled|\\[1.5mm]
    \begin{tabular}{cp{110mm}}
       ~~ & The asynchronous operation has been canceled, i.e.
            \T{cancel()} has been called on the task instance.  It
            corresponds to the BES state 'Canceled'.  This state is
            final.  
    \end{tabular}
 
    |Failed|\\[1.5mm]
    \begin{tabular}{cp{110mm}}
       ~~ & The synchronous or asynchronous operation has
            finished unsuccessfully.  It corresponds to the BES state
            'Failed'.  This state is final.  
    \end{tabular}
 
 
  \subsubsection*{Enum \T{wait\_mode}}
 
    The \T{wait\_mode} enum specifies the condition on
    which a \T{wait()} operation on a \T{saga::task\_container}
    returns:
 
    |All|\\[1.5mm]
    \begin{tabular}{cp{110mm}}
       ~~ & \T{wait()} returns if all tasks in the 
            container reached a final state.
    \end{tabular}
    
    |Any|\\[1.5mm]
    \begin{tabular}{cp{110mm}}
       ~~ & \T{wait()} returns if one or more tasks in 
            the container reached a final state.
    \end{tabular}
 
  \subsubsection*{Class \T{task}}
 
    Objects of this class represent asynchronous API calls.
    They are only created by invoking a method on a SAGA object
    which returns a task object (with |saga::task::ASync| or
    |saga::task::Task|).  But as |saga::job| instances
    inherit from the task class, jobs are also
    effectively created as tasks.
 
    If a task gets created, it will share the state of the
    object it was created from.  For more information on state
    sharing, see Section~\ref{ssec:lifetime}).
    
    Note that no \T{CONSTRUCTOR} is available, as tasks
    are only created through asynchronous method calls.
 
 \begin{myspec}
    - DESTRUCTOR
      Purpose:  destroy the object
      Format:   DESTRUCTOR           (in  task obj)
      Inputs:   obj:                  the object to destroy
      InOuts:   -
      Outputs:  -
      PreCond:  -
      PostCond: - state is no longer shared with the object
                  the task was created from.
                - the task instance is 'Canceled' prior to 
                  resource deallocation.
      Perms:    -
      Throws:   -
      Notes:    - if the instance was not in a final state 
                  before, the destructor performs a cancel() 
                  on the instance, and all notes to cancel() 
                  apply.
 
    
    State Management
    ----------------
 
    - run
      Purpose:  Start the asynchronous operation.
      Format:   run (void);
      Inputs:   -
      InOuts:   -
      Outputs:  -
      PreCond:  - task is in 'New' state.
      PostCond: - task left the 'New' state.
      Perms:    - appropriate permissions for the method 
                  represented by the task
      Throws:   NotImplemented
                IncorrectState
                Timeout
                NoSuccess
      Notes:    - run can only be called on a task in 'New' 
                  state.  All other states will cause the 
                  'IncorrectState' exception to be thrown.
                - a 'Timeout' or 'NoSuccess' exception indicates
                  that the backend was not able to start the 
                  task.
 
 
    - wait
      Purpose:  Wait for the task to finish.
      Format:   wait                 (in  float   timeout,
                                      out boolean done);
      Inputs:   timeout:              seconds to wait
      InOuts:   -
      Outputs:  done:                 indicating if the task
                                      is done running
      PreCond:  - task is not in 'New' state.
      PostCond: - if no timeout occurs, task is in a final 
                  state.
      Perms:    -
      Throws:   NotImplemented
                IncorrectState
                Timeout
                NoSuccess
      Notes:    - wait returns success (true) as soon as the
                  task enters a final state
                - if the task is already in a final state, the
                  call returns success (true) immediately.
                - if the task is in 'New' state, an
                  'IncorrectState' exception is thrown.
                - wait returns no success (false) if the task 
                  is, even after timeout, not in a final state.
                - a 'Timeout' or 'NoSuccess' exception indicates
                  that the backend was not able to wait for the
                  task.  Note that a 'Timeout' exception does
                  not indicate that the task is not in a final
                  state after the given wait period - that
                  causes an unsuccessful (false) return value.
                - for timeout semantics, see Section 2.
 
 
    - cancel
      Purpose:  Cancel the asynchronous operation.
      Format:   cancel               (in  float  timeout = 0.0);
      Inputs:   timeout:              time for freeing resources
      InOuts:   -
      Outputs:  -
      PreCond:  - task is in 'Running' state.
      PostCond: - task is in 'Canceled' state.
      Perms:    -
      Throws:   NotImplemented
                IncorrectState
                Timeout
                NoSuccess
      Notes:    - for resource deallocation semantics, see 
                  Section 2.
                - if cancel() fails to cancel the task
                  immediately, and tries to continue to cancel
                  the task in the background, the task state 
                  remains 'Running' until the cancel operation 
                  succeeded.  The state then changes to 
                  'Canceled'.
                - if the task is in a final state, the call has
                  no effect, and, in particular, does NOT change
                  the state from 'Done' to 'Canceled', or from 
                  'Failed' to 'Canceled'.  This is to
                  avoid race conditions.
                - if the task is in 'New' state, an
                  'IncorrectState' exception is thrown.
                - a 'NoSuccess' exception indicates
                  that the backend was not able to initiate the
                  cancellation for the task.
                - for timeout semantics, see Section 2.
 
 
    Inspection
    ----------
 
    - get_state 
      Purpose:  Get the state of the task.
      Format:   get_state            (out state state);
      Inputs:   -
      InOuts:   -
      Outputs:  state:                state of the task.
      PreCond:  -
      PostCond: -
      Perms:    -
      Throws:   NotImplemented
                Timeout
                NoSuccess
      Notes:    - a 'Timeout' or 'NoSuccess' exception indicates
                  that the backend was not able to retrieve the
                  task state.
 
 
    - get_result 
      Purpose:  Get the result of the async operation
      Format:   get_result <type>    (out type result);
      Inputs:   -
      InOuts:   -
      Outputs:  result:               return value of async
                                      method
      PreCond:  - task is not in New, Failed or Canceled state.
      PostCond: - task is in a final state.
      Perms:    -
      Throws:   NotImplemented
+               IncorrectURL
+               BadParameter
+               AlreadyExists
+               DoesNotExist
                IncorrectState
+               IncorrectType
+               PermissionDenied
+               AuthorizationFailed
+               AuthenticationFailed
                Timeout
                NoSuccess
      Notes:    - get_result implies a wait() - all notes to
                  wait apply.
+               - if the task is in 'Failed' state after wait(), a
+                 rethrow() is called.  That is why all possible
+                 exceptions can be thrown by get_result().
+               - an IncorrectType exception is thrown when the 
+                 'type' specifier does not match the return type
+                 of the operation represented by the task
                - the method returns the type and value which
                  would be returned by the synchronous version of
                  the respective function call.
 
 
    - get_object
      Purpose:  Get the object from which this task was created
      Format:   get_object <type>    (out type object);
      Inputs:   -
      InOuts:   -
      Outputs:  object:               object this task was
                                      created from
      PreCond:  -
      PostCond: -
      Perms:    -
      Throws:   NotImplemented
                Timeout
                NoSuccess
      Notes:    - the method returns a shallow copy of the
                  object this task was created from.
 
 
    - rethrow
      Purpose:  re-throw any exception a failed task caught.
      Format:   rethrow (void);
      Inputs:   -
      InOuts:   -
      Outputs:  -
      PreCond:  -
      PostCond: -
      Perms:    -
      Throws:   NotImplemented
                IncorrectURL
                BadParameter
                AlreadyExists
                DoesNotExist
                IncorrectState
                PermissionDenied
                AuthorizationFailed
                AuthenticationFailed
                Timeout
                NoSuccess
      Notes:    - that method does nothing unless the task is in
                  'Failed' state, and also MUST NOT throw 
                  'IncorrectState' if the task is in any other 
                  state.
                - if in 'Failed' state, the method MUST raise an
                  exception which indicates the reason why that
                  task entered the 'Failed' state (i.e. it throws 
                  the exception which caused it to enter the 
                  'Failed' state.
                - language bindings for languages with no
                  support for exceptions MUST change the state
                  of the object from which the task was created
                  so that a subsequent call to has_error() on
                  that object returns true.  A subsequent call to
                  get_error() must then return the respective
                  exception.
                - rethrow can be called multiple times, always
                  throwing the same exception.
 \end{myspec}
 
  \subsubsection*{Class \T{task\_container}}
 
    Managing a large number of tasks can be tedious.
    The |task_container| class is designed to help in
    these situations, and to effectively handle a large
    number of asynchronous operations.
    
    For example, when an application uses many
     tasks, it would be inefficient to
    invoke the |wait()| method on each of them
    individually.  The |task_container| class provides (amongst
    other operations) a mechanism to wait for a set of
    tasks.
 
    Language bindings CAN specify the \T{task\_container} to
    be, or to inherit from, a native container type, if that allows
    for the same semantics as described below, and if that helps to
    'naturalize' the SAGA Look\,\&\,Feel for that language.
 
 
 \begin{myspec}
    - CONSTRUCTOR
      Purpose:  create a task_container
      Format:   CONSTRUCTOR          (out task_container tc);
      Inputs:   -
      InOuts:   -
      Outputs:  tc:                   newly created container
      PreCond:  -
      PostCond: -
      Perms:    -
      Throws:   NotImplemented
                Timeout
                NoSuccess
      Notes:    - a 'Timeout' or 'NoSuccess' exception indicates
                  that the backend was not able to create a task
                  container.
 
 
    - DESTRUCTOR
      Purpose:  destroy a task_container
      Format:   DESTRUCTOR           (in task_container tc);
      Inputs:   tc:                   container to destroy
      InOuts:   -
      Outputs:  -
      PreCond:  -
      PostCond: -
      Perms:    -
      Throws:   -
      Notes:    - tasks in the task_container during its
                  destruction are not affected by its
                  destruction, and, in particular, are not
                  canceled.
 
 
    Task Management
    ---------------
 
    - add
      Purpose:  Add a task to a task_container.
      Format:   add                  (in  task task);
      Inputs:   task:                 task to add to the
                                      task_container
      InOuts:   -
      Outputs:  -
      PreCond:  -
      PostCond: - the task is managed by the task container.
      Perms:    -
      Throws:   NotImplemented
                Timeout
                NoSuccess
      Notes:    - a task can be added only once.  Any attempt 
                  to add a task to the container which already 
                  is in the container is silently ignored.
                - a 'Timeout' or 'NoSuccess' exception indicates
                  that the backend was not able to add the task
                  to the container.
 
 
    - remove
      Purpose:  Remove a task from a task_container.
      Format:   remove               (in  task task);
      Inputs:   task:                 the task to be removed
      InOuts:   -
      Outputs:  -
      PreCond:  - the task is managed by the task container.
      PostCond: - the task is not managed by the task container.
      Perms:    -
      Throws:   NotImplemented
                DoesNotExist
                Timeout
                NoSuccess
      Notes:    - if a task was added more than once, it can be
                  removed only once - see notes to add().
                - if the task is not in the task_container, a 
                  'DoesNotExist' exception is thrown.
                - a 'Timeout' or 'NoSuccess' exception indicates
                  that the backend was not able to remove the 
                  task from the container.
 
 
    State Management
    ----------------
 
    - run
      Purpose:  Start all asynchronous operations in the
                container.
      Format:   run (void);
      Inputs:   -
      InOuts:   -
      Outputs:  -
      PreCond:  - all tasks in the container are in 'New' state.
      PostCond: - all tasks in the container are in 'Running'
                  state.
      Perms:    - see permissions on task::run()
      Throws:   NotImplemented
                IncorrectState
                DoesNotExist
                Timeout
                NoSuccess
      Notes:    - run() MUST cause an 'IncorrectState' exception 
                  if any of the tasks in the container causes 
                  that exception on run().
                - a 'Timeout' or 'NoSuccess' exception indicates
                  that the backend was not able to run one or
                  more tasks in the container.
                - if the task_container is empty, an
                  'DoesNotExist' exception is thrown.
                - As the order of execution of the tasks is
                  undefined, no assumption on the individual
                  task states can be made after any
                  exception gets thrown.
 
 
    - wait
      Purpose:  Wait for one or more of the tasks to finish.
      Format:   wait                 (in  wait_mode mode = All,
                                      in  float  timeout = -1.0,
                                      out task   done);
      Inputs:   mode:                 wait for All or Any task
                timeout:              seconds to wait
      InOuts:   -
      Outputs:  done:                 finished task
      PreCond:  -
      PostCond: - if no timeout occurs, All/Any tasks in the 
                  container are in a final state.
      Perms:    -
      Throws:   NotImplemented
                IncorrectState
                DoesNotExist
                Timeout
                NoSuccess
      Notes:    - if mode is 'All', the wait call returns only
                  if all tasks in the container are finished,
                  or on timeout, whichever occurs first.
                  The output task is then any of the finished
                  tasks.
                - if mode is 'Any', the wait call returns on the
                  first task which would return on task::wait in
                  that timeout period, and returns that task.
                - the default wait mode is 'All' (0).
                - the returned task is removed from the
                  container, which allows constructs like
                    while ( tc.size () )
                    { 
                       saga::task t = tc.wait (saga::task::Any) ) 
                       ... 
                    }
                - wait() MAY cause an 'IncorrectState' exception
                  if any of the tasks in the container causes 
                  that exception on wait().
                - if the task_container is empty, an
                  'DoesNotExist' exception is thrown.
                - a 'Timeout' or 'NoSuccess' exception indicates
                  that the backend was not able to wait for one 
                  or more tasks in the container.
                - As the order of execution of the tasks is
                  undefined, no assumption on the individual
                  task states can be made after any
                  exception gets thrown.
                - for timeout semantics, see Section 2.
 
 
    - cancel
      Purpose:  Cancel all the asynchronous operations in the
                container.
      Format:   cancel               (in  float  timeout = 0.0);
      Inputs:   timeout:              time for freeing resources
      InOuts:   -
      Outputs:  -
      PreCond:  -
      PostCond: - if no timeout occurs, all tasks in the 
                  container are in 'Canceled' state.
      Perms:    -
      Throws:   NotImplemented
                IncorrectState
                DoesNotExist
                Timeout
                NoSuccess
      Notes:    - see semantics of task cancel.
                - cancel() MUST cause an 'IncorrectState' 
                  exception if any of the tasks in the container
                  causes that exception on cancel().
                - a 'Timeout' or 'NoSuccess' exception indicates
                  that the backend was not able to run one or
                  more tasks in the container.
                - if the task_container is empty, an
                  'DoesNotExist' exception is thrown.
                - As the order of execution of the tasks is
                  undefined, no assumption on the individual
                  task states can be made after any
                  exception gets thrown.
 
 
    Inspection
    ----------
 
    - size
      Purpose:  return the number of tasks in the task 
                task_container.
      Format:   size                 (out int  n);
      Inputs:   -
      InOuts:   -
      Outputs:  n:                    number of tasks in 
                                      task_container
      PreCond:  -
      PostCond: -
      Perms:    -
      Throws:   NotImplemented
                Timeout
                NoSuccess
      Notes:    - a 'Timeout' or 'NoSuccess' exception indicates
                  that the backend was not able to list the
                  tasks in the container.
 
 
    - list_tasks
      Purpose:  List the tasks in the task_container.
      Format:   list_tasks           (out array<task>   tasks);
      Inputs:   -
      InOuts:   -
      Outputs:  tasks:                array of all tasks in 
                                      the task_container
      PreCond:  -
      PostCond: -
      Perms:    -
      Throws:   NotImplemented
                Timeout
                NoSuccess
      Notes:    - a 'Timeout' or 'NoSuccess' exception indicates
                  that the backend was not able to list the
                  tasks in the container.
 
 
    - get_task
      Purpose:  Get a single task from the task_container.
      Format:   get_task             (in  string   id,
                                      out task     t);
      Inputs:   id:                   the object id identifying
                                      the task to return
      InOuts:   -
      Outputs:  t:                    the task identified by
                                      id
      PreCond:  -
      PostCond: -
      Perms:    -
      Throws:   NotImplemented
                DoesNotExist
                Timeout
                NoSuccess
      Notes:    - the returned task is NOT removed from the
                  task_container.
                - if the id specifies a task which is not in 
                  the container, a 'DoesNotExist' exception is
                  thrown.
                - a 'Timeout' or 'NoSuccess' exception indicates
                  that the backend was not able to list the
                  tasks in the container.
 
 
    - get_tasks 
      Purpose:  Get the tasks in the task_container.
      Format:   get_tasks            (out array<task>   tasks);
      Inputs:   -
      InOuts:   -
      Outputs:  tasks:                array of tasks in
                                      task_container
      PreCond:  -
      PostCond: -
      Perms:    -
      Throws:   NotImplemented
                Timeout
                NoSuccess
      Notes:    - the returned tasks are NOT removed from the
                  task_container.
                - if the task_container is empty, an
                  empty list is returned.
                - a 'Timeout' or 'NoSuccess' exception indicates
                  that the backend was not able to list the
                  tasks in the container.
 
 
    - get_states
      Purpose:  Get the states of all tasks in the 
                task_container.
      Format:   get_states           (out array<state>   states);
      Inputs:   -
      InOuts:   -
      Outputs:  states:               array of states for
                                      tasks in task_container
      PreCond:  -
      PostCond: -
      Perms:    -
      Throws:   NotImplemented
                Timeout
                NoSuccess
      Notes:    - the returned list is not ordered
                - if the task_container is empty, an
                  empty list is returned.
                - a 'Timeout' or 'NoSuccess' exception indicates
                  that the backend was not able to obtain the
                  states of the tasks in the container.
 \end{myspec}
 
 
 \subsubsection{Examples}
 
 \begin{mycode}
  // c++ example
  saga::directory dir;
  saga::job       job;
 
  ...
 
  /* create tasks */
  saga::task t1 = dir.ls         <saga::task> (result);
  saga::task t2 = dir.copy       <saga::task> (source,target);
  saga::task t3 = dir.move       <saga::task> (source,target);
  saga::task t4 = job.checkpoint <saga::task> ();
  saga::task t5 = job.signal     <saga::task> (SIG_USR);
 
  // start tasks
  t1.run ();
  t2.run ();
  t3.run ();
  t4.run ();
  t5.run ();
 
  // put all tasks into container
  saga::task_container tc;
 
  tc.add (t1);
  tc.add (t2);
  tc.add (t3);
  tc.add (t4);
  tc.add (t5);
 
  // take one out again
  tc.remove (t5);
 
  // wait for all other tasks in container to finish
  tc.wait ();
 
  // wait for the last task
  t5.wait ();
 
  +-------------------------------------------------------------+
 
  // example for error handling in C++
  {
    task.run  ();
    task.wait ();
 
    if ( task.get_state () == saga::task::Failed )
    {
      try {
        task.rethrow ();
      }
      catch ( const saga::exception & e )
      {
        std::cout << "task failed: " 
                  << e.get_message () 
                  << std::endl;
      }
    }
  }
 \end{mycode}
 
