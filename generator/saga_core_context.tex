 
  The \T{saga::context} class provides the
  functionality of a security information container. A context
  gets created, and attached to a session handle.  As such it is
  available to all objects instantiated in that session.
  Multiple contexts can co-exist in one session -- it is up to
  the implementation to choose the correct context for a
  specific method call.  Also, a single \T{saga::context}
  instance can be shared between multiple
  sessions.  SAGA objects created from other SAGA objects
  inherit its session and thus also its context(s).
  Section~\ref{ssec:session} contains more information about the
  \T{saga::session} class, and also about the management and
  lifetime of \T{saga::context} instances associated with a SAGA
  session.
 
  A typical usage scenario is:
 
  \begin{mycode}
   // context usage scenario in c++
 
   saga::context c_1, c_2;
 
   // c_1 will use a Globus proxy.  Set the type to Globus, pick
   // up the default Globus settings, and then identify the proxy
   // to be used
   c_1.set_attribute ("Type", "Globus");
   c_1.set_attribute ("UserProxy", "/tmp/special_x509up_u500");
 
   // c_2 will be used as ssh context, and will just pick up the
   // public/private key from $HOME/.ssh
   c_2.set_attribute ("Type", "ssh");
 
   // a saga session gets created, and uses both contexts
   saga::session s;
   s.add_context (c_1);
   s.add_context (c_2);
 
   // a remote file in this session can now be accessed via 
   // gridftp or ssh
   saga::file f (s, "any://remote.net/tmp/data.txt");
   f.copy ("data.bak");
  \end{mycode}
 
  A context has a set of attributes which can be
  set/get via the SAGA attributes interface.  Exactly which
  attributes a context actually evaluates, depends upon its type
  (see documentation to the \T{set\_defaults()} method.
 
 \newpage
  
  An implementation CAN implement multiple types of
  contexts.  The implementation MUST document which context types
  it supports, and which values to the \T{Type} attribute are used
  to identify these context types.  Also, the implementation
  MUST document what default values it supports for the various
  context types, and which attributes need to be or can be set by
  the application.
 
  The lifetime of \T{saga::context} instances is
  defined by the lifetime of those \T{saga::session} instances
  the contexts are associated with, and of those SAGA objects
  which have been created in these sessions.  For detailed
  information about lifetime management, see 
  Section~\ref{ssec:lifetime}, and the description of
  the SAGA session class in Section~\ref{ssec:session}.
 
  For application level Authorization (e.g. for
  streams, monitoring, steering),  contexts are
  used to inform the application about the requestor's
  identity.  These contexts represent the security
  information that has been used to initiate the connection to
  the SAGA application.  To support that mechanism, a number of specific
  attributes are available, as specified below.  They are named
  \T{"Remote<attribute>"}.  An implementation MUST at
  least set the \T{Type} attribute for such contexts, and SHOULD
  provide as many attribute values as possible.
 
  For example, a SAGA application \I{A} creates a
  \T{saga::stream\_server} instance.  A SAGA application \I{B}
  creates a 'globus' type context, and, with a session using
  that context, creates a \T{saga::stream} instance connecting
  to the stream server of \I{A}.  \I{A} should then obtain a
  context upon connection accept (see Sections on
  Monitoring,~\ref{ssec:monitoring}, and
  Streams,~\ref{ssec:stream}, for details).  That context should
  then also have the type 'globus', its 'RemoteID' attribute
  should contain the distinguished name of the user  certificate,
  and its attributes 'RemoteHost' and 'RemotePort' should have
  the appropriate values.
 
  Note that \T{UserID}s SHOULD be formatted so that they
  can be used as user identifiers in the SAGA permission model
  -- see Section~\ref{ssec:permission} for details.
 
 
 \subsubsection{Specification}
 
 \begin{myspec}
  package saga.core
  {
    class context : implements   saga::object
                    implements   saga::attributes
                 // from object  saga::error_handler
    {
      CONSTRUCTOR      (in  string       type = "",
                        out context      obj);
      DESTRUCTOR       (in  context      obj);
 
-     set_defaults     (void);                   
-
      // Attributes:
      //
      //   name:  Type
      //   desc:  type of context
      //   mode:  ReadWrite
      //   type:  String
      //   value: naming conventions as described above apply
      //
      //   name:  Server
      //   desc:  server which manages the context
      //   mode:  ReadWrite
      //   type:  String
      //   value: -
      //   note:  - a typical example would be the contact
      //            information for a MyProxy server, such as 
      //            'myproxy.remote.net:7512', for a 'myproxy'
      //            type context.
      //
      //   name:  CertRepository
      //   desc:  location of certificates and CA signatures
      //   mode:  ReadWrite
      //   type:  String
      //   value: -
      //   note:  - a typical example for a globus type context 
      //            would be "/etc/grid-security/certificates/".
      //
      //   name:  UserProxy
      //   desc:  location of an existing certificate proxy to
      //          be used
      //   mode:  ReadWrite
      //   type:  String
      //   value: -
      //   note:  - a typical example for a globus type context 
      //            would be "/tmp/x509up_u<uid>".
      //
      //   name:  UserCert
      //   desc:  location of a user certificate to use
      //   mode:  ReadWrite
      //   type:  String
      //   value: -
      //   note:  - a typical example for a globus type context 
      //            would be "$HOME/.globus/usercert.pem".
      //
      //   name:  UserKey
      //   desc:  location of a user key to use
      //   mode:  ReadWrite
      //   type:  String
      //   value: -
      //   note:  - a typical example for a globus type context 
      //            would be "$HOME/.globus/userkey.pem".
      //
      //   name:  UserID
      //   desc:  user id or user name to use
      //   mode:  ReadWrite
      //   type:  String
      //   value: -
      //   note:  - a typical example for a ftp type context 
      //            would be "anonymous".
      //
      //   name:  UserPass
      //   desc:  password to use
      //   mode:  ReadWrite
      //   type:  String
      //   value: -
      //   note:  - a typical example for a ftp type context 
      //            would be "anonymous@localhost".
      //
      //   name:  UserVO
      //   desc:  the VO the context belongs to
      //   mode:  ReadWrite
      //   type:  String
      //   value: -
      //   note:  - a typical example for a globus type context 
      //            would be "O=dutchgrid".
      //
      //   name:  LifeTime
      //   desc:  time up to which this context is valid
      //   mode:  ReadWrite
      //   type:  Int
      //   value: -1
      //   note:  - format: time and date specified in number of 
      //            seconds since epoch
      //          - a value of -1 indicates an infinite lifetime.
      //
      //   name:  RemoteID
      //   desc:  user ID for an remote user, who is identified
      //          by this context.
      //   mode:  ReadOnly
      //   type:  String
      //   value: -
      //   note:  - a typical example for a globus type context 
      //            would be
      //            "/O=dutchgrid/O=users/O=vu/OU=cs/CN=Joe Doe".
      //
      //   name:  RemoteHost
      //   desc:  the hostname where the connection origininates
      //          which is identified by this context.
      //   mode:  ReadOnly
      //   type:  String
      //   value: -
      //
      //   name:  RemotePort
      //   desc:  the port used for the connection which is
      //          identified by this context.
      //   mode:  ReadOnly
      //   type:  String
      //   value: -
      //
    }
  }
 \end{myspec}
 
 
 \subsubsection{Specification Details}
 
 \subsubsection*{Class \T{context}}
 
 \begin{myspec}
 
    - CONSTRUCTOR
      Purpose:  create a security context
      Format:   CONSTRUCTOR          (in  stringt type = "",
                                      out context obj);
      Inputs:   type:                 initial type of context
      InOuts:   -
      Outputs:  obj:                  the newly created object
      PreCond:  -
      PostCond: -
      Perms:    -
-     Throws:   NotImplemented
!     Throws:   IncorrectState
                Timeout
                NoSuccess
+     Notes:    - 
-     Notes:    - if type is given (i.e. non-empty), then the
-                 CONSTRUCTOR internally calls set_defaults().
-                 The notes to set_defaults apply.
 
    - DESTRUCTOR
      Purpose:  destroy a security context
      Format:   DESTRUCTOR           (in context obj);
      Inputs:   obj:                  the object to destroy
      InOuts:   -
      Outputs:  -
      PreCond:  -
      PostCond: - See notes about lifetime management 
                  in Section 2
      Perms:    -
      Throws:   -
      Notes:    -
-
-   - set_defaults
-     Purpose:  set default values for specified context type
-     Format:   set_defaults         (void);                 
-     Inputs:   -
-     InOuts:   -
-     Outputs:  -
-     PreCond:  -
-     PostCond: - the context is valid, and can be used for 
-                 authorization.
-     Perms:    -
-     Throws:   NotImplemented
-     Throws:   IncorrectState
-               Timeout
-               NoSuccess
-     Notes:    - the method evaluates the value of the 'Type'
-                 attribute, and of all other non-empty
-                 attributes, and, based on that information, 
-                 tries to set sensible default values for all 
-                 previously empty attributes.
-               - if the 'Type' attribute has an empty value, an
-                 'IncorrectState' exception is thrown.
-               - this method can be called more than once on
-                 a context instance.
-               - if the implementation cannot create valid
-                 default values based on the available
-                 information, an 'NoSuccess' exception is
-                 thrown, and a detailed error message is given,
-                 describing why no default values could be
-                 set.
 \end{myspec}
 
 
