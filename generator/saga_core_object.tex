 
  The SAGA object interface provides methods which are essential
  for all SAGA objects.  It provides a unique ID which helps
  maintain a list of SAGA objects at the application level as
  well as allowing for inspection of objects type and its
  associated session.
        
  The object id MUST be formatted as UUID, as standardized by
  the Open Software Foundation (OSF) as part of the Distributed
  Computing Environment (DCE).  The UUID format is also
  described in the IETF RFC-4122~\cite{rfc-4122}.
 
  Note that there are no object IDs for the various
  SAGA exceptions, but only one ID for the \T{saga::exception}
  base class. Also, it is not possible to inspect a SAGA object
  instance for the availability of certain SAGA interfaces, as
  they are fixed and well defined by the SAGA specification.
  Language bindings MAY, however, add such inspection, if that
  is natively supported by the language.
 
 \subsubsection{Specification}
 
 \begin{myspec}
  package saga.core
  {
    enum object_type
    {
-     Exception        =   1,
      URL              =   1,
      Buffer           =   2,
      Session          =   3,
      Context          =   4,
      Task             =   5,
      TaskContainer    =   6,
      Metric           =   7,
      NSEntry          =   8,
      NSDirectory      =   9,
      IOVec            =  10,
      File             =  11,
      Directory        =  12,
      LogicalFile      =  13,
      LogicalDirectory =  14,
      JobDescription   =  15,
      JobService       =  16,
      Job              =  17,
      JobSelf          =  18,
      StreamService    =  19,
      Stream           =  20,
      Parameter        =  21,
      RPC              =  22,
    }
 
 
    interface object : implements saga::core::error-handler
    {
      get_id       (out string      id     );
      get_type     (out object_type type   );
      get_session  (out session     s      );
 
      // deep copy
      clone        (out object      clone  );
    }
  }
 \end{myspec}
 
 
 \subsubsection{Specification Details}
 
 \subsubsection*{Enum \T{object\_type}}
 
  The SAGA \T{object\_type} enum allows for inspection of
  SAGA object instances.  This, in turn, allows to treat large
  numbers of SAGA object instances in containers, without the
  need to create separate container types for each specific SAGA
  object type.  Bindings to languages that natively support inspection on
  object types MAY omit this enum and the \T{get\_type()}
  method.
 
  SAGA extensions which introduce new SAGA objects (i.e.
  introduce new classes which implement the \T{saga::object}
  interface) MUST define the appropriate \T{object\_type} enums
  for inspection.  SAGA implementations SHOULD support these
  enums for all packages which are provided in that
  implementation, even for classes which are not implemented.
 
 
 \subsubsection*{Interface \T{object}}
 
 \begin{myspec}
    - get_id
      Purpose:  query the object ID
      Format:   get_id               (out string id);
      Inputs:   -
      InOuts:   -
      Outputs:  id:                   uuid for the object
      PreCond:  -
      PostCond: -
      Perms:    -
      Throws:   -
      Notes:    - 
 
 
    - get_type
      Purpose:  query the object type
      Format:   get_type             (out object_type type);
      Inputs:   -
      InOuts:   -
      Outputs:  type:                 type of the object
      PreCond:  -
      PostCond: -
      Perms:    -
      Throws:   -
      Notes:    - 
 
 
    - get_session
      Purpose:  query the objects session
      Format:   get_session          (out session s);
      Inputs:   -
      InOuts:   -
      Outputs:  s:                    session of the object
      PreCond:  - the object was created in a session, either
                  explicitly or implicitly.
      PostCond: - the returned session is shallow copied.
      Perms:    -
      Throws:   DoesNotExist
      Notes:    - if no specific session was attached to the
                  object at creation time, the default SAGA
                  session is returned.
                - some objects do not have sessions attached,
                  such as job_description, task, metric, and the
                  session object itself.  For such objects, the
                  method raises a 'DoesNotExist' exception.
 
 
    // deep copy:
    -------------
 
    - clone
      Purpose:  deep copy the object
      Format:   clone                (out object clone);
      Inputs:   -
      InOuts:   -
      Outputs:  clone:                the deep copied object
      PreCond:  -
      PostCond: - apart from session and callbacks, no other
                  state is shared between the original object 
                  and it's copy.
      Perms:    -
      Throws:   NoSuccess
      Notes:    - that method is overloaded by all classes 
                  which implement saga::object, and returns 
                  a deep copy of the respective class type 
                  (the method is only listed here).
                - the method SHOULD NOT cause any backend 
                  activity, but is supposed to clone the client
                  side state only.
+               - the object id is not copied -- a new id MUST 
+                 be assigned instead.
                - for deep copy semantics, see Section 2.
 \end{myspec}
 
 
 \subsubsection{Examples}
 
 \begin{mycode}
  // c++ example
 
  // have 2 objects, streams and files, and do:
  //  - read 100 bytes
  //  - skip 100 bytes
  //  - read 100 bytes
 
  int  out;
  char data1[100];
  char data2[100];
  char data[100];
 
  saga::buffer buf1 (data1, 100);
  saga::buffer buf2 (data2, 100);
  saga::buffer buf;
 
  // create objects
  saga::file   f (url[1]);
  saga::stream s (url[2]);
 
  // f is opened at creation, s needs to be connected
  s.connect ();
 
  // create tasks for reading first 100 bytes ...
  saga::task t1 = f.read <saga::task> (100, buf1);
  saga::task t2 = s.read <saga::task> (100, buf2);
 
  // create and fill the task container ...
  saga::task_container tc;
 
  tc.add (t1);
  tc.add (t2);
 
  // ... and wait who gets done first
  while ( saga::task t = tc.wait (saga::task::Any) )
  {
     // depending on type, skip 100 bytes then create a
     // new task for the next read, and re-add to the tc
 
     switch ( t.get_object().get_type () )
     {
       case saga::object::File :
         // point buf to results
         buf = buf1;
 
         // get back file object
         saga::file f = saga::file (t.get_object ());
 
         // skip for file type (sync seek)
         saga::file (f.seek (100, SEEK_SET);
 
         // create a new read task
         saga::task t2 = f.read <saga::task> (100, buf1));
 
         // add the task to the container again
         tc.add (t2);
 
         break;
 
       case saga::object::Stream :
         // point buf to results
         buf = buf2;
 
         // get back stream object
         saga::stream s = saga::stream (t.get_object ());
 
         // skip for stream type (sync read and ignore)
         saga::stream (s.read (100, buf2);
 
         // create a new read task
         saga::task t2 = s.read <saga::task> (100, buf2));
 
         // add the task to the container again
         tc.add (t2);
 
         break;
 
       default:
         throw exception ("Something is terribly wrong!");
     }
 
     std::cout << "found: '" << out << " bytes: " 
                             << buf.get_data () 
                             << std::endl;
 
     // tc is filled again, we run forever, read/seeking from
     // whatever we find after the wait.
  }
 \end{mycode}
 
 
 % \subsubsection{Notes}
 % 
 %   The |saga::object| interface will be much more useful once the
 %   SAGA API supports serialization and deserialization of saga object
 %   instances.  Such functionality was discussed in respect to
 %   persistent information storage, and to task dependencies and
 %   workflows, and will be revisited on respective future extensions.
 
 
