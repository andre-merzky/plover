 
   A number of use cases involve launching  remotely located
   components in order to create distributed applications. These
   use cases require simple remote socket connections to be
   established between these components and their control
   interfaces.
 
   The target of the streams API is to establish the simplest
   possible authenticated socket connection with hooks to
   support application level authorization.  The stream API has
   the following characteristics
 
   \begin{enumerate}
 
     \item It is not performance oriented:  If performance is
     required, then it is better to program directly against the
     APIs of existing performance oriented protocols like
     GridFTP or XIO.  The API design should allow, however, for
     high performance implementations.
 
     \item It is focused on TCP/IP socket connections.  There
     has been no attempt to generalize this to arbitrary
     streaming interfaces (although it does not prevent such
     things as connectionless protocols from being supported).
 
     \item It does not attempt to create a programming paradigm
     that diverges very far from baseline BSD sockets, Winsock,
     or Java Sockets.
 
   \end{enumerate}
   
   This API greatly reduces the complexity of establishing
   authenticated socket connections in order to communicate with
   remotely located components.  It however, provides very
   limited functionality and is thus suitable for applications
   that do not have very sophisticated requirements (as per
   80-20 rule).  It is envisaged that as applications become
   progressively more sophisticated, they will gradually move to
   more  sophisticated, native APIs in order to support those
   needs.
 
   Several SAGA use cases require a more abstract communication
   API, which exchanges opaque messages instead of byte streams.
   That behavior can be modeled on top of this stream API, but
   future versions of the SAGA API may introduce higher level
   communication APIs.
 
   \subsubsection{Endpoint URLs}
 
    The SAGA stream API uses URLs to specify connection
    endpoints.  These URLs are supposed to allow SAGA
    implementations to be interoperable.  For example, the URL
 
    \shift |tcp://remote.host.net:1234/|
 
    is supposed to signal that a standard |tcp| connection can
    be established with host |remote.host.net| on port |1234|.
    No matter what the specified URL scheme is, the SAGA stream
    API implementation MUST have the same semantics on API
    level, i.e. behave like a reliable byte-oriented data
    stream.
 
   \subsubsection{Endpoint Permissions}
 
    The SAGA API allows for application level authorization of
    stream communications: an application is able to set
    permissions on \T{saga::stream\_server} and \T{saga::stream}
    instances.  These permissions control what remote party can
    perform what action on those streams, e.g.  control what
    remote parties are able to connect to an endpoint, or to
    write to them etc.
 
    Not all implementations will be able to fully implement that
    security model -- the implementation MUST carefully document
    which permissions are supported, and which are not.
 
 \subsubsection{Specification}
 
 \begin{myspec}
  package saga.stream
  {
    enum state
    {
      New          =  1
      Open         =  2,
      Closed       =  3,
      Dropped      =  4,
      Error        =  5
    }
 
    enum activity
    {
      Read         =  1,
      Write        =  2,
      Exception    =  4
    }
 
 
    class stream_server  : implements   saga::object
                           implements   saga::async
                           implements   saga::monitorable
                           implements   saga::permissions
                        // from object  saga::error_handler
    {
      CONSTRUCTOR       (in    session         s,
                         in    url             url,
                         out   stream_server   obj);
      DESTRUCTOR        (in    stream_server   obj);
 
      get_url           (out   url             url);

      serve             (in    float           timeout = -1.0,
                         out   stream          stream);
 
+     connect           (in    float           timeout = -1.0,
+                        out   stream          stream);
+
      close             (in    float           timeout = 0.0);
 
      // Metrics:
      //   name:  stream_server.client_connect
      //   desc:  fires if a client connects
      //   mode:  ReadOnly
      //   unit:  1
      //   type:  Trigger
      //   value: 1
    }
 
 
    class stream : extends      saga::object
                   implements   saga::async
                   implements   saga::attributes
                   implements   saga::monitorable
                // from object  saga::error_handler
    {
      // constructor / destructor
      CONSTRUCTOR  (in    session          s,
                    in    url              url = "",
                    out   stream           obj);
      DESTRUCTOR   (in    stream           obj);
 
      // inspection methods
      get_url      (out   url              url);
      get_context  (out   context          ctx);
 
      // management methods
-     connect      (void);
+     connect      (in    float            timeout = -1.0);
      wait         (in    int              what,
                    in    float            timeout = -1.0,
                    out   int              cause);
      close        (in    float            timeout = 0.0);
 
      // I/O methods
      read         (inout buffer           buf, 
                    in    int              len_in = -1,
                    out   int              len_out);
      write        (in    buffer           buf,
                    in    int              len_in = -1,
                    out   int              len_out);
 
      // Attributes:
      //
      //   name:  BufSize
      //   desc:  determines the size of the send buffer, 
      //          in bytes
      //   mode:  ReadWrite, optional
      //   type:  Int
      //   value: system dependent
      //   notes: - the implementation MUST document the
      //            default value, and its meaning (e.g. on what
      //            layer that buffer is maintained, or if it
      //            disables zero copy).
      //
      //   name:  Timeout
      //   desc:  determines the amount of idle time
      //          before dropping the line, in seconds
      //   mode:  ReadWrite, optional
      //   type:  Int
      //   value: system dependent
      //   notes: - the implementation MUST document the
      //            default value
      //          - if this attribute is supported, the
      //            connection MUST be closed by the
      //            implementation if for that many seconds
      //            nothing has been read from or written to 
      //            the stream.
      //
      //   name:  Blocking
      //   desc:  determines if read/writes are blocking
      //          or not
      //   mode:  ReadWrite, optional
      //   type:  Bool
      //   value: True
      //   notes: - if the attribute is not supported, the
      //            implementation MUST be blocking
      //          - if the attribute is set to 'True', a read or
      //            write operation MAY return immediately if
      //            no data can be read or written - that does
      //            not constitute an error (see EAGAIN in
      //            POSIX).
      //
      //   name:  Compression
      //   desc:  determines if data are compressed
      //          before/after transfer
      //   mode:  ReadWrite, optional
      //   type:  Bool
      //   value: schema dependent
      //   notes: - the implementation MUST document the
      //            default values for the available schemas
      //
      //   name:  Nodelay
      //   desc:  determines if packets are sent
      //          immediately, i.e. without delay
      //   mode:  ReadWrite, optional
      //   type:  Bool
      //   value: True
      //   notes: - similar to the TCP_NODELAY option
      //
      //   name:  Reliable
      //   desc:  determines if all sent data MUST arrive
      //   mode:  ReadWrite, optional
      //   type:  Bool
      //   value: True
      //   notes: - if the attribute is not supported, the
      //            implementation MUST be reliable
 
 
      // Metrics:
      //   name:  stream.state
      //   desc:  fires if the state of the stream changes,
      //          and has the value of the new state
      //          enum
      //   mode:  ReadOnly
      //   unit:  1
      //   type:  Enum
      //   value: New
      //
      //   name:  stream.read
      //   desc:  fires if a stream gets readable
      //   mode:  ReadOnly
      //   unit:  1
      //   type:  Trigger
      //   value: 1
      //   notes: - a stream is considered readable if a
      //            subsequent read() can successfully read 
      //            1 or more bytes of data.
      //
      //   name:  stream.write
      //   desc:  fires if a stream gets writable
      //   mode:  ReadOnly
      //   unit:  1
      //   type:  Trigger
      //   value: 1
      //   notes: - a stream is considered writable if a
      //            subsequent write() can successfully write
      //            1 or more bytes of data.
      //
      //   name:  stream.exception
      //   desc:  fires if a stream has an error condition
      //   mode:  ReadOnly
      //   unit:  1
      //   type:  Trigger
      //   value: 1
      //   notes: - 
      //
      //   name:  stream.dropped
      //   desc:  fires if the stream gets dropped by the
      //          remote party
      //   mode:  ReadOnly
      //   unit:  1
      //   type:  Trigger
      //   value: 1
    }
  }
 \end{myspec}
 
 
 \subsubsection{Specification Details}
 
  \subsubsection*{Enum \T{state}}
  \label{t}
 
  \begin{figure}[!ht]
    \begin{center}
      \includegraphics[width=0.95\textwidth]{stream_states}
      \caption{\label{fig:stream_states} \footnotesize The SAGA stream state model
      (See Figure~\ref{fig:states} for a legend).}
    \end{center}
  \end{figure}
 
   A SAGA  stream can be in several states -- the complete state
   diagram is shown in Figure~\ref{fig:stream_states}\ref{t}.  The
   stream states are:
 
   |New|\\[1.5mm]
   \begin{tabular}{cp{110mm}}
     ~~ & A newly constructed stream enters the initial |New| state.
          It is not connected yet, and no I/O operations can be
          performed on it.   |connect()| must be called to advance
          the state to |Open| (on success) or |Error| (on failure).
   \end{tabular}
 
   |Open|\\[1.5mm]
   \begin{tabular}{cp{110mm}}
     ~~ & The stream is connected to the remote endpoint, and I/O
          operations can be called.  If any error occurs on the
          stream, it will move into the |Error| state.  If the remote
          party closes the connection, the stream will move into the
          |Dropped| state.  If |close()| is called on the stream, the
          stream will enter the |Closed| state.
   
   \end{tabular}
 
   |Closed|\\[1.5mm]
   \begin{tabular}{cp{110mm}}
     ~~ & The |close()| method was called on the stream -- I/O is no
          longer possible.  This is a final state.
   \end{tabular}
 
   |Dropped|\\[1.5mm]
   \begin{tabular}{cp{110mm}}
     ~~ & The remote party closed the connection -- I/O is no longer
          possible.  This is a final state.
   \end{tabular}
 
   |Error|\\[1.5mm]
   \begin{tabular}{cp{110mm}}
     ~~ & An error occurred on the stream -- I/O is no longer
          possible.  This is a final state.  The exact reason for
          reaching this state MUST be available through the
          |error_handler| interface.
   \end{tabular}
 
   All method calls, apart from the \T{DESTRUCTOR}, will cause
   an \T{IncorrectState} exception if the stream is in a final
   state.
 
 
  \subsubsection*{Enum \T{activity\_type}}
 
   The SAGA stream API allows for event driven communication.  A
   stream can flag activities, i.e. |Read|, |Write| and
   |Exception|, and the application can react on these
   activities.  It is possible to poll for these events (using
   |wait()| with a potential timeout), or to get asynchronous
   notification of these events, by using the respective
   metrics.
 
   |Read|\\[1.5mm]
   \begin{tabular}{cp{110mm}}
     ~~ & Data are available on the stream, and a subsequent
          \T{read()} will succeed.
   \end{tabular}
 
   |Write|\\[1.5mm]
   \begin{tabular}{cp{110mm}}
     ~~ & The stream is accepting data, and a subsequent 
          \T{write()} will succeed.
   \end{tabular}
 
   |Exception|\\[1.5mm]
   \begin{tabular}{cp{110mm}}
     ~~ & An error occurred on the stream, and a following I/O
          operation may fail.
   \end{tabular}
 
 
  \subsubsection*{Class \T{stream\_server}}
 
    The |stream_server| object \XAddn{represents a connection
    endpoint.  If it represents aconnection endpoint managed by
    the application, it} establishes a listening/server object
    that waits for client connections.  \XAdd[5]{If it
    represents a remote connection point, calling \T{connect()}
    on it will create a client \T{stream} instance connected to
    that remote endpoint.}
    
    The |stream_server| can \XAddn{thus} \I{only} be used as a
    factory for client sockets.  It doesn't \XCorr[2]{support}
    any read/write I/O.  
    
 
 \begin{myspec}
    - CONSTRUCTOR
      Purpose:  create a new stream_server object
      Format:   CONSTRUCTOR          (in  session   s,
                                      in  url       url = "",
                                      out stream_server obj);
      Inputs:   s:                    session to be used for
                                      object creation
                url:                  channel name or url,
                                      defines the source side
                                      binding for the stream
      InOuts:   -
      Outputs:  obj:                  new stream_server object
      PreCond:  -
      PostCond: - stream_server can wait for incoming
                  connections.
                - 'Owner' of name is the id of the context
                  used to create the stream_server.
                - the stream_server has 'Exec', 'Query', 'Read' 
                  and 'Write' permissions for '*'.
      Perms:    -
      Throws:   NotImplemented
                IncorrectURL
                BadParameter
                PermissionDenied
                AuthorizationFailed
                AuthenticationFailed
                Timeout
                NoSuccess
      Notes:    - if the given url is an empty string (the 
                  default), the implementation will choose an
                  appropriate default value.
                - the implementation MUST ensure that the given
                  URL is usable, and a later call to 'serve' 
                  will not fail because of the information given
                  by the URL - otherwise, a 'BadParameter'
                  exception MUST be thrown.
 
 
    - DESTRUCTOR
      Purpose:  Destructor for stream_server object.
      Format:   DESTRUCTOR           (in stream_server obj)
      Inputs:   obj:                  object to be destroyed
      InOuts:   -
      Outputs:  -
      PreCond:  -
      PostCond: - the stream_server is closed.
      Perms:    -
      Throws:   -
      Notes:    - if the instance was not closed before, the 
                  destructor performs a close() on the instance,
                  and all notes to close() apply.
 
    // inspection
    - get_url
      Purpose:  get URL to be used to connect to this server
      Format:   get_url              (out url       url);
      Inputs:   -
      InOuts:   -
      Outputs:  url:                  the URL of the connection.
      PreCond:  -
      PostCond: -
      Perms:    -
      Throws:   NotImplemented
                IncorrectState
                PermissionDenied
                AuthorizationFailed
                AuthenticationFailed
                Timeout
                NoSuccess
      Notes:    - returns a URL which can be passed to
                  the stream constructor to create a connection
                  to this stream_server.
      
 
    // stream management
    - serve
      Purpose:  wait for incoming client connections
      Format:   serve                (in  float   timeout,
                                      out stream  client);
      Inputs:   timeout:              number of seconds to wait
                                      for a client
      InOuts:   -
      Outputs:  client:               new Connected stream object
      PreCond:  -
      PostCond: - the returned client is in 'Open' state.
                - the session of the returned client is that of 
                  the stream_server.
      Perms:    - Exec.
                - Exec for the connecting remote party.
      Throws:   NotImplemented
                IncorrectState
                PermissionDenied
                AuthorizationFailed
                AuthenticationFailed
                NoSuccess
                Timeout
      Notes:    - if successful, it returns a new stream object
                  that is connected to the client.
                - if no client connects within the specified 
                  timeout, a 'Timeout' exception is thrown.
                - if connection setup failed (not on timeout!),
                  the returned client is in the 'Error' state.
                  Its error_handler interface should give
                  detailed information about the reason.
                - for timeout semantics, see Section 2.

+
+   - connect
+     Purpose:  Establishes a connection to the remote
+               connection endpoint represented by this instance
+     Format:   connect              (in float timeout = -1.0);
+     Inputs:   timeout:              connection timeout
+     InOuts:   -
+     Outputs:  -
+     PreCond:  - 
+     PostCond: - the returned stream is in 'Open' state.
+     Perms:    Exec
+     Throws:   NotImplemented
+               PermissionDenied
+               AuthorizationFailed
+               AuthenticationFailed
+               Timeout
+               NoSuccess
+     Notes:    - on failure, the stream state is changed to
+                 'Error'
+               - this call is equivalent to creating a
+                 stream instance with the URL used to
+                 create this stream_server instance, and 
+                 calling connect() on that stream.
+               - if the stream instance is not in 'New' state,
+                 an 'IncorrectState' exception is thrown.
+               - for timeout semantics, see Section 2.
+

    - close
      Purpose:  closes a stream server
      Format:   close                (in float timeout)
      Inputs:   timeout               seconds to wait
      InOuts:   -
      Outputs:  -
      PreCond:  - 
      PostCond: - no clients are accepted anymore.
                - no callbacks registered for the
                  'ClientConnect' metric are invoked.
      Perms:    -
      Throws:   NotImplemented
-               IncorrectState
                NoSuccess
      Notes:    - any subsequent method call on the object
                  MUST raise an 'IncorrectState' exception
                  (apart from DESTRUCTOR and close()).
                - if close() is implicitly called in the
                  DESTRUCTOR, it will never throw an exception.
                - close() can be called multiple times, with no
                  side effects.
                - for resource deallocation semantics, see 
                  Section 2.
                - for timeout semantics, see Section 2.
 \end{myspec}
 
 
  \subsubsection*{Class \T{stream}}
 
    This is the object that encapsulates all client stream
    objects.
 
 \begin{myspec}
    Constructor / Destructor:
    -------------------------
 
    - CONSTRUCTOR
      Purpose:  Constructor, initializes a client stream,
                for later connection to a server.
      Format:   CONSTRUCTOR          (in  session    s,
                                      in  url        url,
                                      out stream  obj);
      Inputs:   s:                    saga session handle
                url:                  server location as URL
      InOuts:   -
      Outputs:  obj:                  new, unconnected stream
                                      instance
      PreCond:  -
      PostCond: - the state of the socket is 'New'.
      Perms:    - Query for the stream_server represented by
                  url.
      Throws:   NotImplemented
                IncorrectURL
                BadParameter
                PermissionDenied
                AuthorizationFailed
                AuthenticationFailed
                Timeout
                NoSuccess
      Notes:    - server location and possibly protocol are
                  described by the input URL - see description
                  above.
                - the 'url' can be empty (which is the default).
                  A stream so constructed is only to be used
                  as parameter to an asynchronous
                  stream_server::serve() call.  For such a
                  stream, a later call to connect() will fail.  
                - the implementation MUST ensure that the
                  information given in the URL are usable -
                  otherwise a 'BadParameter' exception MUST be
                  thrown.
                - the socket is only connected after the
                  connect() method is called.
 
 
    - DESTRUCTOR
      Purpose:  destroy a stream object
      Format:   DESTRUCTOR           (in stream obj)
      Inputs:   obj:                  stream to destroy
      InOuts:   -
      Outputs:  -
      PreCond:  -
      PostCond: - the socket is closed.
      Perms:    -
      Throws:   -
      Notes:    - if the instance was not closed before, the 
                  destructor performs a close() on the instance,
                  and all notes to close() apply.
 
 
    Inspection methods:
    -------------------
 
    - get_url
      Purpose:  get URL used for creating the stream
      Format:   get_url               (out url       url);
      Inputs:   -
      InOuts:   -
      Outputs:  url:                  the URL of the connection.
      PreCond:  -
      PostCond: -
      Perms:    -
      Throws:   NotImplemented
                IncorrectState
                PermissionDenied
                AuthorizationFailed
                AuthenticationFailed
                Timeout
                NoSuccess
      Notes:    - returns a URL which can be passed to a
                  stream constructor to create another
                  connection to the same stream_server.
                - the returned url may be empty, indicating that
                  this instance has been created with an empty
                  url as parameter to the stream CONSTRUCTOR().
 
 
    - get_context
      Purpose:  return remote authorization info
      Format:   get_context           (out context ctx);
      Inputs:   -
      InOuts:   -
      Outputs:  ctx:                   remote context
      PreCond:  - the stream is, or has been, in the 'Open' 
                  state.
      PostCond: - the returned context is deep copied, and does
                  not share state with any other object.
      Perms:    -
      Throws:   NotImplemented
                IncorrectState
                PermissionDenied
                AuthorizationFailed
                AuthenticationFailed
                Timeout
                NoSuccess
      Notes:    - the context returned contains the security
                  information from the REMOTE party, and can be 
                  used for authorization.
                - if the stream is in a final state, but has
                  been in 'Open' state before, the returned
                  context represents the remote party the stream
                  has been connected to while it was in 'Open'
                  state.
                - if the stream is not in 'Open' state, and is
                  not in a final state after having been in
                  'Open' state, an 'IncorrectState' exception is
                  thrown.
                - if no security information are available, the
                  returned context has the type 'Unknown' and no
                  attributes are attached.
                - the returned context MUST be authenticated, or
                  must be of type 'Unknown' as described above.
 
 
    Management methods:
    -------------------
 
    - connect
      Purpose:  Establishes a connection to the target defined
                during the construction of the stream.
-     Format:   connect              (void);
+     Format:   connect              (in float timeout = -1.0);
!     Inputs:   timeout:              connection timeout
      InOuts:   -
      Outputs:  -
      PreCond:  - the stream is in 'New' state.
      PostCond: - the stream is in 'Open' state.
      Perms:    Exec for the stream_server represented by the
                url used for creating this stream instance.
      Throws:   NotImplemented
                IncorrectState
                PermissionDenied
                AuthorizationFailed
                AuthenticationFailed
                Timeout
                NoSuccess
      Notes:    - on failure, the stream state is changed to
                  'Error'
+               - this call is equivalent to creating a
+                 stream_server instance with the URL used ot
+                 create this stream instance, and calling
+                 connect() on that stream_server.
                - if the stream instance is not in 'New' state,
                  an 'IncorrectState' exception is thrown.
+               - for timeout semantics, see Section 2.
 
 
    - close
      Purpose:  closes an active connection
      Format:   close                (in float timeout)
      Inputs:   timeout               seconds to wait
      InOuts:   -
      Outputs:  -
      PreCond:  -
      PostCond: - stream is in 'Closed' state
      Perms:    -
      Throws:   NotImplemented
                IncorrectState
                NoSuccess
      Notes:    - any subsequent method call on the object
                  MUST raise an 'IncorrectState' exception
                  (apart from DESTRUCTOR and close()).
                - if close() is implicitly called in the
                  DESTRUCTOR, it will never throw an exception.
                - close() can be called multiple times, with no
                  side effects.
                - for resource deallocation semantics, see 
                  Section 2.
                - for timeout semantics, see Section 2.
 
 
    Stream I/O methods:
    -------------------
 
    - read
      Purpose:  Read a data buffer from stream.
      Format:   read                 (inout buffer      buf,
                                      in    int         len_in = -1,
                                      out   int         len_out);
      Inputs:   len_in:               Maximum number of bytes
                                      that can be copied into
                                      the buffer.
      InOuts:   buf:                  buffer to store read data 
                                      into
      Outputs:  len_out:              number of bytes read, if
                                      successful. 
      PreCond:  - the stream is in 'Open' state.
      PostCond: - data from the stream are available in the
                  buffer.
      Perms:    Read for the stream_server represented by the
                url used for creating this stream instance.
      Throws:   NotImplemented
                BadParameter
                IncorrectState
                PermissionDenied
                AuthorizationFailed
                AuthenticationFailed
                Timeout
                NoSuccess
      Notes:    - if the stream is blocking, the call waits
                  until data become available.
                - if the stream is non-blocking, the call
                  returns immediately, even if no data are
                  available -- that is not an error condition.
                - the actually number of bytes read into buffer
                  is returned in len_out.  It is not an error
                  to read less bytes than requested, or in fact
                  zero bytes.
                - errors are indicated by returning negative
                  values for len_out, which correspond to
                  negatives of the respective ERRNO error code
                - the given buffer must be large enough to
                  store up to len_in bytes, or managed by the
                  implementation - otherwise a 'BadParameter'
                  exception is thrown.
                - the notes about memory management from the
                  buffer class apply.
                - if len_in is smaller than 0, or not given, 
                  the buffer size is used for len_in.
                  If that is also not available, a
                  'BadParameter' exception is thrown.
                - if the stream is not in 'Open' state, an
                  'IncorrectState' exception is thrown.
                - similar to read (2) as specified by POSIX
 
 
    - write
      Purpose:  Write a data buffer to stream.
      Format:   write                (in  buffer        buf,
                                      in  int           len_in = -1,
                                      out int           len_out);
      Inputs:   len_in:               number of bytes of data in
                                      the buffer
                buffer:               buffer containing data
                                      that will be sent out via
                                      socket
      InOuts:   -
      Outputs:  len_out:              bytes written if successful
      PreCond:  - the stream is in 'Open' state.
      PostCond: - the buffer data are written to the stream.
      Perms:    Write for the stream_server represented by the
                url used for creating this stream instance.
      Throws:   NotImplemented
                BadParameter
                IncorrectState
                PermissionDenied
                AuthorizationFailed
                AuthenticationFailed
                Timeout
                NoSuccess
      Notes:    - if the stream is blocking, the call waits
                  until the data can be written.
                - if the stream is non-blocking, the call
                  returns immediately, even if no data are
                  written -- that is not an error condition.
                - it is not an error to write less than len_in
                  bytes.
                - errors are indicated by returning negative
                  values for len_out, which correspond to
                  negatives of the respective ERRNO error code
                - the given buffer must be large enough to
                  store up to len_in bytes, or managed by the
                  implementation - otherwise a 'BadParameter'
                  exception is thrown.
                - the notes about memory management from the
                  buffer class apply.
                - if len_in is smaller than 0, or not given, 
                  the buffer size is used for len_in.
                  If that is also not available, a
                  'BadParameter' exception is thrown.
                - if the stream is not in 'Open' state, an
                  'IncorrectState' exception is thrown.
                - similar to write (2) as specified by POSIX
 
 
    - wait
      Purpose:  check if stream is ready for reading/writing, or 
                if it has entered an error state.
      Format:   wait                 (in  int      what,
                                      in  float    timeout,
                                      out int      cause);
      Inputs:   what:                 activity types to wait for
                timeout:              number of seconds to wait
      InOuts:   -
      Outputs:  cause:                activity type causing the
                                      call to return
      PreCond:  - the stream is in 'Open' state.
      PostCond: - the stream can be read from, or written to, or
                  it is in 'Error' state.
      Perms:    -
      Throws:   NotImplemented
                IncorrectState
                PermissionDenied
                AuthorizationFailed
                AuthenticationFailed
                NoSuccess
      Notes:    - wait will only check on the conditions 
                  specified by 'what'
                - 'what' is an integer representing 
                   OR'ed 'Read', 'Write', or 'Exception' flags.
                - 'cause' describes the availability of the 
                  socket (e.g. OR'ed 'Read', 'Write', or 
                  'Exception')
                - for timeout semantics, see Section 2.
                - if the stream is not in 'Open' state, an
                  'IncorrectState' exception is thrown.
 \end{myspec}
 
 
 \subsubsection{Examples}
 
 \begin{mycode}
  Sample SSL/Secure Client:
  -------------------------
 
    Opens a stream connection using native security: the 
    context is passed in implicitly via the default SAGA
    session's contexts.
 
    // C++/JAVA Style
       ssize_t recvlen;
       saga::buffer b;
       saga::stream s ("localhost:5000");
 
       s.connect ();
       s.write   (saga::buffer ("Hello World!"));
 
       // blocking read, read up to 128 bytes
       recvlen = s.read (b, 128);
 
 
    /* C Style */
       ssize_t recvlen;
 
       SAGA_stream sock  = SAGA_Stream_open ("localhost:5000");
       SAGA_buffer b_in  = SAGA_Buffer_create ("Hello World");
       SAGA_buffer b_out = SAGA_Buffer_create ("Hello World");
 
       SAGA_Stream_connect (sock);
       SAGA_Stream_write   (sock, b_in);
 
       /* blocking read, read up to 128 bytes */
       recvlen = SAGA_Stream_read (sock, b_ou, 128);
 
 
     c  Fortran Style */
        INTEGER   err,SAGAStrRead,SAGAStrWrite,err
        INTEGER*8 SAGAStrOpen,streamhandle
        CHARACTER buffer(128)
        SAGAStrOpen("localhost:5000",streamhandle)
        call SAGAStrConnect(streamhandle)
        err = SAGAStrWrite(streamhandle,"localhost:5000",12)
        err = SAGAStrRead(streamhandle,buffer,128)
 
 
  Sample Secure Server:
  ---------------------
 
    Once a connection is made, the server can use information
    about the authenticated client to make an authorization
    decision
 
     // c++ example
        saga::stream_server server ("tcp://localhost/5000");
 
        saga::stream client;
 
        // now wait for a connection
        while ( saga::stream::Open != client.get_state () )
        {
          // wait forever for connection
          client = server.serve ();
 
          // get remote security details
          saga::context ctx = client.get_context ();
 
          // check if context type is X509, and if DN is the
          // authorized one
          if ( ctx.type ()              == "X509"       &&
               ctx.get_attribute ("DN") == some_auth_dn )
          {
            // allowed - keep open and leave loop
            client.write (saga::buffer ("Hello!"));
          }
          else
          {
            client.close (); // not allowed
          }
        } 
 
        // start activity on client socket...
 
 
  Example for async stream server
  -------------------------------
 
    // c++ example
    class my_cb : public saga::callback
    {
      privat:
        saga::stream_server ss;
        saga::stream        s;
 
      public:
 
        my_cb (saga::stream_server ss_,
               saga::stream        s_)
        {
          ss = ss_;
          s   = s_;
        }
 
        bool cb (saga::monitorable mt,
                 saga::metric      m,
                 saga::context     c)
        {
          s = ss.serve ();
          return (false); // want to be called only once
        }
     }
 
     int main ()
     {
       saga::stream_server ss;
       saga::stream        s;

       my_cb cb (ss, s);
 
       ss.add_callback ("client_connect", cb);
 
       while ( true )
       {
         if ( s.state != saga::stream::Open )
         {
           // no client, yet
           sleep (1);
         }
         else
         {
           // handle open socket
           s.write ("Hello Client\r\n", 14);
           s.close ();
 
           // restart listening
           ss.add_callback ("client_connect", cb);
         }
       }
 
       return (-1); // unreachable
     }
 \end{mycode}
 
