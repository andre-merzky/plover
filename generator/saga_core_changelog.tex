
  This appendix lists the errata changes which have been applied to
  the originally published SAGA Core API specification.  As most
  changes are not breaking backward compatibility, the version number
  of this document has not been changed, and remains 1.0.


  \subsection*{Errata to SAGA Core API Version 1.0}

  \begin{shortlist}

  \item The |context| c'tor does not call |set_defaults()| anymore.
  In fact, the method |set_defaults()| is gone, and its functionality
  is now performed on |session.add_context()| -- the original context
  is left untouched (as it is specified that |add_context()| performs
  a deep copy on the context to be added).  \B{This is the only change
  which breakes backward compatibility.}
  
  % complicates usage if default ctx cannot be initialized.
  
  \item Typos, spelling and grammar have been fixed in several
  places.  These fixes are not listed individually.

  \item The biggest change is that saga exceptions are now
  recursive objects, i.e. they can provide a list of lower
  level exceptions.  That change is backward compatible, and
  is introduced mainly for the sake of late binding
  implementations.  At the same time, the stricly prescribed
  exception precedence has been relaxed, and can be changed by
  the implementation.  The |NotImplemented| exception now has
  lowest precedence.  This change is also backward compatible.

  \item It has been clarified what \LF classes MUST be implemented,
  and when |NotImplemented| exceptions MAY be thrown.

  \item The |read_v()| method now throws |BadParameter| when "out of
  bounds": when  no |len_in| is specified, the buffer size is used
  instead as |len_in|.  If, in this case, |offset > 0|, a
  |BadParameter| exception is thrown.

  \item |write_v| method now throws |BadParameter| when "out of
  bounds": when  no |len_in| is specified, the buffer size is used
  instead as |len_in|.  If, in this case, |offset > 0|, a
  |BadParameter| exception is thrown.

  \item The default flag for file |open()| is now |Read|.

  \item The |Create| flag now implies |Write|.

  \item The |CreateParents| flag now implies |Create|.

  \item Callbacks now can remove conditions to be called again, i.e.
  shut down the metric, read more than one message, etc.
  Implementations MUST be able to handle this.

  \item The  URL behaviour for relative path elements, and their time
  of expansion, has been clarified.

  \item |task.get_result()| now calls |rethrow()| if the task is in
  |Failed| state.

  \item The |url.get_xxxx()| methods return an empty string on
  undefined or unknown values, or |-1| for |get_port()|.
  
  \item |JobProject| and |WallTimeLimit| have been added to the
  |job_description| attributes.
  
  \item The |run()| postcondition is now 'left New state' instead of
  'is in Running state', to avoid races with jobs entering a final
  state immediately.
  
  \item The url class was added to the list of \LF packages/classes in
  paragraph 6 on page 17.
  
  \item The behaviour of |get_link()| has been clarifies: it resolves
  only one level.
  
  \item The |namespace| package got |Read| and |Write| flags, as they
  are needed for directories.
  
  \item URL escaping has been clarified, and a |get_escaped()|
  method has been added, to enforce character escaping.
  
  \item |close()| is not throwing |IncorrectState| anymore.
  
  \item |object.clone()| does not copy the object id anymore, but
  assigns a new, unique one.
  
  \item On page 225, the notes of |NSDirectory.copy (source, target, flags)| 
  have been fixed.
  
  \item The RPC c'tor signature has been fixed (parameter name).
  
  \item The signature for |task_container.wait ()| has been fixed
  (default timeout value was missing).
  
  \item The |url| class was added to the class diagram, and the
  |iovec| and |parameter| classes have been moved into their
  respective packages.
  
  \item The size parameter in the |rpc| c'tor has been fixed.
  
  \item |Exception| has been removed from |object::type| enum.
  
  \item A |saga::job| now provides the |ServiceURL| attribute which
  allows the re-creation of the |job::service| instance which handles
  the job.
  
  \item A |url::translate()| variant with explicit session parameter
  has been added.
  
  \item |session.list_contexts()| now returns deep copies of session
  contexts, not shallow copies.  

  \end{shortlist}

