 
  The session object provides the functionality of a session,
   which isolates independent sets of SAGA objects
  from each other.  Sessions also support the management of
  security information (see |saga::context| in
  Section~\ref{ssec:context}).
 
 
 \subsubsection{Specification}
 
 
 \begin{myspec}
  package saga.core
  {
    class session : implements   saga::object
                 // from object  saga::error_handler
    {
      CONSTRUCTOR        (in  bool             default = true,
                          out session          obj);
      DESTRUCTOR         (in  session          obj);
 
      add_context        (in  context          context);
      remove_context     (in  context          context);
      list_contexts      (out array<context,1> contexts);
    }
  }
 \end{myspec}
 
 
 \subsubsection{Specification Details}
 
 \subsubsection*{Class \T{session}}
 
    Almost all SAGA objects are created in a SAGA session, and
    are associated with this (and only this) session for their
    whole life time.
 
    A session instance to be used on object
    instantiation can explicitly be given as
    first parameter to the SAGA object instantiation call
    (|CONSTRUCTOR|).
 
    If the session is omitted as first parameter, a
    default session is used, with default security
    context(s) attached.
    The default session can be obtained by passing
    \T{true} to the session \T{CONSTRUCTOR}.
 
 \newpage

 \begin{mycode}
      // Example in C++:
 
      // create a file object in a specific session:
      saga::file f1 (session, url);
 
      // create a file object in the default session:
      saga::file f2 (url);
 \end{mycode}
 
 
    SAGA objects created from 
    another SAGA object inherit its
    session, such as, for example, |saga::streams| from
    |saga::stream_server|.  Only some objects do not need a
    session at creation time, and can hence be shared
    between sessions.  These include:\up
 
    \begin{tabbing}
     XXXX \= \kill
          \> |saga::exception|\\
          \> |saga::buffer|\\
          \> |saga::iovec|\\
          \> |saga::parameter|\\
          \> |saga::context|\\
          \> |saga::job_description|\\
          \> |saga::metric|\\
          \> |saga::exception|\\
          \> |saga::task|\\
          \> |saga::task_container|\\
    \end{tabbing}
    
    \up\up
 
    Note that tasks have no explicit session attached.  The
    |saga::object| the task was created from, however, has a
    |saga::session| attached, and that session
    instance is indirectly available, as the application can
    obtain that object via the \T{get\_object} method call on
    the respective task instance.
 
    Multiple sessions can co-exist. 
 
    If a |saga::session| object instance gets destroyed, or goes
    out of scope, the objects associated with that session
    survive.  The implementation MUST ensure that the session is
    internally kept alive until the last object of that session gets destroyed.
 
    If the session object instance itself gets
    destroyed, the resources associated with that session MUST
    be freed immediately as the last object associated with that
    session gets destroyed.  The lifetime of the
    default session is, however, only limited by the lifetime of
    the SAGA application itself (see Notes about life time
    management in Section~\ref{ssec:lifetime}).
 
 
    Objects associated with different sessions MUST NOT
    influence each other in any way - for all practical
    purposes, they can be considered to be running in different
    application instances.
 
    
 
    Instances of the |saga::context| class (which encapsulates
    security information in SAGA) can be attached to a
    |saga::session| instance.  The context instances are to be
    used by that session for authentication and authorization to
     the backends used.
 
    If a |saga::context| gets removed from a session, but that
    context is already/still used by any object created in that
    session, the context MAY continue to be used by these
    objects, and by objects which inherit the session from these
    objects, but not by any other objects.  However, a call to
    |list_contexts| MUST NOT list the removed context after it
    got removed.
 
    For the default \T{session} instance,  the list
    returned by a call to \T{list\_contexts()} MUST include the
    default |saga::context| instances.  These are those
    contexts that are added to any \T{saga::session} by default,
    e.g. because they are picked up by the SAGA implementation
    from the application's run time environment. An application
    can, however, subsequently remove default contexts from the
    default session.  A new, non-default session has initially
    no contexts attached.
 
    A SAGA implementation MUST document which default context
    instances it may create and attach to a saga::session.  That
    set MAY change during runtime, but SHOULD NOT be
    changed once a |saga::session| instance was created.
    For example, two |saga::session| instances might
    have different default |saga::context| instances attached.
    Both sessions, however, will have these attached for their
    complete lifetime  -- unless they expire or get
    otherwise invalidated.
 
    Default |saga::context| instances on a session can be
    removed from a session, with a call to |remove_context()|.
    That may result in a session with no contexts attached.
    That session is still valid, but likely to fail on
    most authorization points.
 
 
 \begin{myspec}
    - CONSTRUCTOR
      Purpose:  create the object
      Format:   CONSTRUCTOR          (in  bool    default = true,
                                      out session obj)
      Inputs:   default:              indicates if the default
                                      session is returned
      InOuts:   -
      Outputs:  obj:                  the newly created object
      PreCond:  -
      PostCond: -
      Perms:    -
-     Throws:   NotImplemented
!     Throws:   NoSuccess
      Notes:    - the created session has no context
                  instances attached.
                - if 'default' is specified as 'true', the
                  constructor returns a shallow copy of the
                  default session, with all the default 
                  contexts attached.  The application can then
                  change the properties of the default session,
                  which is continued to be implicetly used on
                  the creation of all saga objects, unless
                  specified otherwise.
 
 
    - DESTRUCTOR
      Purpose:  destroy the object
      Format:   DESTRUCTOR           (in  session obj)
      Inputs:   obj:                  the object to destroy
      InOuts:   -
      Outputs:  -
      PreCond:  -
      PostCond: - See notes about lifetime management 
                  in Section 2
      Perms:    -
      Throws:   -
      Notes:    -
 
 
    - add_context
      Purpose:  attach a security context to a session
      Format:   add_context          (in context  c);
      Inputs:   c:                    Security context to add
      InOuts:   -
      Outputs:  -
      PreCond:  -
      PostCond: - the added context is deep copied, and no 
                  state is shared.
+               - after the deep copy, the implementation MAY 
+                 try to initialize those context attributes
+                 which have not been explicitely set, e.g. to
+                 sensible default values.  
                - any object within that session can use the
                  context, even if it was created before
                  add_context was called.
      Perms:    -
-     Throws:   NotImplemented
+     Throws:   NoSuccess
+               TimeOut
      Notes:    - if the session already has a context attached
                  which has exactly the same set of attribute
                  values as the parameter context, no action is
                  taken.
+               - if the implementation is not able to
+                 initialize the context, and cannot use the
+                 context as-is, a NoSuccess exception is
+                 thrown.
+               - if the context initialization implies remote
+                 operations, and that operations times out, a
+                 TimeOut exception is thrown.
 
 
    - remove_context
      Purpose:  detach a security context from a session
      Format:   remove_context       (in context  c);
      Inputs:   c:                    Security context to remove
      InOuts:   -
      Outputs:  -
-     Throws:   NotImplemented
!     Throws:   DoesNotExist
      PreCond:  - a context with completely identical attributes
                  is available in the session.
      PostCond: - that context is removed from the session, and
                  can from now on not be used by any object in
                  that session, even if it was created before
                  remove_context was called.
      Perms:    -
      Notes:    - this methods removes the context on the
                  session which has exactly the same set of
                  parameter values as the parameter context.
                - a 'DoesNotExist' exception is thrown if no
                  context exist on the session which has the
                  same attributes as the parameter context.
 

    - list_contexts
      Purpose:  retrieve all contexts attached to a session
      Format:   list_contexts        (out array<context>
                                                  contexts);
      Inputs:   -
      InOuts:   -
      Outputs:  contexts:             list of contexts of this
                                      session
      PreCond:  -
      PostCond: -
      Perms:    -
-     Throws:   NotImplemented
+     Throws:   -
      Notes:    - a empty list is returned if no context is
                  attached.
                - contexts may get added to a session by 
                  default, hence the returned list MAY be 
                  non-empty even if add_context() was never 
                  called before.
                - a context might still be in use even if not
                  included in the returned list.  See notes
                  about context life time above.
+               - the contexts in the returned list MUST be
+                 deep copies of the session's contexts.
 \end{myspec}
 
 
 \subsubsection{Examples}
 
 \begin{mycode}
  // c++ example
  saga::session s;
  saga::context c (saga::context::X509);
 
  s.add_context   (c);
 
  saga::directory  d (s, "gsiftp://remote.net/tmp/");
  saga::file       f = d.open ("data.txt");
 
  // file has same session attached as dir,
  // and can use the same contexts
 \end{mycode}
 
 \begin{mycode}
  // c++ example
  saga::task    t;
  saga::session s;
 
  {
    saga::context c ("X509");
 
    s.add_context (c);
 
    saga::file f (s, url);
 
    t = f.copy <saga::task::Task> (target);
 
    s.remove_context (c);
  }
 
  // As it leaves the scope, the X509 context gets 'destroyed'.
  // However, the copy task and the file object MAY continue to
  // use the context, as its destruction is actually delayed
  // until the last object using it gets destroyed.
 
  t.run (); // can still use the context
 \end{mycode}
 
