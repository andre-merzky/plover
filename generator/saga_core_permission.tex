 
 A number of SAGA use cases imply the ability of applications to
 allow or deny specific operations on SAGA objects or grid
 entities, such as files, streams, or monitorables.  This
 packages provides a generic interface to query and set such
 permissions, for (a) everybody, (b) individual users, and (c)
 groups of users.
 
 
 Objects implementing this interface maintain a set of
 permissions for each object instance, for a set of IDs.  These
 permissions can be queried, and, in many situations, set.  The
 SAGA specification defines which permissions are available on a
 SAGA object, and which operations are expected to respect these
 permissions.
 
 A general problem with this approach is that it is difficult to
 anticipate how users and user groups are identified by various
 grid middleware systems.  In particular, any translation of
 permissions specified for one grid middleware is likely not
 completely translatable to permissions for another grid
 middleware.  
 
 For example, assume that a |saga::file| instance gets created
 via ssh, and permissions are set for the file to be readable
 and executable by a specific POSIX user group ID.  Which
 implications do these permissions have with respect to operations
 performed with GridFTP, using a Globus certificate?  The
 used X509 certificates have (a) no notion of groups (groups are
 implicit due to the mapping of the |grid-mapfile|), and (b) are
 not mappable to group ids; and (c) GridFTP ignores the
 executable flag on files.
 
 For this reason, it is anticipated that the permission model
 described in this section has the following, undesired
 consequences and limitations:
 
 \begin{shortlist}
 
  \item Applications using this interface are not expected to be
  fully portable between different SAGA implementations.
  (In cases like having two SAGA implementations that use different
  middleware backends for accessing the same resources.)
 
  \item A SAGA implementation MUST document which permission it
  supports, for which operations.
 
  \item A SAGA implementation MUST document if it supports group
  level permissions.
 
  \item A SAGA implementation MUST document how user and group
  IDs are to be formed.
 
 \end{shortlist}
 
 Note that there are no separate calls to get/set user, group
 and world permissions: this information must be part of the
 IDs the methods operate upon.  To set/get permissions for
 'world' (i.e.  anybody), the ID |'*'| is used.
 
 \subsubsection*{IDs}
 
  SAGA can not, by design, define globally unique identifiers in
  a portable way.  For example, it would be impossible to map,
  transparently and bi-directionally, a Unix user ID and an
  associated X509 Distinguished Name on any resource onto the
  same hypothetical SAGA user ID, at least not without explicit
  support by the grid middleware (e.g., by having access to the
  Globus |grid-mapfile|).  That support is, however, rarely
  available.
 
  It is thus required that SAGA implementations MUST specify how the
  user and group IDs are formed that they support.  In general, IDs
  which are valid for the \T{UserID} attribute of the SAGA
  \T{context} instances SHOULD also be valid IDs to be used in
  the SAGA permission model.
 
  A typical usage scenario is (extended from the context usage
  scenario):
 
  \begin{mycode}
   // context and permission usage scenario in C++
 
   saga::context c_1 ("globus")
   saga::context c_2 ("ssh");
 
   // c_1 is a globus proxy.  Identify the proxy to be used, 
   // and pick up the other default globus settings
   c_1.set_attribute ("UserProxy", "/tmp/special_x509up_u500");
   c_1.set_defaults  (); 
 
   // c_2 is a ssh context, and will just pick up the
   // public/private key from $HOME/.ssh
   c_2.set_defaults (); 
 
   // a saga session gets created, and uses both contexts
   saga::session s;
   s.add_context (c_1);
   s.add_context (c_2);
 
   // a remote file in this session can now be accessed via 
   // gridftp or ssh
   saga::file f (s, "any://remote.net/tmp/data.txt");
   f.copy ("data.bak");
 
   // write permissions can be set for both context IDs
   f.permission_allow (c_1.get_attribute ("UserID"), Write);
   f.permission_allow (c_2.get_attribute ("UserID"), Write);
  \end{mycode}
 
 
 
  For middleware systems where group and user ids can clash, the
  IDs should be implemented as |'user-<id>'| and |'group-<id>'|.
  For example: on Unix, the name 'mail' can (and often does)
  refer to a user and a group.  In that case, the IDs should be
  expressed as |'user-mail'| and |'group-mail'|, respectively.
  The ID |'*'| is always reserved, as described above.
 
  Permissions for a user ID supersede the permissions for a
  group ID, which supersede the permissions for |'*'| (all).  If
  a user is in multiple groups, and the group's permissions
  differ, the most permissive permission applies.
 
 
 \subsubsection{Permissions for Multiple Backends}
 
  In SAGA, an entity which provides the |permissions| interface
  always has exactly one owner, for one middleware backend.
  However, this implies that for SAGA implementations with
  multiple backend bindings, multiple owner IDs may be valid.
  For example, |"/O=dutchgrid/O=users/O=vu/OU=cs/CN=Joe Doe"|
  and |"user-jdoe"| might be equally valid IDs, at the same
  time, if the implementation supports local Unix access and
  GridFTP access to a local file.  As long as the ID spaces do
  not conflict, the |permissions| interface obviously allows to
  set permissions individually for both backends. In case of conflicts,
  the application would need to create new SAGA objects from
  sessions that contain only a single context, representing the
  desired backend's security credentials.  As such situations are
  considered to be very rare exceptions in the known SAGA use cases, we
  find this limitation accetable.
 
  Note that, for SAGA implementations supporting multiple
  middleware backends, the |permissions| interface can operate
  on permissions for any of these backends, not only for the one
  that was used by the original creation of the object
  instance.  Such a restriction would basically inhibit implementations
  with dynamic (``late'') binding to backends.
 
 
 \subsubsection*{Conflicting Backend Permission Models}
 
  Some middleware backends may not support the full range of
  permissions, e.g., they might not distinguish between |Query| and
  |Read| permissions.  A SAGA implementation MUST document which
  permissions are supported.  Trying to set an unsupported
  permission reults in a |BadParameter| exception, and NOT in
  a |NotImplemented| exception -- that would indicate that the
  method is not available at all, i.e. that no permission model at all
  is available for this particular implementation.
 
  \newpage
 
  An implementation MUST NOT silently merge permissions,
  according to its own model -- that would break for example the
  following code:
 
  \shift |file.permissions_allow ("user-jdoe", Query);|\\
  \shift |file.permissions_deny  ("user-jdoe", Read );|\\
  \shift |off_t file_size = file.get_size ();|
 
  If an implementation binds to a system with standard Unix
  permissions and does not throw a |BadParameter| exception on
  the first call, but silently sets |Read| permissions instead,
  because that does also allow query style operations on Unix,
  then the code in line three would fail for no obvious reason,
  because the second line would revoke the permissions from line
  one.
 
 
 \subsubsection*{Initial Permission Settings}
 
  If new grid entities get created via the SAGA API, the owner
  of the object is set to the value of the |'UserID'| attribute
  of the context used during the creation.  Note that for SAGA
  implementations with support for multiple middleware backends,
  and support for late binding, this may imply that the owner is
  set individually for one, some or all of the supported
  backends.
 
  Creating grid entities may require specific permissions on
  other entities.  For example:
 
  \begin{shortlist}
   \item file creation requires |Write| permissions on the 
         parent directory.
   \item executing a file requires |Read| permissions on the
         parent directory.
  \end{shortlist}
 
 
  An implementation CAN set initial permissions other than
  |Owner|.  An implementation SHOULD document which initial
  permission settings an application can expect.
 
  The specification of the |ReadOnly| flag on the creation or
  opening of SAGA object instances, such as |saga::file|
  instances, causes the implementation to behave as if the
  |Write| permission on the entity on that instance is not
  available, even if it is, in reality, available.  The same
  holds for the |WriteOnly| flag and the availability of the
  |Read| permission on that entity.
 
 \subsubsection*{Permission Definitions in the SAGA specification}
 
  The SAGA specification normatively defines for each operation,
  which permissions are required for that operation.  If a
  permission is supported, but not set, the method invocation
  MUST cause a |PermissionDenied| exception.  An implementation
  MUST document any deviation from this scheme, e.g., if a
  specified permission is not supported at all, or cannot be
  tested for a particular method.  An example of such a definition is
  (from the |monitorable| interface):
 
  \begin{myspec}
    - list_metrics
      Purpose:  list all metrics associated with the object
      Format:   list_metrics       (out array<string>   names);
      Inputs:   -
      InOuts:   -
      Outputs:  names:              array of names identifying
                                    the metrics associated with
                                    the object instance
      PreCond:  -
      PostCond: -
      Perms:    Query
      Throws:   NotImplemented
                PermissionDenied
                AuthorizationFailed
                AuthenticationFailed
                Timeout
                NoSuccess
      Notes:    - [...]
  \end{myspec}
 
  This example implies that for the session in which the
  |list_metrics()| operation gets performed, there must be at least
  one context for which's attribute |'UserID'| the |Query|
  permission is both supported and available; otherwise, the method
  MUST throw a |PermissionDenied| exception.  If |Query| is not
  supported by any of the backends for which a context exists,
  the implementation MAY try the backends to perform the
  operation anyway.
 
  For some parts of the specification, namely for attributes and
  metrics, the |mode| specification is normative for the
  respective, required permission.  For example, the mode
  attribute |ReadOnly| implies that a |Write| permission,
  required to change the attribute, is never available.
 
 
 \subsubsection*{The \T{PermissionDenied} exception in SAGA}
 
  SAGA supports a |PermissionDenied| exception, as documented in
  Section~\ref{ssec:error}.  This exception can originate from
  various circumstances, that are not necessarily related to
  the SAGA permission model as described here.  However, if
  the reason why that exception is raised maps onto the SAGA
  permission model, the exception's error message MUST have the
  following format (line breaks added for readability):
 
  \shift |PermissionDenied: no <PERM> permission|\\
  \shift\shift\shift\shift\, |on <ENTITY> <NAME>|\\
  \shift\shift\shift\shift\, |for <ID>|
 
  Here, |<PERM>| denotes which permission is missing, |<ENTITY>|
  denotes on what kind of entity this permission is missing.
  |<NAME>| denotes which entity misses that permission, and
  |<ID>| denotes which user is missing that permission.  
  
  |<PERM>| is the literal string of the |permission| enum
  defined in this section.  |<ENTITY>| is the type of backend
  entity which is missing the permission, e.g. |file|,
  |directory|, |job_service| etc.  Whenever possible, the
  literal class name of the respective SAGA class name SHOULD be
  used.  |<NAME>| SHOULD be a URL or literal name which allows
  the end user to uniquely identify the entity in question.
  |<ID>| is the value of the |UserID| attribute of the context
  used for the operation (the notes about user IDs earlier in
  this section apply).
 
  Some examples for complete error messages are:
 
  \shift |PermissionDenied: no Read permission |\\
  \shift\shift\shift\shift\, |on file http:////tmp/test.dat|\\
  \shift\shift\shift\shift\, |for user-jdoe|\\[1em]
  \shift |PermissionDenied: no Write permission |\\
  \shift\shift\shift\shift\, |on directory http:////tmp/|\\
  \shift\shift\shift\shift\, |for user-jdoe|\\[1em]
  \shift |PermissionDenied: no Query permission |\\
  \shift\shift\shift\shift\, |on logical_file rls:////tmp/test|\\
  \shift\shift\shift\shift\, |for /O=ca/O=users/O=org/CN=Joe Doe|\\[1em]
  \shift |PermissionDenied: no Query permission |\\
  \shift\shift\shift\shift\, |on job [fork://localhost]-[1234]|\\
  \shift\shift\shift\shift\, |for user-jdoe|\\[1em]
  \shift |PermissionDenied: no Exec permission |\\
  \shift\shift\shift\shift\, |on RPC [rpc://host/matmult] for |\\
  \shift\shift\shift\shift\, |for /O=ca/O=users/O=org/CN=Joe Doe|
 
 \subsubsection*{Note to users}
 
  The description of the SAGA permission model above should have
  made clear that, in particular, the support for multiple
  backends makes it difficult to strictly enforce the
  permissions specified on application level.  Until a standard
  for permission management for Grid application emerges, this
  situation is unlikely to change.  Applications should thus be
  careful to trust permissions specified in SAGA, and should
  ensure to use an implementation which fully supports and
  enforces the permission model, e.g., they should choose an
  implementation which binds to a single backend.
 
 \subsubsection{Specification}
 
 \begin{myspec}
  package saga.core
  {
    enum permission
    {
      None      =  0,
      Query     =  1,
      Read      =  2,
      Write     =  4,
      Exec      =  8,
      Owner     = 16,
      All       = 31
    }
 
    interface permissions : implements saga::core::async
    {
      // setter / getters
      permissions_allow       (in  string          id,
                               in  int             perm);
      permissions_deny        (in  string          id,
                               in  int             perm);
      permissions_check       (in  string          id,
                               in  int             perm,
                               out bool            value);
      get_owner               (out string          owner);
      get_group               (out string          group);
    }
  }
 \end{myspec}
 
 
 \subsubsection{Specification Details}
 
 \subsubsection*{Enum \T{permission}}
 
  This enum specifies the available permissions in SAGA.  The
  following examples demonstrate which type of operations are
  allowed for certain permissions, and which aren't.  To keep
  these examples concise, they are chosen from the following
  list, with the convention that those operations in this list,
  which are not listed in the respective example section, are
  \I{not} allowed for that permission.  In general, the
  availability of one permission does not imply the availability
  of any other permission (with the exception of |Owner|, as
  described below).
 
  \begin{shortlist}
   \item provide information about a metric, and its properties
   \item provide information about file size, access time and ownership
   \item provide information about job description, ownership, and runtime
   \item provide information about logical file access time and ownership
   \item provide access to a job's I/O streams
   \item provide access to the list of replicas of a logical file
   \item provide access to the contents of a file
   \item provide access to the value of a metric
   \item provide means to change the ownership of a file or job
   \item provide means to change the permissions of a file or job
   \item provide means to fire a metric
   \item provide means to connect to a stream server
   \item provide means to manage the entries in a directory
   \item provide means to manipulate a file or its meta data
   \item provide means to manipulate a job's execution or meta data 
   \item provide means to manipulate the list of replicas of a logical file
   \item provide means to run an executable
  \end{shortlist}
 
 The following permissions are available in SAGA:
 
 \begin{list}{}{}
 
  \item{\sunshift|Query|}
  
  This permission identifies the ability to \I{access all meta
  data of an entity}, and thus to obtain any information about
  an entity.  If that permission is not available for an actor,
  that actor MUST NOT be able to obtain any information about
  the queried entity, if possible not even about its existence.
  If that permission is available for an actor, the actor MUST
  be able to query for any meta data on the object which (a) do
  imply changes on the entities state, and (b) are part of the
  \I{content} of the entity (i.e., do not comprise its data).
 
  Note that for logical files, attributes are part of the data
  of the entities (i.e., the meta data belong to the logical
  file's data).
 
  An authorized |Query| operation can:
 
  \begin{shortlist}
   \item provide information about a metric, and its properties
   \item provide information about file size, access time and ownership
   \item provide information about job description, ownership, and runtime
   \item provide information about logical file access time and ownership
  \end{shortlist}
 
 
  \item{\sunshift|Read|}
 
  This permission identifies the ability to \I{access the
  contents and the output of an entity}.  If that permission is
  not available for an actor, that actor MUST NOT be able to
  access the data of the entity.  That permission does not imply
  the authorization to change these data, or to manipulate the
  entity.  That permission does also not imply |Query|
  permissions, i.e. the permission to access the entity's meta
  data.
 
  An authorized |READ| operation can:
 
  \begin{shortlist}
   \item provide access to a job's I/O streams
   \item provide access to the list of replicas of a logical file
   \item provide access to the contents of a file
   \item provide access to the value of a metric
  \end{shortlist}
 
 
  \item{\sunshift|Write|}
 
  This permission identifies the ability to \I{manipulate the
  contents of an entity}.  If that permission is not
  available for an actor, that actor MUST NOT be able to change
  neither data nor meta data of the entity.  That permission
  does not imply the authorization to read these data of the
  entity, nor to manipulate the entity.  That permission does
  also not imply |Query| permissions, i.e., the permission to
  access the entity's meta data.
 
  Note that, for a directory, its entries comprise its data.
  Thus, |Write| permissions on a directory allow to manipulate
  all entries in that directory -- but do not imply the ability
  to change the data of these entries.  For example, |Write|
  permissions on the directory |'/tmp'| allows to move
  |'/tmp/a'| to |'/tmp/b'|, or to remove these entries, but does
  not imply the ability to perform a |read()| operation on
  |'/tmp/a'|.
 
  An authorized |Write| operation can:
 
  \begin{shortlist}
   \item provide means to manage the entries in a directory
   \item provide means to manipulate a file or its meta data
   \item provide means to manipulate a job's execution or meta data
   \item provide means to manipulate the list of replicas of a logical file
  \end{shortlist}
 
 
  \item{\sunshift|Exec|}
 
  This permission identifies the ability to \I{perform an action
  on an entity}.  If that permission is not available for an
  actor, that actor MUST NOT be able to perform that action.
  The actions covered by that permission are usually those which
  affect the state of the entity, or which create a new entity.
 
  An authorized |Exec| operation can:
 
  \begin{shortlist}
   \item provide means to fire a metric
   \item provide means to connect to a stream server
   \item provide means to run an executable
  \end{shortlist}
 
 
  \item{\sunshift|Owner|}
 
  This permission identifies the ability to \I{change
  permissions and ownership of an entity}.  If that permission
  is not available for an actor, that actor MUST NOT be able to
  change any permissions or the ownership of an entity.  As this
  permission indirectly implies full control over all other
  permissions, it does also imply that an actor with that
  permission can perform \I{any} operation on the entity.
  |Owner| is not listed as additional required permission in the
  specification details for the individual methods, but only
  listed for those methods, where |Owner| is an explicit
  permission requirement which cannot be replaced by any other
  permission.
 
  An authorized |Owner| operation can:
 
  \begin{shortlist}
   \item provide means to change the ownership of a file or job
   \item provide means to change the permissions of a file or job
   \item perform \I{any} other operation, including all
         operations from the original list of examples above
  \end{shortlist}
 
  Note that only one user can own an entity.  For example, the
  following sequence:
 
   \shift |file.permissions_allow ("Tarzan", saga::permission::Owner);|
   \shift |file.permissions_allow ("Jane",   saga::permission::Owner);|
 
  would result in a file ownership by |'Jane'|.  
 
  Also note that 
 
   \shift |file.permissions_allow ("*", saga::permission::Owner);|
 
  or 
 
   \shift |file.permissions_deny (id, saga::permission::Owner);|
 
  will never be possible, and will throw a |BadParameter|
  exception.
 
 \end{list}
 
 
 \subsubsection*{Interface \T{permissions}}
 
 \begin{myspec}
    - permissions_allow
      Purpose:  enable permission flags
      Format:   permissions_allow    (in  string     id,
                                      in  int        perm);
      Inputs:   id:                   id to set permission for
                perm:                 permissions to enable
      InOuts:   -
      Outputs:  -
      PreCond:  -
      PostCond: - the permissions are enabled.
      Perms:    Owner
      Throws:   NotImplemented
                BadParameter
                PermissionDenied
                AuthorizationFailed
                AuthenticationFailed
                Timeout
                NoSuccess
      Notes:    - an id '*' sets the permissions for all (world)
                - whether an id is interpreted as a group id is up to
                  the implementation.  An implementation MUST
                  specify how user and group id's are formed.
                - the 'Owner' permission can not be set to the
                  id '*' (all).
                - if the given id is unknown or not supported, a
                  'BadParameter' exception is thrown.
 
 
    - permissions_deny
      Purpose:  disable permission flags
      Format:   permissions_deny     (in  string     id,
                                      in  int        perm);
      Inputs:   id:                   id to set permissions for
                perm:                 permissions to disable
      InOuts:   -
      Outputs:  -
      PreCond:  -
      PostCond: - the permissions are disabled.
      Perms:    Owner
      Throws:   NotImplemented
                BadParameter
                PermissionDenied
                AuthorizationFailed
                AuthenticationFailed
                Timeout
                NoSuccess
      Notes:    - an id '*' sets the permissions for all (world)
                - whether an id is interpreted as a group id is up to
                  the implementation.  An implementation MUST
                  specify how user and group id's are formed.
                - the 'Owner' permission can not be set to the
                  id '*' (all).
                - if the given id is unknown or not supported, a
                  'BadParameter' exception is thrown.
 
 
    - permissions_check
      Purpose:  check permission flags
      Format:   permissions_check    (in  string     id,
                                      in  int        perm,
                                      out bool       allow);
      Inputs:   id:                   id to check permissions for
                perm:                 permissions to check
      InOuts:   -
      Outputs:  allow:                indicates if, for that id,
                                      the permissions are granted
                                      (true) or not.
      PreCond:  -
      PostCond: -
      Perms:    Query
      Throws:   NotImplemented
                BadParameter
                PermissionDenied
                AuthorizationFailed
                AuthenticationFailed
                Timeout
                NoSuccess
      Notes:    - an id '*' gets the permissions for all (world)
                - 'true' is only returned when all permissions
                  specified in 'perm' are set for the given id.
                - if the given id is unknown or not supported, a
                  'BadParameter' exception is thrown.
 
 
    - get_owner 
      Purpose:  get the owner of the entity
      Format:   get_owner            (out string     owner);
      Inputs:   -
      InOuts:   -
      Outputs:  owner:                id of the owner
      PreCond:  -
      PostCond: -
      Perms:    Query
      Throws:   NotImplemented
                PermissionDenied
                AuthorizationFailed
                AuthenticationFailed
                Timeout
                NoSuccess
      Notes:    - returns the id of the owner of the entity
                - an entity, on which the permission interface is
                  available, always has exactly one owner: this 
                  method MUST NOT return an empty string, and 
                  MUST NOT return '*' (all), and MUST NOT return 
                  a group id.
 
 
    - get_group 
      Purpose:  get the group owning the entity
      Format:   get_group            (out string     group);
      Inputs:   -
      InOuts:   -
      Outputs:  group:                id of the group
      PreCond:  -
      PostCond: -
      Perms:    Query
      Throws:   NotImplemented
                PermissionDenied
                AuthorizationFailed
                AuthenticationFailed
                Timeout
                NoSuccess
      Notes:    - returns the id of the group owning the entity
                - this method MUST NOT return '*' (all), and 
                  MUST NOT return a user id.
                - if the implementation does not support groups, 
                  the method returns an empty string.
 \end{myspec}
 
 
 \subsubsection{Examples}
 
 \begin{mycode}
 
  // c++ example
  {
    // create a file in the default session
    saga::file f (url, saga::file::Create 
                     | saga::file::Exclusive):
 
    // get all contexts of the default session, and for each...
    std::list <saga::context> ctxs = theSession.list_contexts ();
 
    for ( int i = 0; i < ctxs.size (); i++ )
    {
      saga::context ctx = ctxs[i];
 
      // set the file to be executable
      f.permission_allow (ctx.get_attribute ("UserID"),
                          saga::permission::Exec);
    }
 
    // the file should now be usable for job submission for all
    // contexts in the default session.  Often, however, only 
    // one context will succeed in setting the permission: the 
    // one which was used for creation in the first place.  In 
    // that case, job submission is most likely to succeed with 
    // that context, too.
  }
 \end{mycode}
 
