 
  This document specifies SAGA CORE, the Core of the
  \emph{Simple API for Grid Applications}.  SAGA %has been
  % defined 
  is a high-level API that directly addresses the needs of application
  developers.  The purpose of SAGA is two-fold:
 
  \begin{enumerate}
 
  \item Provide an {\bf simple} API that can be used with much less
    effort compared to the vanilla interfaces of existing grid
    middleware.  A guiding principle for achieving this simplicity is
    the \emph{80--20 rule}: serve 80\,\% of the use cases with 20\,\%
    of the effort needed for serving 100\,\% of all possible
    requirements.
 
  \item Provide a standardized, common interface across various grid
    middleware systems and their versions.
 
  \end{enumerate}
 
  \subsection{How to read this Document}
 
  This document is an API \I{specification}, and as such targets
   \I{implementors of the API}, rather than its end users.
  In particular, this document should not be confused with a SAGA
  Users' Guide.  This document might be useful as an API reference,
  but, in general, the API users' guide and reference should be
  published as separate documents, and should accompany SAGA
  implementations. The latest version of the users guide and
    reference can be found at \url{http://saga.cct.lsu.edu}
   
  An implementor of the SAGA API should read the complete document
  carefully. It will very likely be insufficient{unlikely be
    sufficient} to extract the embedded SIDL specification of the API
  and implement a SAGA-compliant API. In particular,
  the general design considerations in Section~\ref{sec:design} give
  essential, additional information to be taken into account for any
  implementation in order to be SAGA
  compliant.
 
   This document is structured as follows. This Section
   focuses on the formal aspects of an OGF
   recommendation document.  Section~\ref{sec:design} outlines
   the general design considerations of the SAGA API.
   Sections~\ref{sec:nonfunc} and~\ref{sec:func} contain the
   SAGA API specification itself. Section~\ref{sec:disclaimers}
   gives author contact information and provides disclaimers
   concerning intellectual property rights and copyright issues,
   according to OGF policies.  Finally,
   Appendix~\ref{sec:examples} gives illustrative,
   non-normative, code examples of using the SAGA API.
 
 
 \subsection{Notational Conventions}
 
   The key words \MUST, \MUSTNOT, \REQUIRED, \SHALL, \SHALLNOT,
   \SHOULD, \SHOULDNOT, \RECOMMENDED, \MAY, and \OPTIONAL are to
   be interpreted as described in
   RFC~2119~\cite{rfc-2119}.
 
 
 \subsection{Security Considerations}
 
  As the SAGA API is to be implemented on different types of
  grid (and non-grid) middleware, it does not
  specify a single security model, but rather provides hooks to
  interface to various security models -- see the documentation of the
  |saga::context| class in Section~\ref{ssec:context} for details.
 
  A SAGA implementation is considered secure if and only if it
  fully supports (i.e. implements) the security models of the
  middleware layers it builds upon, and neither provides any
  (intentional or unintentional) means to by-pass these security
  models, nor weakens these security models' policies in any
  way.
 
