 
  This appendix shows a couple of SAGA examples in different
  languages.  As stated in the introduction, these examples are not
  normative -- language bindings are outside the scope of this
  document.  This appendix is rather supposed to illustrate how the
  authors imagine the use of the API in various languages.
  
  We hope that the examples illustrate that the API stays SIMPLE in
  various language incarnations, as was the major design intent for
  the \B{\_S\_}AGA API.
 
 \begin{mycode}
 
  Example 1 (C++): Object State:
  ==============================
  
    // This example illustrates the expected life 
    // times of object states.  State is shared in 
    // these cases, as only shallow copies occur.
 
    int main (void)
    {
      { // task scope
        saga::task t;
  
        { // file scope
          saga::file f;
  
          { // session scope
            saga::session s;
  
            { // context scope
              saga::context c (saga::context::UserPass);
  
              s.add_context (c);
              f (s, saga::url ("file:////tmp/data.bin"));
              t = f.copy <saga::task::Task> 
                    (saga::url ("file:////tmp/data.bak"));
  
            } // leave context scope
              // session keeps context state
  
          } // leave session scope
            // file keeps session state
  
        } // file scope
          // task keeps file state
  
        t.run ();
        // task runs, and uses state of file, session, 
        // and context.
        t.wait ();
  
      } // task scope
        // task    releases file state
        // file    releases session state
        // session releases context state
  
      return (0);
    }
  
  
  +-------------------------------------------------------------+
  
  Example 2: Files:
  =================
  
    open a file. if its size is > 10, then read the first 10
    bytes into a string, print it, end return it.
  
    --------------------------------------------------------------
    Example 2a: C++
    --------------------------------------------------------------
    // c++ example
    void head (const saga::url url)
    {
      try {
        // get type and other info
        saga::file f (url);
 
        off_t size = f.get_size ();
 
        if ( size > 10 )
        {
          char   buf[11];
 
          ssize_t len_out = f.read (saga::buffer (buf));
 
          if ( 10 == len_out )
          {
            std::cout << "head: "
                      << buffer.get_data ()
                      << std::endl;
          }
        }
      }
 
      catch ( const saga::exception & e )
      {
        std::cerr << "Oops! SAGA error: " 
                  << e.get_message () 
                  << std::endl;
      }
 
      return;
    }
    --------------------------------------------------------------
    --------------------------------------------------------------
    Example 2b: C
    -------------
      void head (const SAGA_URL url)
      {
        SAGA_File my_file = SAGA_File_create (url);
  
        if ( NULL == my_file )
        {
          fprintf (stderr, "Could not create SAGA_File "
                           "for %s: %s\n",
                   SAGA_URL_get_url (url), 
                   SAGA_Session_get_error (theSession));
          return (NULL);
        }
  
        off_t size = SAGA_File_get_size (my_file);
  
        if ( size < 0 )
        {
          fprintf (stderr, "Could not determine file size "
                           "for %s: %s\n",
                   SAGA_URL_get_url (url), 
                   SAGA_Session_get_error (theSession));
          return (NULL);
        }
        else if ( size >= 10 )
        {
          SAGA_buffer b = SAGA_Buffer_create ();
          size_t buflen;
  
          ssize_t ret = SAGA_File_read (my_file, b, 10);
  
          if ( ret < 0 )
          {
            fprintf (stderr, "Could not read file %s: %s\n",
                     SAGA_URL_get_url (url), 
                     SAGA_Session_get_error (theSession));
          }
          else if ( ret < 10 )
          {
            fprintf (stderr, "head: short read: %d\n", ret);
          }
          else
          {
            printf ("head: '%s'\n", SAGA_Buffer_get_data (b));
          }
        }
        else
        {
          fprintf (stderr, "head: file %s is too short: %d\n", 
                   file, size);
        }
 
        return;
      }
  
    --------------------------------------------------------------
    Example 2c: Java
    ----------------
  
    import org.ogf.saga.URI;
    import org.ogf.saga.buffer.Buffer;
    import org.ogf.saga.buffer.BufferFactory;
    import org.ogf.saga.file.File;
    import org.ogf.saga.file.FileFactory;
    import org.ogf.saga.namespace.Flags;
    import org.ogf.saga.session.Session;
    
    public class Example {
      // open a file. if its size is >= 10, then read the first
      // 10 bytes into a string, print it, end return it.
      public String head(Session session, URI uri)
      {
        try
        {
          File f = FileFactory.createFile(session, uri, Flags.READ);
          long size = f.getSize();
 
          if (10 <= size) {
            Buffer    buffer = BufferFactory.createBuffer(10);
            int       res    = f.read(10, buffer);
 
            if (10 == res) {
              System.out.println("head: " + buffer);
            } else {
              System.err.println("head: read is short! " + res);
            }
            return new String(buffer.getData());
          } else {
            System.err.println("file is too small: " + size);
          }
        } catch (Exception e) {
          // catch any possible error - see elsewhere for better
          // examples of error handling in SAGA
          System.err.println ("Oops! " + e);
        }
 
        return null;
      }
    }
    --------------------------------------------------------------
    Example 2d: Perl ('normal' error handling)
    ------------------------------------------
  
      sub head ($)
      {
        my $url     = shift;
        my $my_file = new saga::file (url)
                 or die ("can't create file for $url: $!\n");
  
        my $size    = my_file->get_size ();
  
        if ( $size > 10 )
        {
          my $buffer = new saga::buffer (10)l
          my $ret    = my_file->read ($buffer)
                 or die ("can't read from file $url: $!\n");
  
          if ( $ret == 10 )
          {
            print "head: ", $buffer->get_data (), "\n";
          }
          else
          {
            printf STDERR "head: short read: %d\n" ($buffer);
          }
        }
        else
        {
          print STDERR "file $url is too short: $size\n";
        }
  
        return;
      }
  
    --------------------------------------------------------------
    Example 2e: Perl (exceptions)
    -----------------------------
      sub head ($)
      {
        my $url     = shift;
  
        eval 
        {
          my $my_file = new saga::file (url);
          my $size    = my_file->get_size ();
  
          if ( $size > 10 )
          {
            my $buffer = new saga::buffer (10)l
            my $ret    = my_file->read ($buffer);
  
            if ( $ret == 10 )
            {
              print "head: ", $buffer->get_data (), "\n";
            }
            else
            {
              printf "head: short read: %d \n", length ($buffer);
            }
          }
          else
          {
            print "file $url is too short: $size\n";
          }
        }
  
        if ( $@ =~ /^saga/i )
        {
          print "catched saga error: $@\n" if $@;
        }
  
        return;
      }
  
    --------------------------------------------------------------
    Example 2f: Fortran 90
    ----------------------
  
    C Fortran 90 example
       SUBROUTINE HEAD(session, url, buffer)
 
       INTEGER      :: session, url, file, size, buflen
       CHARACTER*10 :: buffer
       
       CALL SAGA_FILE_CREATE(session, url, file)
       CALL SAGA_FILE_GET_SIZE(file, size)
       
       IF size .GT. 10 THEN
 
         CALL SAGA_FILE_READ(file, 10, buffer, buflen)
         
         IF buflen .EQ. 10 THEN
           WRITE(5, *) 'head: ', buffer
         ELSE
           WRITE(5, *) 'head: short read: ', buflen
         ENDIF
       ELSE
         WRITE(5, *) 'file is too short'
       ENDIF
 
       END
  
    --------------------------------------------------------------
    Example 2g: Python
    ------------------
    # Python example
    def head (session,url):
 
      try:
        my_file = saga.file(session,url)
        size = my_file.get_size()
  
        if (size > 10):
          my_buffer = saga.buffer (10)
          ret = my_file.read (my_buffer)
          if (ret == 10):
            print "head: ", my_buffer.get_data ()
          else
            print "head: short read: ", ret
        else
          print "head: file is too short: ", size
 
      # catch any possible error - see elsewhere for better
      # examples of error handling in SAGA
      except saga.Exception, e:
        print "Oops! SAGA error: ", e.get_message ()
 \end{mycode}
 
 
