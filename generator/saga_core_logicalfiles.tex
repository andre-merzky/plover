 
   This section of the SAGA API describes the interaction with
   replica systems.  Numerous SAGA use cases required replica
   management functionality in the API -- however, only a small
   number of operation have been requested.  The methods
   described here are hence limited to the creation and
   maintainance of logical files, replicas, and to search on
   logical file meta data.
 
   The |saga::logical_file| class implements the
   |saga::attributes| interface.  It is important to realize that
   this is intended to reflect the ability of replica
   systems to associate meta data with logical files.  The SAGA
   attribute model (string based key/value pairs) can, with all
   probability, only give a crude representation of
   meta data models used in real world replica systems --
   however, the definition of a more abstract and comprehensive
   data model for replica meta data was felt to be outside the
   scope of a SAGA API definition.  Implementations are expected
   to map the native data model to key/value pairs as well as
   possible, and MUST document that mapping process (and in
   particular the supported keys) carefully.
 
   Please note that the interactions with logical files as
   opaque entities (as entries in logical file name spaces) are
   covered by the |namespace| package.  The interfaces presented
   here supplement the |namespace| package with operations for
   operating on entries in replica catalogues.
 
   It is up to the used backend to ensure that multiple
   replica locations registered on a logical file are indeed
   identical copies -- the SAGA API does not imply any specific
   consistency model.  The SAGA implementation MUST document the
   consistency model used.
 
   \subsubsection{Definitions}
 
    \paragraph{Logical File:} A \I{logical file} represents
    merely an entry in a name space which has (a) an associated
    set of registered (physical) replicas of that file, and (b)
    an associated set of meta data describing that logical file.
    Both sets can be empty.  To access the \I{content}
    of a logical file, a \T{saga::file} needs to be created with
    one of the registered replica locations.
 
    \paragraph{Replica:} A \I{replica} (or \I{physical file}) is
    a file which is registered on a logical file.  In general,
    all replicas registered on the same logical file are
    identical.  Often, one of these replicas is deemed to be a
    master copy (often it is the first replica
    registered, and/or the only one which can be changed) --
    that distinction is, however, not visible in the SAGA API.
 
    \paragraph{Logical Directory:} A \I{logical directory}
    represents a directory entry in the name space of logical
    files.  Several replica system implementations have the
    notion of \I{container}s, which, for our purposes, represent
    directories which can have, just as logical files,
    associated sets of meta data.  In the presented API, logical
    directories and containers are the same.
 
    % Note that the 'Truncate' flag on opening logical files is
    % interpreted as to truncate the set of registered replicas on
    % that logical file -- the associated meta data set is \I{not}
    % truncated.
 
    Note that the |Truncate|, |Append| and |Binary| flags have no
    meaning on logical files.  The respective enum values for these
    flags for |saga::file|s have been reserved though, for (a) future
    use, and (b) consistency with the |saga::file| flag values.
 
    The |find()| method of the |saga::logical_directory| class
    represents a combination of (a) the |find()| method from the
    |saga::ns_directory| class, and (b) the |find_attributes()|
    method from the |saga::attributes| interface.  The method
    accepts patterns for meta data matches
    (\T{attr\_pattern}) and a single pattern for file
    name matches (|name_pattern|), and returns a list of logical
    file names which match
    all \T{attr\_pattern} and the \T{name\_pattern} (\T{AND}
    semantics).  The \T{attr\_pattern} are formatted as
    defined for |find_attribute()| of the |saga::attributes|
    interface.  The |name_pattern| are formatted as defined for
    the |find()| method of the |saga::ns_directory| class.  In
    general, the allowed patterns are the same as defined as
    wildcards in the description of the SAGA
    |namespace| package.
 
 \subsubsection{Specification}
 
 \begin{myspec}
  package saga.logical_file
  {
    // some flag description comment
    // going over several lines
    enum flags
    {
      None            =    0, // same as in namespace::flags
      Overwrite       =    1, // same as in namespace::flags
      Recursive       =    2, // same as in namespace::flags
      Dereference     =    4, // same as in namespace::flags
      Create          =    8, // same as in namespace::flags
      Exclusive       =   16, // same as in namespace::flags
      Lock            =   32, // same as in namespace::flags
      CreateParents   =   64, // same as in namespace::flags
      //                 128,    reserved for Truncate
      //                 256,    reserved for Append
!     Read            =  512, // same as in namespace::flags 
!     Write           = 1024, // same as in namespace::flags 
!     ReadWrite       = 1536, // same as in namespace::flags 
      //                2048     reserved for Binary
    }
 
 
    class logical_file : extends        saga::ns_entry
                         implements     saga::attributes
                      // from ns_entry  saga::object
                      // from ns_entry  saga::async
                      // from object    saga::error_handler
    {
      CONSTRUCTOR     (in  session             s,
                       in  saga::url           name,
                       in  int                 flags = Read,
                       out logical_file        obj);
      DESTRUCTOR      (in  logical_file        obj);
 
 
      // manage the set of associated replicas
      add_location    (in  saga::url           name);
      remove_location (in  saga::url           name);
      update_location (in  saga::url           name_old,
                       in  saga::url           name_new);
      list_locations  (out array<saga::url>    names);
 
      // create a new physical replica
      replicate       (in  saga::url           name, 
                       in  int                 flags = None);
 
      // Attributes (extensible):
      // 
      // no attributes pre-defined
    }
 
 
    class logical_directory : extends            saga::ns_directory
                              implements         saga::attributes
                           // from ns_directory  saga::ns_entry
                           // from ns_entry      saga::object
                           // from ns_entry      saga::async
                           // from object        saga::error_handler
    {
 
      CONSTRUCTOR     (in  session             s,
                       in  saga::url           name,
                       in  int                 flags = Read,
                       out logical_directory   obj);
      DESTRUCTOR      (in  logical_directory   obj);
 
 
      // inspection methods
      is_file         (in  saga::url           name,
                       out boolean             test);
 
      // open methods
      open_dir        (in  saga::url           name,
                       in  int                 flags = Read,
                       out logical_directory   dir);
 
      open            (in  saga::url           name,
                       in  int                 flags = Read,
                       out logical_file        file);
 
      // find logical files based on name and meta data
      find            (in  string              name_pattern,
                       in  array<string>       attr_pattern,
                       in  int                 flags = Recursive,
                       out array<saga::url>    names   );
    }
  }
 \end{myspec}
 
 
 \subsubsection{Specification Details}
 
  \subsubsection*{Enum \T{flags}}
 
   The \T{flags} enum is inherited from the \T{namespace}
   package.  \XRep{A number of replica specific flags are added
   to it.  All added flags are used for the opening of
   \T{logical\_\-file} and \T{logical\_\-directory} instances, and
   are not applicable to the operations inherited from the
   \T{namespace} package.}{No additional flags are added.}
  
  % |Read|\\[0.3mm]
  % \begin{tabular}{cp{110mm}}
  %   ~~ & The logical file or directory is opened for 
  %        reading -- that does not imply the ability to change 
  %        the logical file or directory.
  % \end{tabular}
  %
  % |Write|\\[0.3mm]
  % \begin{tabular}{cp{110mm}}
  %   ~~ & The logical file or directory is opened for 
  %        writing -- that does not imply the ability to read 
  %        from the logical file or directory.
  % \end{tabular}
  %
  % |ReadWrite|\\[0.3mm]
  % \begin{tabular}{cp{110mm}}
  %   ~~ & The logical file or directory is opened for 
  %        reading and writing.
  % \end{tabular}
 
 
 
  \subsubsection*{Class \T{logical\_file}}
 
    This class provides the means to handle the contents
    of logical files.  These contents consists of strings
    representing locations of physical files (replicas)
    associated with the logical file.  
 
 \begin{myspec}
    - CONSTRUCTOR
      Purpose:  create the object
      Format:   CONSTRUCTOR      (in session    s,
                                  in  saga::url name,
                                  in  int       flags = Read,
                                  out logical_file  obj)
      Inputs:   s:                session to associate with
                                  the object
                name:             location of file
                flags:            mode for opening
      InOuts:   -
      Outputs:  obj:              the newly created object
      PreCond:  -
      PostCond: - the logical_file is opened.
                - 'Owner' of target is the id of the context
                  use to perform the operation, if the
                  logical_file gets created.
      Perms:    Exec  for parent directory.
                Write for parent directory if Create is set.
                Write for name if Write is set.
                Read  for name if Read  is set.
      Throws:   NotImplemented
                IncorrectURL
                BadParameter
                AlreadyExists
                DoesNotExist
                PermissionDenied
                AuthorizationFailed
                AuthenticationFailed
                Timeout
                NoSuccess
      Notes:    - the semantics of the inherited constructors
                  and of the logical_directory::open() method 
                  apply.
                - the default flags are 'Read' (512).
 
 
    - DESTRUCTOR
      Purpose:  destroy the object
      Format:   DESTRUCTOR       (in  logical_file   obj)
      Inputs:   obj:              the object to destroy
      InOuts:   -
      Outputs:  -
      PreCond:  -
      PostCond: - the logical_file is closed.
      Perms:    -
      Throws:   -
      Notes:    - the semantics of the inherited destructors
                  apply.
 
 
      manage the set of associated replicas:
      --------------------------------------
 
    - add_location
      Purpose:  add a replica location to the replica set
      Format:   add_location     (in  saga::url name);
      Inputs:   name:             location to add to set
      InOuts:   -
      Outputs:  -
      PreCond:  -
      PostCond: - name is in the list of replica locations for
                  the logical file.
      Perms:    Write
      Throws:   NotImplemented
                IncorrectURL
                BadParameter
                IncorrectState
                PermissionDenied
                AuthorizationFailed
                AuthenticationFailed
                Timeout
                NoSuccess
      Notes:    - this methods adds a given replica location 
                  (name) to the set of locations associated with 
                  the logical file.
                - the implementation MAY choose to interpret the
                  replica locations associated with the logical 
                  file.  It MAY return an 'IncorrectURL' error 
                  indicating an invalid location if it is unable 
                  or unwilling to handle that specific locations
                  scheme.  The implementation documentation MUST 
                  specify how valid replica locations are formed.
                - if 'name' can be parsed as URL, but contains 
                  an invalid entry name, a 'BadParameter'
                  exception is thrown.
                - if the replica is already in the set, this
                  method does nothing, and in particular MUST
                  NOT raise an 'AlreadyExists' exception
                - if the logical file was opened ReadOnly, a
                  'PermissionDenied' exception is thrown.
 
 
    - remove_location
      Purpose:  remove a replica location from the replica set
      Format:   remove_location  (in  saga::url name);
      Inputs:   name:             replica to remove from set
      InOuts:   -
      Outputs:  -
      PreCond:  -
      PostCond: - name is not anymore in list of replica 
                  locations for the logical file.
      Perms:    Write
      Throws:   NotImplemented
                IncorrectURL
                BadParameter
                DoesNotExist
                IncorrectState
                PermissionDenied
                AuthorizationFailed
                AuthenticationFailed
                Timeout
                NoSuccess
      Notes:    - this method removes a given replica location 
                  from the set of replicas associated with the 
                  logical file.
                - the implementation MAY choose to interpret the
                  replica locations associated with the logical 
                  file.  It MAY return an 'IncorrectURL' error 
                  indicating an invalid location if it is unable 
                  or unwilling to handle that specific locations
                  scheme.  The implementation documentation MUST 
                  specify how valid replica locations are formed.
                - if 'name' can be parsed as URL, but contains 
                  an invalid entry name, a 'BadParameter'
                  exception is thrown.
                - if the location is not in the set of
                  replicas, a 'DoesNotExist' exception is 
                  thrown.
                - if the set of locations is empty after this
                  operation, the logical file object is still 
                  a valid object (see replicate() method
                  description).
                - if the logical file was opened ReadOnly, a
                  'PermissionDenied' exception is thrown.
 
 
    - update_location
      Purpose:  change a replica location in replica set
      Format:   update_location  (in saga::url name_old,
                                  in saga::url name_new);
      Inputs:   name_old          replica to be updated
                name_new          update of replica
      InOuts:   -
      Outputs:  -
      PreCond:  -
      PostCond: - name_old is not anymore in list of replica 
                  locations for the logical file.
                - name_new is in the list of replica locations
                  for the logical file.
      Perms:    Read 
                Write
      Throws:   NotImplemented
                IncorrectURL
                BadParameter
                AlreadyExists
                DoesNotExist
                IncorrectState
                PermissionDenied
                AuthorizationFailed
                AuthenticationFailed
                Timeout
                NoSuccess
      Notes:    - this method removes a given replica location 
                  from the set of locations associated with the 
                  logical file, and adds a new location.
                - the implementation MAY choose to interpret the
                  replica locations associated with the logical 
                  file.  It MAY return an 'IncorrectURL' error 
                  indicating an invalid location if it is unable 
                  or unwilling to handle that specific locations
                  scheme.  The implementation documentation MUST 
                  specify how valid replica locations are formed.
                - if 'name' can be parsed as URL, but contains 
                  an invalid entry name, a 'BadParameter'
                  exception is thrown.
                - if the old replica location is not in the 
                  set of locations, a 'DoesNotExist' exception 
                  is thrown.
                - if the new replica location is already in the 
                  set of locations, an 'AlreadyExists' exception 
                  is thrown.
                - if the logical file was opened ReadOnly, an
                  'PermissionDenied' exception is thrown.
                - if the logical file was opened WriteOnly, an
                  'PermissionDenied' exception is thrown.
 
 
    - list_locations
      Purpose:  list the locations in the location set
      Format:   list_locations   (out array<saga::url> names);
      Inputs:   -
      InOuts:   -
      Outputs:  names:            array of locations in set
      PreCond:  -
      PostCond: - 
      Perms:    Read
      Throws:   NotImplemented
                IncorrectState
                PermissionDenied
                AuthorizationFailed
                AuthenticationFailed
                Timeout
                NoSuccess
      Notes:    - this method returns an array of urls
                  containing the complete set of locations
                  associated with the logical file.
                - an empty array returned is not an error - 
                  the logical file object is still a valid 
                  object (see replicate() method description).
                - if the logical file was opened WriteOnly, an
                  'PermissionDenied' exception is thrown.
 
 
    - replicate 
      Purpose:  replicate a file from any of the known
                replica locations to a new location, and, on 
                success, add the new replica location to the 
                set of associated replicas 
      Format:   replicate        (in  saga::url name, 
                                  in  int       flags = None);
      Inputs:   name:             location to replicate to
                flags:            flags defining the operation
                                  modus
      InOuts:   -
      Outputs:  -
      PreCond:  -
      PostCond: - an identical copy of one of the available
                  replicas exists at name.
                - name is in the list of replica locations
                  for the logical file.
      Perms:    Read 
                Write
      Throws:   NotImplemented
                IncorrectURL
                BadParameter
                AlreadyExists
                DoesNotExist
                IncorrectState
                PermissionDenied
                AuthorizationFailed
                AuthenticationFailed
                Timeout
                NoSuccess
      Notes:    - the method implies a two step operation:
                  1) create a new and complete replica at the
                     given location, which then represents
                     a new replica location.
                  2) perform an add_location() for the new
                     replica location.
                - all notes to the saga::ns_entry::copy() and
                  saga::logical_file::add_location methods
                  apply.
                - the method is not required to be atomic, but:
                  the implementation MUST be either
                  successful in both steps, or throw an
                  exception indicating if both methods failed, 
                  or if one of the methods succeeded.
                - a replicate call on an instance with empty
                  location set raises an 'IncorrectState'
                  exception, with an descriptive error message.  
                - the default flags are 'None' (0).  The
                  interpretation of flags is as described for 
                  the ns_entry::copy() method.  
                - The 'Recursive' flag is not allowed, and 
                  causes a 'BadParameter' exception.
                - if the logical file was opened ReadOnly, an
                  'PermissionDenied' exception is thrown.
                - if the logical file was opened WriteOnly, an
                  'PermissionDenied' exception is thrown.
 \end{myspec}
 
 
  \subsubsection*{Class \T{logical\_directory}}
 
    This class represents a container for logical files in a
    logical file name space.  It allows traversal of the
    catalog's name space, and the manipulation and creation
    (open) of logical files in that name space.
    
 \begin{myspec}
    Constructor / Destructor:
    -------------------------
 
    - CONSTRUCTOR
      Purpose:  create the object
      Format:   CONSTRUCTOR      (in  session       s,
                                  in  saga::url     name,
                                  in  int           flags = Read,
                                  out logical_directory
                                                    obj)
      Inputs:   s:                session to associate with
                                  the object
                name:             location of directory
                flags:            mode for opening
      InOuts:   -
      Outputs:  obj:              the newly created object
      PreCond:  -
      PostCond: - the logical_directory is opened.
                - 'Owner' of target is the id of the context
                  use to perform the operation, if the
                  logical_directory gets created.
      Perms:    Exec  for parent directory.
                Write for parent directory if Create is set.
                Write for name if Write is set.
                Read  for name if Read  is set.
      Throws:   NotImplemented
                IncorrectURL
                BadParameter
                AlreadyExists
                DoesNotExist
                PermissionDenied
                AuthorizationFailed
                AuthenticationFailed
                Timeout
                NoSuccess
      Notes:    - the semantics of the inherited constructors
                  and of the logical_directory::open_dir()
                  method apply.
                - the default flags are 'Read' (512).
 
 
    - DESTRUCTOR
      Purpose:  destroy the object
      Format:   DESTRUCTOR         (in  logical_directory obj)
      Inputs:   obj:                the object to destroy
      InOuts:   -
      Outputs:  -
      PreCond:  -
      PostCond: - the logical_directory is closed.
      Perms:    -
      Throws:   -
      Notes:    - the semantics of the inherited destructors
                  apply.
 
 
    - is_file
      Alias:    for is_entry of saga::ns_directory
 
 
    - open_dir
      Purpose:  creates a new logical_directory instance
      Format:   open_dir         (in  saga::url name,
                                  in  int       flags = Read,
                                  out logical_directory dir);
      Inputs:   name:             name of directory to open
                flags:            flags defining operation
                                  modus
      InOuts:   -
      Outputs:  dir:              opened directory instance
      PreCond:  -
      PostCond: - the session of the returned instance is that of
                  the calling instance.
                - 'Owner' of name is the id of the context
                  used to perform the operation if name gets
                  created.
      Perms:    Exec  for name's parent directory.
                Write for name's parent directory if Create is set.
                Write for name if Write is set.
                Read  for name if Read  is set.
      Throws:   NotImplemented
                IncorrectURL
                BadParameter
                AlreadyExists
                DoesNotExist
                IncorrectState
                PermissionDenied
                AuthorizationFailed
                AuthenticationFailed
                Timeout
                NoSuccess
      Notes:    - all notes from the ns_directory::open_dir()
                  method apply.
                - default flags are 'Read' (512).
 
 
    - open
      Purpose:  creates a new logical_file instance
      Format:   open             (in  saga::url    name,
                                  in  int          flags = Read,
                                  out logical_file file);
      Inputs:   name:             file to be opened
                flags:            flags defining operation
                                  modus
      InOuts:   -
      Outputs:  file:             opened file instance
      PreCond:  -
      PostCond: - the session of the returned instance is that of
                  the calling instance.
                - 'Owner' of name is the id of the context
                  used to perform the operation if name gets
                  created.
      Perms:    Exec  for name's parent directory.
                Write for name's parent directory if Create is set.
                Write for name if Write is set.
                Read  for name if Read  is set.
      Throws:   NotImplemented
                IncorrectURL
                BadParameter
                AlreadyExists
                DoesNotExist
                IncorrectState
                PermissionDenied
                AuthorizationFailed
                AuthenticationFailed
                Timeout
                NoSuccess
      Notes:    - all notes from the ns_directory::open() method
                  apply.
                - the flag set 'Read | Write' is equivalent to
                  the flag 'ReadWrite'.
                - default flags are 'Read' (512).
 
 
    - find
      Purpose:  find entries in the current directory and below, 
                with matching names and matching meta data
      Format:   find             (in  string           name_pattern,
                                  in  array<string>    attr_pattern,
                                  in  int              flags = Recursive,
                                  out array<saga::url> names);
      Inputs:   name_pattern:     pattern for names of
                                  entries to be found
                attr_pattern:     pattern for meta data
                                  key/values of entries to be 
                                  found
                flags:            flags defining the operation
                                  modus
      InOuts:   -
      Outputs:  names:            array of names matching both
                                  pattern
      PreCond:  -
      PostCond: -
      Perms:    Read  for cwd.
                Query for entries specified by name_pattern.
                Exec  for parent directories of these entries.
                Query for parent directories of these entries.
                Read  for directories specified by name_pattern.
                Exec  for directories specified by name_pattern.
                Exec  for parent directories of these directories.
                Query for parent directories of these directories.
      Throws:   NotImplemented
                BadParameter
                IncorrectState
                PermissionDenied
                AuthorizationFailed
                AuthenticationFailed
                Timeout
                NoSuccess
      Notes:    - the description of find() in the Introduction
                  to this section applies.
                - the semantics for both the find_attributes()
                  method in the saga::attributes interface and 
                  for the find() method in the 
                  saga::ns_directory class apply.  On 
                  conflicts, the find() semantic supersedes 
                  the find_attributes() semantic.  Only entries
                  matching all attribute patterns and the name 
                  space pattern are returned.
                - the default flags are 'Recursive' (2).
 \end{myspec}
 
 
 \subsubsection{Examples}
 
 \begin{mycode}
  // c++ example
  int main ()
  {
    saga::logical_file lf ("lfn://remote.catalog.net/tmp/file1");
 
    lf.replicate ("gsiftp://localhost//tmp/file.rep");
    saga::file f ("gsiftp://localhost//tmp/file.rep");
 
    std::cout << "size of local replica: "
              << f.get_size ()
              << std::endl;
 
    return (0);
  }
 \end{mycode}
 
 
