 
  There are various places in the SAGA API where attributes need to be
  associated with objects, for instance for job descriptions and
  metrics.  The |attributes| interface provides a common interface for
  storing and retrieving attributes.
 
  Objects implementing this interface maintain a set of
  attributes.  These attributes can be considered as a set of
  key-value pairs attached to the object.  The key-value pairs
  are string based for now, but might cover other value types in
  later versions of the SAGA API specification.
 
  The interface name |attributes| is somewhat misleading: it
  seems to imply that an object implementing this interface
  |IS-A| set of attributes.  What we actually mean is that an
  object implementing this interface |HAS| attributes.  
  In the absence of a better name,
  we left it |attributes|, but implementors
  and users should be aware of the actual meaning  (the
  proper interface name would be 'attributable', which sounds
  awkward).

  \XAdd[7]{Several functional classes will need to implement attributes
  as remote functionality, and such an implementation is by definition
  middleware dependent, and thus not always implementable.  That is why
  the \T{NotImplemented} exception is listed for all attribute
  interface methods.  However, SAGA \LF classes which MUST be
  implemented by SAGA compliant implementations (see intro to
  Section~\ref{sec:nonfunc}, on page~\pageref{sec:nonfunc}), and which
  do implement the \T{attributes} interface, MUST NOT throw the
  \T{NotImplemented} exception, ever.}
 
  The SAGA specification defines attributes which MUST be supported by
  the various SAGA objects, and also defines their default values, and
  those which CAN be supported.  An implementation MUST motivate and
  document if a specified attribute is not supported.  
 
 
 \subsubsection{Specification}
 
 \begin{myspec}
  package saga.core
  {
    interface attributes
    {
      // setter / getters
      set_attribute           (in  string          key,
                               in  string          value);
      get_attribute           (in  string          key,
                               out string          value);
      set_vector_attribute    (in  string          key,
                               in  array<string>   values);
      get_vector_attribute    (in  string          key,
                               out array<string>   values);
      remove_attribute        (in  string          key);
 
      // inspection methods
      list_attributes         (out array<string>   keys);
      find_attributes         (in  array<string>   pattern,
                               out array<string>   keys);
      attribute_exists        (in  string          key,
                               out bool            test);
      attribute_is_readonly   (in  string          key,
                               out bool            test);
      attribute_is_writable   (in  string          key,
                               out bool            test);
      attribute_is_removable  (in  string          key,
                               out bool            test);
      attribute_is_vector     (in  string          key,
                               out bool            test);
    }
  }
 \end{myspec}
 
 \subsubsection{Specification Details}
 
  The |attributes| interface in SAGA provides a uniform paradigm
  to set and query parameters and properties of SAGA objects.
  Although the |attributes| interface is generic by design (i.e.
  it allows arbitrary keys and values to be used), its use in
  SAGA is mostly limited to a finite and well defined set of
  keys.
 
  In several languages, attributes can much more elegantly
  be expressed by native means - e.g. by using hash
  tables in Perl.  Bindings for such languages MAY allow to use
  a native interface \I{additionally} to the one described here.
 
  Several SAGA objects have very frequently used attributes.  To
  simplify usage of these objects, setter and getter methods MAY
  be defined by the various language bindings, again
  \I{additionally} to the interface described below.  For
  attributes of native non-string types, these setter/getters
  MAY be typed.
 
  For example, additionally to:
 
  \up
  \begin{myio}
    stream.set_attribute ("BufferSize", "1024");
  \end{myio}
 
  \up
  a language binding might allow:
 
  \up
  \begin{myio}
    stream.set_buffer_size (1024);  // int type
  \end{myio}
 
  \up
  Further, in order to limit semantic and syntactic ambiguities
  (e.g., due to spelling deviations), language bindings MUST
  define known attribute keys as constants, such as (in C):
 
  \up
  \begin{myio}
    #define SAGA_BUFFERSIZE "BufferSize"
    ...
    stream.set_attribute (SAGA_BUFFERSIZE, "1024");
  \end{myio}
  
  \up
  The distinction between scalar and vector attributes is
  supposed to help those
  languages where this aspect of attributes cannot
  be handled transparently, e.g. by overloading.  Bindings for languages
  such as Python, Perl and C++ CAN hide this distinction as long
  as both access types are supported.
 
  Elements of vector attributes are ordered. This order MUST be 
  preserved by the SAGA implementation.
  Comparison also relies on ordering (i.e. |'one two'| does
  not equal |'two one'|).  For example, this order is
  significant for the |saga::job_description| attribute
  |'Arguments'|, which represents command line arguments for a
  job.
 
  Attributes are expressed as string values. They have,
  however, a type, which defines the formatting of that
  string.  The allowed types are |String|, |Int|, |Enum|,
  |Float|, |Bool|, and |Time| (the same as metric value types).
  Additionally, attributes are qualified as either |Scalar| or
  |Vector|.  The default is |Scalar|.
 
  Values of |String| type attributes are expressed as-is.
 
  Values of |Int| (i.e. Integer) type attributes are expressed
  as they would in result of a printf of the format
  \T{'\%lld'}, as defined by POSIX.
 
  Values of |Enum| type attributes are expressed as strings, and
  have the literal value of the respective enums as defined in
  this document.  For example, the initial task states would
  have the values |'New'|, |'Running'|, |'Done'|, etc.
 
  Values of |Float| (i.e. floating point) type attributes
  are expressed as they would in result of a printf of the
  format \T{'\%Lf'}, as defined by POSIX.
 
  Values of |Bool| type attributes MUST be expressed as
  'True' or 'False'.
 
  Values of |Time| type attributes MUST be expressed as they
  would in result of a call to |ctime()|, as defined by POSIX.
  Applications can also specify these attribute values as
  seconds since epoch (this formats the string as
  an |Int| type), but all time attributes set by the
  implementation MUST be in |ctime()| format.  Applications
  should be aware of the |strptime()| and |strftime()| methods
  defined in POSIX, which assist time conversions.
 
 
 \subsubsection{Attribute Definitions in the SAGA specification}
 
  The SAGA specification defines a number of attributes which
  MUST or CAN be supported, for various SAGA objects.  An
  example of such a definition is (from the Metric
  object):
 
  \begin{myspectxt}
        class metric ...
        {
          ...
 
          // Attributes:
          //   name:  Name
          //   desc:  name of metric
          //   mode:  ReadOnly
          //   type:  String
          //   value: -
          //   notes: naming conventions as described below apply
          //
          //   ...
        }
  \end{myspectxt}
 
    These specifications are NORMATIVE, even if described as
    comments in the SIDL specification!  The specified
    attributes MUST be supported by an implementation, unless
    noted otherwise, as:
 
    \shift |//  mode:  ReadOnly, optional|\\
    \shift |//  mode:  ReadWrite, optional|
 
    If an attribute MUST be supported, but the SAGA
    implementation cannot support that attribute, any set/get on
    that attribute MUST throw a |NotImplemented| exception, and
    the error message MUST state \T{"Attribute $<$name$>$ not
    available in this implementation"}.
 
    If the default value is denoted as '--', then the
    attribute is, by default, not set at all.
 
    Attribute support can 'appear' and 'go away' during the
    lifetime of an object (e.g., as late binding implementations
    switch the backend).  Any set on an attribute which
    got removed ('dead attribute') MUST throw a |DoesNotExist|
    exception.  However, dead attributes MUST stay available for
    read access.  The SAGA implementation MUST NOT change such an
    attribute's value, as long as it is not available.  Allowed
    values for mode are |ReadOnly| and |ReadWrite|.
 
    It is not allowed to add attributes other than
    those specified in this document, unless explicitly
    allowed, as:
 
    \shift |//  Attributes (extensible):|
 
    The |find_attributes()| method accepts a list of
    patterns, and returns a list of keys for those attributes
    which match any one of the specified patterns (|OR|
    semantics).  The patterns  describe both
    attribute keys and values, and are formatted as:
    
    \shift |<key-pattern>=<value-pattern>|
 
    Both the \T{key-pattern} and the \T{value-pattern} can
    contain wildcards as defined in the description of
    the SAGA \T{namespace} package.  If a
    \T{key-pattern} contains an \T{'='} character, that character
    must be escaped by a backslash, as must any backslash
    character itself.  The \T{value-pattern} can be empty, and the method
    will then return all attribute keys which match the
    \T{key-pattern}.  The equal sign \T{'='} can then be omitted
    from the pattern.
 
 
 \subsubsection*{Interface \T{attributes}}
 
 \begin{myspec}
    - set_attribute
      Purpose:  set an attribute to a value
      Format:   set_attribute        (in string key,
                                      in string value);
      Inputs:   key:                  attribute key
                value:                value to set the
                                      attribute to
      InOuts:   -
      Outputs:  -
      PreCond:  -
      PostCond: -
      Perms:    Write
      Throws:   NotImplemented
                BadParameter
                DoesNotExist
                IncorrectState
                PermissionDenied
                AuthorizationFailed
                AuthenticationFailed
                Timeout
                NoSuccess
      Notes:    - an empty string means to set an empty value
                  (the attribute is not removed).
                - the attribute is created, if it does not exist
                - a 'PermissionDenied' exception is thrown if the
                  attribute to be changed is ReadOnly.
                - only some SAGA objects allow to create new
                  attributes - others allow only access to
                  predefined attributes.  If a non-existing
                  attribute is queried on such objects, a
                  'DoesNotExist' exception is raised
                - changes of attributes may reflect changes of
                  endpoint entity properties.  As such,
                  authorization and/or authentication may fail
                  for settings such attributes, for some
                  backends.  In that case, the respective
                  'AuthenticationFailed', 'AuthorizationFailed',
                  and 'PermissionDenied' exceptions are thrown.
                  For example, an implementation may forbid to
                  change the saga::stream 'BufSize' attribute.
                - if an attribute is not well formatted, or
                  outside of some allowed range, a 'BadParameter'
                  exception with a descriptive error message is
                  thrown.
                - if the operation is attempted on a vector
                  attribute, an 'IncorrectState' exception is
                  thrown.
                - setting of attributes may time out, or may fail
                  for other reasons - which causes a 'Timeout' or
                  'NoSuccess' exception, respectively.
 
 
    - get_attribute
      Purpose:  get an attribute value
      Format:   get_attribute        (in  string key,
                                      out string value);
      Inputs:   key:                  attribute key
      InOuts:   -
      Outputs:  value:                value of the attribute
      PreCond:  -
      PostCond: -
      Perms:    Query
      Throws:   NotImplemented
                DoesNotExist
                IncorrectState
                PermissionDenied
                AuthorizationFailed
                AuthenticationFailed
                Timeout
                NoSuccess
      Notes:    - queries of attributes may imply queries of
                  endpoint entity properties.  As such,
                  authorization and/or authentication may fail
                  for querying such attributes, for some
                  backends.  In that case, the respective
                  'AuthenticationFailed', 'AuthorizationFailed',
                  and 'PermissionDenied' exceptions are thrown.
                  For example, an implementation may forbid to
                  read the saga::stream 'BufSize' attribute.
                - reading an attribute value for an attribute
                  which is not in the current set of attributes
                  causes a 'DoesNotExist' exception.
                - if the operation is attempted on a vector
                  attribute, an 'IncorrectState' exception is
                  thrown.
                - getting attribute values may time out, or may 
                  fail for other reasons - which causes a 
                  'Timeout' or 'NoSuccess' exception, 
                  respectively.
 
 
    - set_vector_attribute
      Purpose:  set an attribute to an array of values.
      Format:   set_vector_attribute (in string          key,
                                      in array<string>   values);
      Inputs:   key:                  attribute key
                values:               array of attribute values
      InOuts:   -
      Outputs:  -
      PreCond:  -
      PostCond: -
      Perms:    Write
      Throws:   NotImplemented
                BadParameter
                DoesNotExist
                IncorrectState
                PermissionDenied
                AuthorizationFailed
                AuthenticationFailed
                Timeout
                NoSuccess
      Notes:    - the notes to the set_attribute() method apply.
                - if the operation is attempted on a scalar
                  attribute, an 'IncorrectState' exception is
                  thrown.
 
 
    - get_vector_attribute
      Purpose:  get the array of values associated with an
                attribute
      Format:   get_vector_attribute (in string           key,
                                      out array<string>   values);
      Inputs:   key:                  attribute key
      InOuts:   -
      Outputs:  values:               array of values of the
                                      attribute.
      PreCond:  -
      PostCond: -
      Perms:    Query
      Throws:   NotImplemented
                DoesNotExist
                IncorrectState
                PermissionDenied
                AuthorizationFailed
                AuthenticationFailed
                Timeout
                NoSuccess
      Notes:    - the notes to the get_attribute() method apply.
                - if the operation is attempted on a scalar
                  attribute, an 'IncorrectState' exception is
                  thrown.
 
 
    - remove_attribute
      Purpose:  removes an attribute.
      Format:   remove_attribute     (in string key);
      Inputs:   key:                  attribute to be removed
      InOuts:   -
      Outputs:  -
      PreCond:  -
      PostCond: - the attribute is not available anymore.
      Perms:    Write
      Throws:   NotImplemented
                DoesNotExist
                PermissionDenied
                AuthorizationFailed
                AuthenticationFailed
                Timeout
                NoSuccess
      Notes:    - a vector attribute can also be removed with
                  this method
                - only some SAGA objects allow to remove
                  attributes.
                - a ReadOnly attribute cannot be removed - any
                  attempt to do so throws a 'PermissionDenied' 
                  exception.
                - if a non-existing attribute is removed, a
                  'DoesNotExist' exception is raised.
                - exceptions have the same semantics as defined
                  for the set_attribute() method description.
 
 
    - list_attributes
      Purpose:  Get the list of attribute keys.
      Format:   list_attributes      (out array<string>   keys);
      Inputs:   -
      InOuts:   -
      Outputs:  keys:                 existing attribute keys
      PreCond:  -
      PostCond: -
      Perms:    Query
      Throws:   NotImplemented
                PermissionDenied
                AuthorizationFailed
                AuthenticationFailed
                Timeout
                NoSuccess
      Notes:    - exceptions have the same semantics as defined
                  for the get_attribute() method description.
                - if no attributes are defined for the object, 
                  an empty list is returned.
 
 
    - find_attributes
      Purpose:  find matching attributes.
      Format:   find_attributes      (in  array<string>   pattern,
                                      out array<string>   keys);
      Inputs:   pattern:              search patterns
      InOuts:   -
      Outputs:  keys:                 matching attribute keys
      PreCond:  -
      PostCond: -
      Perms:    Query
      Throws:   NotImplemented
                BadParameter
                PermissionDenied
                AuthorizationFailed
                AuthenticationFailed
                Timeout
                NoSuccess
      Notes:    - the pattern must be formatted as described 
                  earlier, otherwise a 'BadParameter' exception 
                  is thrown. 
                - exceptions have the same semantics as defined
                  for the get_attribute() method description.
 
 
    - attribute_exists
      Purpose:  check the attribute's existence.
      Format:   attribute_exists     (in  string key,
                                      out bool   test);
      Inputs:   key:                  attribute key
      InOuts:   -
      Outputs:  test:                 bool indicating success
      PreCond:  -
      PostCond: -
      Perms:    Query
      Throws:   NotImplemented
                PermissionDenied
                AuthorizationFailed
                AuthenticationFailed
                Timeout
                NoSuccess
      Notes:    - This method returns TRUE if the attribute
                  identified by the key exists.
                - exceptions have the same semantics as defined
                  for the get_attribute() method description, 
                  apart from the fact that a DoesNotExist
                  exception is never thrown.
 
 
    - attribute_is_readonly
      Purpose:  check the attribute mode.
      Format:   attribute_is_readonly(in  string key,
                                      out bool   test);
      Inputs:   key:                  attribute key
      InOuts:   -
      Outputs:  test:                 bool indicating success
      PreCond:  -
      PostCond: -
      Perms:    Query
      Throws:   NotImplemented
                DoesNotExist
                PermissionDenied
                AuthorizationFailed
                AuthenticationFailed
                Timeout
                NoSuccess
      Notes:    - This method returns TRUE if the attribute
                  identified by the key exists, and can be read
                  by get_attribute() or get_vector attribute(),
                  but cannot be changed by set_attribute() and
                  set_vector_attribute().
                - exceptions have the same semantics as defined
                  for the get_attribute() method description.
 
 
    - attribute_is_writable
      Purpose:  check the attribute mode.
      Format:   attribute_is_writable(in  string key,
                                      out bool   test);
      Inputs:   key:                  attribute key
      InOuts:   -
      Outputs:  test:                 bool indicating success
      PreCond:  -
      PostCond: -
      Perms:    Query
      Throws:   NotImplemented
                DoesNotExist
                PermissionDenied
                AuthorizationFailed
                AuthenticationFailed
                Timeout
                NoSuccess
      Notes:    - This method returns TRUE if the attribute
                  identified by the key exists, and can be
                  changed by set_attribute() or
                  set_vector_attribute().
                - exceptions have the same semantics as defined
                  for the get_attribute() method description.
 
 
    - attribute_is_removable
      Purpose:  check the attribute mode.
      Format:   attribute_is_removable (in  string key,
                                        out bool   test);
      Inputs:   key:                    attribute key
      InOuts:   -
      Outputs:  test:                   bool indicating success
      PreCond:  -
      PostCond: -
      Perms:    Query
      Throws:   NotImplemented
                DoesNotExist
                PermissionDenied
                AuthorizationFailed
                AuthenticationFailed
                Timeout
                NoSuccess
      Notes:    - This method returns TRUE if the attribute
                  identified by the key exists, and can be
                  removed by remove_attribute().
                - exceptions have the same semantics as defined
                  for the get_attribute() method description.
 
 
    - attribute_is_vector
      Purpose:  check the 
      Format:   attribute_is_vector  (in  string key,
                                      out bool   test);
      Inputs:   key:                  attribute key
      InOuts:   -
      Outputs:  test                  bool indicating if
                                      attribute is scalar
                                      (false) or vector (true)
      PreCond:  -
      PostCond: -
      Perms:    Query
      Throws:   NotImplemented
                DoesNotExist
                PermissionDenied
                AuthorizationFailed
                AuthenticationFailed
                Timeout
                NoSuccess
      Notes:    - This method returns TRUE if the attribute
                  identified by key is a vector attribute.
                - exceptions have the same semantics as defined
                  for the get_attribute() method description.
 \end{myspec}
 
 
 \subsubsection{Examples}
 
 \begin{mycode}
  // c++ example:
  saga::job::description jd;
 
  std::list <std::string> hosts;
  hosts.push_back ("host_1");
  hosts.push_back ("host_2");
 
  // vector attributes
  jd.set_attribute ("ExecutionHosts", hosts);
 
  // scalar attribute
  jd.set_attribute ("MemoryUsage", "1024");
 
  ...
 \end{mycode}
 
