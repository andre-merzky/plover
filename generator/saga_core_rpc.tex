 
  GridRPC is one of the few high level APIs that have been
  specified by the GGF~\cite{gridrpc}.  Thus including the
  GridRPC specification in the SAGA API benefits both SAGA and
  the GridRPC effort: SAGA becomes more complete and provides a
  better coverage of its use cases with a single
  Look-\&-Feel, whilst GridRPC gets embedded into a set
  of other tools of similar scope, which opens it to a
  potentially wider user community, and ensures its further
  development.
 
  Semantically, the methods defined in the GridRPC
  specification, as described in GFD.52~\cite{gridrpc}, map
  exactly with the RPC package of the SAGA API as described
  here. In essence, the GridRPC API has been imported into the
  SAGA RPC package, and has been equipped with the
  Look-\&-Feel, error conventions, task model, etc.  of
  the SAGA API.
 
  The rpc class constructor initializes the remote function
  handle.  This process may involve connection setup, service
  discovery, etc.  The rpc class further offers one method
  'call', which invokes the remote procedure, and returns the
  respective return data and values.  The asynchronous call
  versions described in the GridRPC specification are realized
  by the SAGA task model, and are not represented as separate
  calls here.
  
  In the constructor, the remote procedure to be invoked is
  specified by a URL, with the syntax:
 
    \shift |gridrpc://server.net:1234/my_function|
 
  with the elements responding to:
 
  \begin{tabbing}
    \shift XXXXXXXXX     \= --~~ XXXXX  \= --~~ \kill
    \shift |gridrpc|     \> --~~ scheme \> --~~ identifying a grid rpc operation\\
    \shift |server.net|  \> --~~ server \> --~~ server host serving the rpc call\\
    \shift |1234|        \> --~~ port   \> --~~ contact point for the server\\
    \shift |my_function| \> --~~ name   \> --~~ name of the remote method to invoke\\
  \end{tabbing}
  
  All elements can be empty, which allows the implementation to
  fall back to a default remote method to invoke.
  
  The argument and return value handling is very basic, and
  reflects the traditional scheme for remote procedure calls,
  that is, an array of structures acts as variable parameter
  vector. For each element of the vector, the |parameter| struct
  describes its data |buffer|, the |size| of that buffer, and
  its input/output |mode|.
  
  The |mode| value has to be initialized for each |parameter|,
  and |size| and |buffer| values have to be initialized for each
  |In| and |InOut| struct. For |Out| parameters, |size| may have
  the value |0| in which case the |buffer| must be un-allocated,
  and is to be created (e.g. allocated) by the SAGA
  implementation upon arrival of the result data, with a size
  sufficient to hold all result data.  The |size| value is to be
  set by the implementation to the allocated buffer size.  SAGA
  language bindings MUST prescribe the responsibilities for
  releasing the allocated buffer, according to usual procedures
  in the respective languages.
  
  When an |Out| or |InOut| struct uses a pre-allocated buffer,
  any data exceeding the buffer size are discarded.  The
  application is responsible for specifying correct buffer sizes
  for pre-allocated buffers; otherwise the behavior is
  undefined.
 
  This argument handling scheme allows efficient (copy-free)
  passing of parameters. The parameter vector must be passed by
  reference because it is specified as |inout| in SIDL. (See
  also Section~\ref{ssec:sidl}.) 
  
   \subsubsection{RPC Permissions}
 
    The SAGA API allows for application level
    authorization of RPC calls an application is able to set
    permissions on \T{saga::rpc} instances.  Not all
    implementations will be able to fully implement that
    security model -- the implementation MUST carefully document
    which permissions are supported, and which are not.
 
 \subsubsection{Specification}
 
 \begin{myspec}
  package saga.rpc 
  {
    enum io_mode 
    {
      In    = 1,          // input  parameter
      Out   = 2,          // output parameter
      InOut = 3           // input and output parameter
    }
 
    class parameter : extends saga::core::buffer
               // from buffer saga::core::object
               // from object saga::core::error_handler
    {
      CONSTRUCTOR (in    array<byte>       data = "",
!                  in    int               size = -1,
                   in    io_mode           mode = In,
                   out   parameter         obj);
      DESTRUCTOR  (in    parameter         obj);
 
      set_io_mode (in    io_mode           mode);
      get_io_mode (out   io_mode           mode);
    }
 
    class rpc : implements   saga::core::object
                implements   saga::core::async   
                implements   saga::core::permissions
             // from object  saga::core::error_handler
    {
      CONSTRUCTOR (in    session           s,
                   in    url               url = "", 
                   out   rpc               obj          );
      DESTRUCTOR  (in    rpc               obj          );
 
      // rpc method invocation
      call        (inout array<parameter>  parameters   );
 
      // handle management
      close       (in    float             timeout = 0.0);
    }
  }
 \end{myspec}
 
 \subsubsection{Specification Details}
 
  \subsubsection*{Enum \T{io\_mode}}
 
   The \T{io\_mode} enum specifies the modus of the
   \T{rpc::parameter} instances:
 
   |In|\\[1.5mm]
   \begin{tabular}{cp{110mm}}
     ~~ & The parameter is an input parameter: its initial
          value will be evaluated, and its data buffer will
          not be changed during the invocation of \T{call()}.
   \end{tabular}
 
   |Out|\\[1.5mm]
   \begin{tabular}{cp{110mm}}
     ~~ & The parameter is an output parameter: its initial
          value will not be evaluated, and its data buffer will
          likely be changed during the invocation of \T{call()}.
   \end{tabular}
 
   |InOut|\\[1.5mm]
   \begin{tabular}{cp{110mm}}
     ~~ & The parameter is input and output parameter: its 
          initial value will not evaluated, \I{and} its data buffer 
          will likely be changed during the invocation of \T{call()}.
   \end{tabular}
 
 
  \subsubsection*{Class \T{parameter}}
 
    The \T{parameter} class inherits the
    \T{saga::buffer} class, and adds one additional state
    attribute: \T{io\_mode}, which is read-only.  With that
    addition, the new class can conveniently be used to define
    input, inout and output parameters for RPC calls.
 
 
 \begin{myspec}
    - CONSTRUCTOR
      Purpose:  create an parameter instance
      Format:   CONSTRUCTOR          (in  array<byte> data = "",
                                      in  int         size = -1,
                                      in  io_mode     mode = In,
                                      out parameter       obj);
      Inputs:   type:                 data to be used
                size:                 size of data to be used
                io_mode:              type of parameter
      InOuts:   -
      Outputs:  parameter:            the newly created parameter 
      PreCond:  -
      PostCond: -
      Perms:    - 
      Throws:   NotImplemented
                BadParameter
                NoSuccess
      Notes:    - all notes from the buffer CONSTRUCTOR apply.
 
 
    - DESTRUCTOR
      Purpose:  destroy an parameter instance
      Format:   DESTRUCTOR           (in  parameter obj);
      Inputs:   obj:                  the parameter to destroy
      InOuts:   -
      Outputs:  -
      PreCond:  -
      PostCond: -
      Perms:    - 
      Throws:   - 
      Notes:    - all notes from the buffer DESTRUCTOR apply.
 
 
    - set_io_mode
      Purpose:  set io_mode
      Format:   set_io_mode          (in  io_mode mode);
      Inputs:   mode:                 value for io mode
      InOuts:   -
      Outputs:  -
      PreCond:  -
      PostCond: -
      Perms:    - 
      Throws:   -
      Notes:    - 
 
    - get_io_mode
      Purpose:  retrieve the current value for io mode
      Format:   get_io_mode          (out io_mode mode);
      Inputs:   -
      InOuts:   -
      Outputs:  mode:                 value of io mode
      PreCond:  -
      PostCond: -
      Perms:    - 
      Throws:   -
      Notes:    -
 \end{myspec}
 
 
  \subsubsection*{Class \T{rpc}}
 
    This class represents a remote function handle, which can be
    called (repeatedly), and returns the result of the
    respective remote procedure invocation.  
    
 
 \begin{myspec}
    - CONSTRUCTOR
      Purpose:  initializes a remote function handle
      Format:   CONSTRUCTOR  (in  session   s, 
                              in  url       url = "", 
                              out rpc       obj);
      Inputs:   s:            saga session to use
!               url:          remote method to
                              initialize
      InOuts:   -
      Outputs:  obj           the newly created object
      PreCond:  -
      PostCond: - the instance is open.
      Perms:    Query
      Throws:   NotImplemented
                IncorrectURL
                BadParameter
                DoesNotExist
                PermissionDenied
                AuthorizationFailed
                AuthenticationFailed
                Timeout
                NoSuccess
      Notes:    - if url is not given, or is empty (the 
                  default), the implementation will choose an 
                  appropriate default value.
                - according to the GridRPC specification, the 
                  constructor may or may not contact the RPC
                  server; absence of an exception does not imply
                  that following RPC calls will succeed, or that
                  a remote function handle is in fact available.
                - the following mapping MUST be applied from
                  GridRPC errors to SAGA exceptions:
                  GRPC_SERVER_NOT_FOUND   : BadParameter
                  GRPC_FUNCTION_NOT_FOUND : DoesNotExist
                  GRPC_RPC_REFUSED        : AuthorizationFailed
                  GRPC_OTHER_ERROR_CODE   : NoSuccess
                - non-GridRPC based implementations SHOULD ensure
                  upon object construction that the remote handle
                  is available, for consistency with the
                  semantics on other SAGA object constructors.
 
    - DESTRUCTOR
      Purpose:  destroy the object
      Format:   DESTRUCTOR           (in  rpc  obj)
      Inputs:   obj:                  the object to destroy
      InOuts:   -
      Outputs:  -
      PreCond:  - 
      PostCond: - the instance is closed.
      Perms:    -
      Throws:   - 
      Notes:    - if the instance was not closed before, the 
                  destructor performs a close() on the instance,
                  and all notes to close() apply.
 
 
    - call
      Purpose:  call the remote procedure
      Format:   call         (inout array<parameter> param);
      Inputs:   - 
      In/Out:   param:        argument/result values for call
      InOuts:   -
      Outputs:  - 
      PreCond:  - the instance is open.
      PostCond: - the instance is available for another call()
                  invocation, even if the present call did not
                  yet finish, in the asynchronous case.
      Perms:    Exec
      Throws:   NotImplemented
                IncorrectURL
                BadParameter
                DoesNotExist
                IncorrectState
                PermissionDenied
                AuthorizationFailed
                AuthenticationFailed
                Timeout
                NoSuccess
      Notes:    - according to the GridRPC specification, the 
                  RPC server might not be contacted before
                  invoking call(). For this reason, all notes to
                  the object constructor apply to the call()
                  method as well.
                - if an implementation finds inconsistent
                  information in the parameter vector, a 
                  'BadParameter' exception is thrown.
                - arbitrary backend failures (e.g. semantic
                  failures in the provided parameter stack, or
                  any errors occurring during the execution of
                  the remote procedure) MUST be mapped to a
                  'NoSuccess' exception, with a descriptive
                  error message.  That way, error semantics of
                  the SAGA implementation and of the RPC
                  function implementation are strictly
                  distinguished.
                - the notes about memory management from the
                  buffer class apply.
 
 
    - close
      Purpose:  closes the rpc handle instance
      Format:   close              (in  float timeout = 0.0);
      Inputs:   timeout             seconds to wait
      InOuts:   -
      Outputs:  -
      PreCond:  -
      PostCond: - the instance is closed.
      Perms:    -
      Throws:   NotImplemented
-               IncorrectState
                NoSuccess
      Notes:    - any subsequent method call on the object
                  MUST raise an 'IncorrectState' exception
                  (apart from DESTRUCTOR and close()).
                - if close() is implicitly called in the
                  DESTRUCTOR, it will never throw an exception.
                - close() can be called multiple times, with no
                  side effects.
                - for resource deallocation semantics, see 
                  Section 2.
                - for timeout semantics, see Section 2.
 \end{myspec}
 
 
 \subsubsection{Examples}
 
 \begin{mycode}
  // c++ example
  // call a remote matrix multiplication A = A * B
  try 
  {
    rpc rpc ("gridrpc://rpc.matrix.net/matrix-mult");
 
    std::vector <saga::rpc::parameter> params (2);
 
    params[0].set_data (A); // ptr to matrix A
    params[0].set_io_mode (saga::rpc::InOut);
 
    params[1].set_data (B); // ptr to matrix B
    params[1].set_io_mode (saga::rpc::In);
 
    rpc.call (params);
 
    // A now contains the result
  }
  catch ( const saga::exception & e)
  {
    std::err << "SAGA error: " 
             << e.get_message () 
             << std::endl;
  }
 
  +------------------------------------------------------------+
 
  // c++ example
  // call a remote matrix multiplication C = A * B
  try 
  {
    rpc rpc ("gridrpc://rpc.matrix.net//matrix-mult-2");
 
    std::vector <saga::rpc::parameter> params (3);
 
    params[0].set_data (NULL); // buffer will be created
    params[0].set_io_mode (saga::rpc::Out);
 
    params[1].set_data (A); // ptr to matrix A
    params[1].set_io_mode (saga::rpc::In);
 
    params[2].set_data (B); // ptr to matrix B
    params[2].set_io_mode (saga::rpc::In);
 
    rpc.call (params);
 
    // params[0].get_data () now contains the result
  }
  catch ( const saga::exception & e)
  {
    std::err << "SAGA error: " 
             << e.get_message () 
             << std::endl;
  }
 
  +------------------------------------------------------------+
 
  // c++ example
  // asynchronous version of A = A * B
  try 
  {
    rpc rpc ("gridrpc://rpc.matrix.net/matrix-mult");
 
    std::vector <saga::rpc::parameter> params (2);
 
    params[0].set_data (A); // ptr to matrix A
    params[0].set_io_mode (saga::rpc::InOut);
 
    params[1].set_data (B); // ptr to matrix B
    params[1].set_io_mode (saga::rpc::In);
 
    saga::task t = rpc.call <saga::task::ASync> (params);
 
    // do something else
 
    t.wait ();
    // A now contains the result
  }
  catch ( const saga::exception & e)
  {
    std::err << "SAGA error: " 
             << e.get_message () 
             << std::endl;
  }
 
  +------------------------------------------------------------+
 
  // c++ example
  // parameter sweep example from
  // http://ninf.apgrid.org/documents/ng4-manual/examples.html
  //
  // Monte Carlo computation of PI
  //
  try 
  {
    saga::url     uri[NUM_HOSTS]; // initialize...
    long times, count[NUM_HOSTS], sum;
 
    std::vector <saga::rpc> servers;
 
    // create the rpc handles for all URIs
    for ( int i = 0; i < NUM_HOSTS; ++i )
    {
      servers.push_back (saga::rpc (uri[i]));
    }
 
    // create persistent storage for tasks and parameter structs
    saga::task_container tc;
    std::vector <std::vector <saga:parameter> > params;
 
    // fill parameter structs and start async rpc calls
    for ( int i = 0; i < NUM_HOSTS; ++i )
    {
      std::vector <saga::rpc::parameter> param (3);
 
      param[0].set_data (i); // use as random seed
      param[0].set_io_mode (saga::rpc::In);
 
      param[1].set_data (times);
      param[1].set_io_mode (saga::rpc::In);
 
      param[2].set_data (count[i]);
      param[2].set_io_mode (saga::rpc::Out);
 
      // start the async calls
      saga::task t = servers[i].call <saga::task::Async> (param);
 
      // save the task;
      tc.add (t[i]);
 
      // save the parameter structs
      params.push_back (param);
    }
 
    // wait for all async calls to finish
    tc.wait (saga::task::All);
 
    // compute and print pi
    for ( int i = 0; i < NUM_HOSTS; ++i )
    {
      sum += count[i];
    }
 
    std::out << "PI = " 
             << 4.0 * ( sum / ((double) times * NUM_HOSTS))
             << std::endl;
  }
  catch ( const saga::exception & e)
  {
    std::err << "SAGA error: " 
             << e.get_message () 
             << std::endl;
  }
 \end{mycode}
 
 
  %  Do we need that detailed list?  We don't have that anywhere
  %  else (JSDL/DRMAA/GridFTP/...), but have a verbose statement
  %  in that respect in the description above... -- AM
 
  % \subsubsection{Notes}
  % 
  % \begin{myspec}
  %   Comparison to the original GridRPC calls:
  %   ------------------------------------------
  % 
  %     initialization:
  %     ---------------
  % 
  %     - grpc_initialize 
  %       
  %       GridRPC: reads the configuration file and initializes the
  %                required modules.
  %       SAGA:    not needed, implicit
  %      
  %      
  %     - grpc_finalize 
  % 
  %       GridRPC: releases any resources being used
  %       SAGA:    not needed, implicit
  %       
  %      
  %     handle management:
  %     ------------------
  %      
  %     - grpc_function_handle_default 
  %     
  %       GridRPC: creates a new function handle using the default
  %                server.  This could be a pre-determined server 
  %                name or it could be a server that is dynamically 
  %                chosen by the resource discovery mechanisms of 
  %                the underlying GridRPC implementation, such as 
  %                the NetSolve agent.
  %       SAGA:    default constructor
  % 
  %      
  %     - grpc_function_handle_init 
  %       
  %       GridRPC: creates a new function handle with a server
  %                explicitly specified by the user.
  %       SAGA:    explicit constructor
  %       
  %     
  %     - grpc_function_handle_destruct 
  %       
  %       GridRPC: releases the memory associated with the
  %                specified function handle.
  %       SAGA:    destructor
  %     
  %     
  %     - grpc_get_handle
  % 
  %       GridRPC: returns the handle corresponding to the given
  %                session ID (that is, corresponding to that 
  %                particular non-blocking request).
  %       SAGA:    not possible right now.
  %                However, status of asynchronous operations can be checked
  %                via the corresponding task objects.
  % 
  % 
  %     call functions:
  %     ---------------
  % 
  %     - grpc_call 
  %     
  %       GridRPC: makes a blocking remote procedure call with a variable
  %                number of arguments.
  %       SAGA:    has no variable number of arguments, this case is covered
  %                via the SAGA version of grpc_call_argstack.
  % 
  % 
  %     - grpc_call_async 
  %     
  %       GridRPC: makes a non-blocking remote procedure call with a
  %                variable number of arguments.
  %       SAGA:    done via task model and equivalent to grpc_call_argstack.
  % 
  % 
  %     - grpc_call_argstack 
  %     
  %       GridRPC: makes a blocking call using the argument stack
  %       SAGA:    call provides a parameter array of variable size
  % 
  %       
  %     - grpc_call_argstack_async 
  %     
  %       GridRPC: makes a non-blocking call using the argument stack.
  %       SAGA:    done via the task model and call
  % 
  % 
  %     asynchronous control functions:
  %     -------------------------------
  % 
  %     - grpc_probe 
  %     
  %       GridRPC: checks whether the asynchronous GridRPC call has
  %                completed.
  %       SAGA:    done via the task model
  % 
  % 
  %     - grpc_cancel 
  %     
  %       GridRPC: cancels the specified asynchronous GridRPC call.
  %       SAGA:    done via the task model
  %        
  % 
  %     asynchronous wait functions:
  %     ----------------------------
  % 
  %     - grpc_wait 
  %     
  %       GridRPC: blocks until the specified non-blocking requests to
  %                complete.
  %       SAGA:    done via the task model
  %        
  %     
  %     - grpc_wait_and 
  %     
  %       GridRPC: blocks until all of the specified non- blocking requests
  %                in a given set have completed.
  %       SAGA:    done via the task container
  % 
  %     
  %     - grpc_wait_or 
  %     
  %       GridRPC: blocks until any of the specified non- blocking requests
  %                in a given set has completed.
  %       SAGA:    done via the task container
  %            
  %     
  %     - grpc_wait_all 
  %     
  %       GridRPC: blocks until all previously issued non-blocking requests
  %                have completed.
  %       SAGA:    done via the task container
  %              
  %     
  %     - grpc_wait_any 
  %     
  %       GridRPC: blocks until any previously issued non-blocking request
  %                has completed.
  %       SAGA:    done via the task container
  %       
  % 
  %     error reporting functions:
  %     --------------------------
  % 
  %     - grpc_perror 
  %     
  %       GridRPC: prints the error string associated with the last GridRPC
  %                call.
  %       SAGA:    exceptions
  %      
  %      
  %     - grpc_error_string 
  %     
  %       GridRPC: returns the error description string, given a numeric
  %                error code.
  %       SAGA:    exceptions
  %      
  %      
  %     - grpc_get_error 
  %     
  %       GridRPC: returns the error code associated with a given
  %                non-blocking request.
  %       SAGA:    exceptions
  %     
  %      
  %     - grpc_get_last_error 
  %     
  %       GridRPC: returns the error code for the last invoked GridRPC call.
  %       SAGA:    exceptions
  % \end{myspec}
  
