 \label{sec:nonfunc}
 
  The SAGA API consists of a number of interface and class
  specifications.  The relation between these is shown in
  Figure~\ref{fig:classes} on Page~\pageref{fig:classes}.  This
  figure also marks which interfaces are part
  of the SAGA Look-\&-Feel, and which classes are
  combined into packages.  
 
  This and the next section form the normative part of the SAGA
  Core API specification.  It has one subsection for each
  package, starting with those interfaces that define the SAGA
  Look-\&-Feel,
  followed by the various, capability-providing packages: job
  management, name space management, file management, replica
  management, streams, and remote procedure call.
 
  The SAGA Look-\&-Feel is defined by a number of
  classes and interfaces which ensure the non-functional
  properties of the SAGA API (see~\cite{saga-req} for a complete
  list of non-functional requirements).  These interfaces and
  classes are intended to be used by the functional SAGA API
  packages, and are hence thought to be orthogonal to the
  functional scope of the SAGA API.
 
  \begin{figure}[!ht]
  \begin{center}
     \up\up
     \includegraphics[angle=90,height=0.97\textheight]{classes}
     \caption{\label{fig:classes}\footnotesize The SAGA class 
              and interface hierarchy.\newline
              \XCommn{added URL class, moved iovec and parameter.}}
   \end{center}
  \end{figure}

  \XAdd[4]{Section~\ref{ssec:compliance} contains important notes on
  the extent the SAGA \LF needs to be implemented by compliant
  implementations.  The \T{NotImplemented} exception is listed for a
  number of method calls, but MUST only be used under the
  circumstances described in~\ref{ssec:compliance}.} \XRem[8]{SAGA implementations should
  be able to implement the SAGA Look-\&-Feel API packages independent
  of the grid middleware backend.  This, however, might not always be
  possible, at least to a full extent.  In particular Monitoring and
  Steering, but also Attributes and asynchronous operations, may need
  explicit support from the backend system.  As such, methods in these
  four packages MUST be expected to throw a \T{NotImplemented}
  exception, in accordance with the SAGA implementation compliance
  guidelines given in Section~\ref{ssec:compliance}.}   Similarly, the
  \T{IncorrectURL} exception is listed when appropriate, but is not,
  in general, separately motivated or detailed -- the semantic
  conventions for this exception are as defined in
  Section~\ref{ssec:urlprob}.

 \clearpage
 
 
