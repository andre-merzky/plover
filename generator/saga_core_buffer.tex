 
 The SAGA API includes a number of calls which perform
 byte-level I/O operations, e.g. |read()|/|write()| on files and
 streams, and |call()| on rpc instances.  Future SAGA API
 extensions are expected to increase the number of I/O methods.
 The |saga::buffer| class encapsulates a sequence of bytes to be
 used for such I/O operations -- that allows for uniform I/O
 syntax and semantics over the various SAGA API packages.
 
 The class is designed to be a simple container containing one
 single element (the opaque data).  The data can either be
 allocated and maintained in application memory, or can be
 allocated and maintained by the SAGA implementation.  The
 latter is the default, and applies when no data and no size are
 specified on buffer construction.
 
 For example, an application that has data memory already
 allocated and filled, can create and use a buffer by calling
 
   \shift |// create buffer with application memory|\\
   \shift |char data[1000];|\\
   \shift |saga::buffer b (data, 1000);|
 
 The same also works when used with the respective I/O
 operations:
 
   \shift |// write to a file using a buffer with application |
          |memory|\\
   \shift |char data[1000] = ...;|\\
   \shift |file.write (saga::buffer (data, 1000));|
 
 Another application, which wants to leave the buffer memory
 management to the SAGA implementation, can use a second
 constructor, which causes the implementation to allocate memory
 on the fly:
 
   \shift |// create empty, implementation managed buffer|\\
   \shift |saga::buffer b;  // no data nor size given!|
 
   \shift |// read 100 byte from file into buffer|\\
   \shift |file.read (b, 100);|
 
   \shift |// get memory from SAGA|\\
   \shift |const char * data = b.get_data ();|
 
   \shift |// or use data directly|\\
   \shift |std::cout << "found: " << b.get_data () << std::endl;|
 
 Finally, an application can leave memory management to the
 implementation, as above, but can specify how much memory
 should be allocated by the SAGA implementation:
 
   \shift |// create an implementation managed buffer of 100 |
          |byte|\\
   \shift |saga::buffer b (100);|
 
   \upp
   \shift |// get memory from SAGA|\\
   \shift |const char * data = b.get_data ();|
 
   \upp
   \shift |// fill the buffer|\\
   \shift |memcpy (data, source, b.get_size ());|
 
   \upp
   \shift |// use data for write|\\
   \shift |file.write (b);|
 
 Application-managed memory MUST NOT be re- or de-allocated by
 the SAGA implementation, and implementation-managed memory MUST
 NOT be re- or de-allocated by the application.  However, an
 application CAN change the \I{content} of implementation
 managed memory, and vice versa.
 
 Also, a buffer's contents MUST NOT be changed by the application
 while it is in use, i.e. while any I/O operation on that buffer
 is ongoing.  For asynchronous operations, an I/O operation is
 considered ongoing if the associated |saga::task| instance is
 not in a final state.
 
 If a buffer is too small (i.e. more data are available for a
 read, or more data are required for a write), only the
 available data are used, and an error is returned
 appropriately.  If a buffer is too large (i.e. read is not able
 to fill the buffer completely, or write does not need the
 complete buffer), the remainder of the buffer data MUST be
 silently ignored (i.e. not changed, and not set to zero).  The
 error reporting mechanisms as listed for the specific I/O
 methods apply.
 
 Implementation-managed memory is released when the buffer is
 destroyed, (either explicitly by calling |close()|, or
 implicitly by going out of scope).  It MAY be re-allocated,
 and reset to zero, if the application calls |set_size()|.
 
 Application-managed memory is released by the application.  In
 order to simplify memory management, language bindings (in
 particular for non-garbage-collecting languages) MAY allow to
 register a callback on buffer creation which is called on
 buffer destruction, and which can be used to de-allocate the
 buffer memory in a timely manner.  The |saga::callback| class
 SHOULD be used for that callback -- those language bindings
 SHOULD thus define the buffer to be |monitorable|, i.e. it
 should implement the |saga::monitorable| interface.  After the
 callback's invocation, the buffer MUST NOT be used by the
 implementation anymore.
 
 When calling |set_data()| for application-managed buffers, the
 implementation MAY copy the data internally, or MAY use the
 given data pointer as is.  The application SHOULD thus not
 change the data while an I/O operation is in progress, and only
 consider the data pointer to be unused after another
 |set_data()| has been called, or the buffer instance was
 destroyed.
 
 Note that these conventions on memory management allow for zero- copy
 SAGA implementations, and also allow to reuse buffer
 instances for multiple I/O operations, which makes, for
 example, the implementation of pipes and filters very simple.
 
 The buffer class is designed to be inherited by application-level
 I/O buffers, which may, for example, add custom data
 getter and setter methods (e.g. |set_jpeg()| and |get_jpeg()|.
 Such derived buffer classes can thus add both data formats and
 data models transparently on top of SAGA I/O.  For developers
 who program applications for a specific community it seems
 advisable to standardize both data format and data model, and
 possibly to standardize derived SAGA buffers -- that work is,
 at the moment, out of scope for SAGA.  The SAGA API MAY,
 however, specify such derived buffer classes in later versions,
 or in future extensions of the API.  
 
 A buffer does not belong to a session, and a buffer object
 instance can thus be used in multiple sessions, for I/O
 operations on different SAGA objects.
 
 Note that even if a buffer size is given, the |len_in|
 parameter to the SAGA I/O operations supersedes the buffer
 size.  If the buffer is too small, a |'BadParameter'| exception
 will be thrown on these operations.  If |len_in| is omitted
 and the buffer size is not known, a |'BadParameter'| exception
 is also thrown.
 
 Note also that the |len_out| parameter of the SAGA I/O
 operations has not necessarily the same value as the buffer
 size, obtained with |buffer.get_size()|.  A read may read only
 a part of the requested data, and a write may have written only
 a part of the buffer.  That is not an error, as is described in
 the notes for the respective I/O operations.
 
 SAGA language bindings may want to define a |const|-version of
 the buffer, in order to allow for safe implementations.
 A non-const buffer SHOULD then inherit the |const| buffer class,
 and add the appropriate constructor and setter methods.
 The same holds for SAGA classes which inherit from the |buffer|.
 
 Also, language bindings MAY allow buffer constructors with
 optional |size| parameter, if the size of the given data is
 implicitly known.  For example, the C++ bindings MAY allow an
 buffer constructor |buffer (std::string s)|.  The same holds
 for SAGA classes that inherit from the |buffer|.
 
 \newpage
 
 \subsubsection{Specification}
 
 \begin{myspec}
  package saga.core
  {
    class buffer : implements   saga::object
                // from object  saga::error_handler
    {
      CONSTRUCTOR (in  array<byte>  data,
                   in  int          size,
                   out buffer       obj);
      CONSTRUCTOR (in  int          size = -1,
                   out buffer       obj);
      DESTRUCTOR  (in  buffer       obj);
 
      set_size    (in  int          size = -1);
      get_size    (out int          size);
 
      set_data    (in  array<byte>  data, 
                   in  int          size);
      get_data    (out array<byte>  data);
 
      close       (in  float        timeout = -0.0);
    }
  }
 \end{myspec}
 
 
 \subsubsection{Specification Details}
 
 \subsubsection*{Class \T{buffer}}
 
 \begin{myspec}
    - CONSTRUCTOR
      Purpose:  create an I/O buffer
      Format:   CONSTRUCTOR          (in  array<byte> data,
                                      in  int         size,
                                      out buffer      obj);
      Inputs:   data:                 data to be used
                size:                 size of data to be used
      InOuts:   -
      Outputs:  buffer:               the newly created buffer
      PreCond:  - size >= 0
      PostCond: - the buffer memory is managed by the
                  application.
      Perms:    -
-     Throws:   NotImplemented
!     Throws:   BadParameter
                NoSuccess
      Notes:    - see notes about memory management.
                - if the implementation cannot handle the 
                  given data pointer or the given size, a 
                  'BadParameter' exception is thrown.
                - later method descriptions refer to this
                  CONSTRUCTOR as 'first CONSTRUCTOR'.
 
 
    - CONSTRUCTOR
      Purpose:  create an I/O buffer
      Format:   CONSTRUCTOR          (in  int         size = -1,
                                      out buffer      obj);
      Inputs:   size:                 size of data buffer
      InOuts:   -
      Outputs:  buffer:               the newly created buffer
      PreCond:  -
      PostCond: - the buffer memory is managed by the
                  implementation.
                - if size > 0, the buffer memory is allocated by
!                 the implementation.
      Perms:    -
-     Throws:   NotImplemented
!     Throws:   BadParameter
                NoSuccess
      Notes:    - see notes about memory management.
                - if the implementation cannot handle the 
                  given size, a 'BadParameter' exception is 
                  thrown.
                - later method descriptions refer to this
                  CONSTRUCTOR as 'second CONSTRUCTOR'.
 
 
    - DESTRUCTOR
      Purpose:  destroy a buffer
      Format:   DESTRUCTOR           (in  buffer obj);
      Inputs:   obj:                  the buffer to destroy
      InOuts:   -
      Outputs:  -
      PreCond:  -
      PostCond: -
      Perms:    -
      Throws:   -
      Notes:    - if the instance was not closed before, the 
                  DESTRUCTOR performs a close() on the instance,
                  and all notes to close() apply.
 
 
    - set_data
      Purpose:  set new buffer data
      Format:   set_data             (in  array<byte> data, 
                                      in  int         size);
      Inputs:   data:                 data to be used in buffer
                size:                 size of given data
      InOuts:   -
      Outputs:  -
      PreCond:  -
      PostCond: - the buffer memory is managed by the
                  application.
      Perms:    -
-     Throws:   NotImplemented
!     Throws:   BadParameter
                IncorrectState
      Notes:    - the method is semantically equivalent to
                  destroying the buffer, and re-creating it with
                  the first CONSTRUCTOR with the given size.
                - the notes for the DESTRUCTOR and the first
                  CONSTRUCTOR apply.
 
 
    - get_data
      Purpose:  retrieve the buffer data
      Format:   get_data             (out array<byte> data);
      Inputs:   -
      InOuts:   -
      Outputs:  data:                 buffer data to retrieve
      PreCond:  -
      PostCond: -
      Perms:    -
-     Throws:   NotImplemented
!     Throws:   DoesNotExist
                IncorrectState
      Notes:    - see notes about memory management
                - if the buffer was created as implementation
                  managed (size = -1), but no I/O operation has
                  yet been successfully performed on the buffer,
                  a 'DoesNotExist' exception is thrown.
 
 
    - set_size
      Purpose:  set size of buffer
      Format:   set_size           (in  int   size = -1);
      Inputs:   size:               value for size
      InOuts:   -
      Outputs:  -
      PreCond:  - 
      PostCond: - the buffer memory is managed by the
                  implementation.
      Perms:    -
-     Throws:   NotImplemented
!     Throws:   BadParameter
                IncorrectState
      Notes:    - the method is semantically equivalent to
                  destroying the buffer, and re-creating it with
                  the second CONSTRUCTOR using the given size.
                - the notes for the DESTRUCTOR and the second
                  CONSTRUCTOR apply.
 
 
    - get_size
      Purpose:  retrieve the current value for size
      Format:   get_size           (out int   size);
      Inputs:   -
      InOuts:   -
      Outputs:  size               value of size
      PreCond:  -
      PostCond: -
      Perms:    -
-     Throws:   NotImplemented
!     Throws:   IncorrectState
      Notes:    - if the buffer was created with negative size
                  with the second CONSTRUCTOR, or the size was
                  set to a negative value with set_size(), this
                  method returns '-1' if the buffer was not yet
                  used for an I/O operation.  
                - if the buffer was used for a successful I/O
                  operation where data have been read into the
                  buffer, the call returns the size of the 
                  memory which has been allocated by the
                  implementation during that read operation.
 
 
    - close
      Purpose:  closes the object
      Format:   close              (in  float timeout = 0.0);
      Inputs:   timeout             seconds to wait
      InOuts:   -
      Outputs:  -
      Perms:    -
      PreCond:  -
      PostCond: - any operation on the object other than 
                  close() or the DESTRUCTOR will cause 
                  an 'IncorrectState' exception.
-     Throws:   NotImplemented
+     Throws:   -
      Notes:    - any subsequent method call on the object
                  MUST raise an 'IncorrectState' exception
                  (apart from DESTRUCTOR and close()).
                - if the current data memory is managed by the
                  implementation, it is freed.
                - close() can be called multiple times, with no
                  side effects.
                - if the current data memory is managed by the
                  application, it is not accessed anymore by the
                  implementation after this method returns.
                - if close() is implicitly called in the
                  DESTRUCTOR, it will never throw an exception.
                - for resource deallocation semantics, see 
                  Section 2.
                - for timeout semantics, see Section 2.
 \end{myspec}
 
 
 \subsubsection{Examples}
 
 \begin{mycode}
  ////////////////////////////////////////////////////////////////
  // C++ I/O buffer examples
  ////////////////////////////////////////////////////////////////
  
  ////////////////////////////////////////////////////////////////
  //
  // general examples
  //
  // all following examples ignore the ssize_t return value, which 
  // should be the number of bytes successfully read
  //
  ////////////////////////////////////////////////////////////////
  {
    char data[x][y][z];
    char* target = data + 200;
    buffer b;
    
    // the following four block do exactly the same, reading 
    // 100 byte (the read parameter supersedes the buffer size)
  
    // apps managed memory
    {
      b.set_data (target);
      stream.read (b, 100);
      printf ("%100s", target);
    }
  
    {
      b.set_data (target, 100);
      stream.read (b);
      printf ("%100s", target);
    }
  
    {
      b.set_data (target, 100);
      stream.read (b, 100);
      printf ("%100s", target);
    }
  
    {
      b.set_data (target, 200);
      stream.read (b, 100);
      printf ("%100s", target);
    }
  
  
    // now for impl managed memory
    {
      b.set_size (100);
      stream.read (b);
      printf ("%100s", b.get_data ());
    }
  
    {
      b.set_size (-1);
      stream.read (b, 100);
      printf ("%100s", b.get_data ());
    }
  
    {
      b.set_size (200);
      stream.read (b, 100);
      printf ("%100s", b.get_data ());
    }
  
  
    // these two MUST throw, even if there is 
    // enough memory available
  
    // app managed memory
    {
      b.set_data (target, 100);
      stream.read (b, 200);
    }
  
    // impl. managed memory
    {
      b.set_size (100);
      stream.read (b, 200);
    }
  }
  
  
  ////////////////////////////////////////////////////////////////
  //
  // the next 4 examples perform two reads from a stream, 
  // first 100 bytes, then 200 bytes.
  //
  ////////////////////////////////////////////////////////////////
  
  // impl managed memory
  {
    {
      buffer b;
  
      stream.read (b, 100);
      printf ("%100s", b.get_data ());
      
      stream.read (b, 200);
      printf ("%200s", b.get_data ());
  
    } // b dies here, data are gone after that
  }
  
  
  // same as above, but with explicit c'tor
  {
    {
      buffer b (100);
      stream.read (b);
      printf ("%100s", b.get_data ());
  
      b.set_size (200);
      stream.read (b);
      printf ("%200s", b.get_data ());
  
    } // b dies here, data are gone after that
  }
  
  
  // apps managed memory
  {
    char   data[x][y][z]; // the complete data set
    char * target = data; // target memory address to read into...
    target += offset;     // ... is somewhere in the data space.
  
    stream.read (buffer (target,       100));
    stream.read (buffer (target + 100, 200));
  
    printf ("%300s", target);
  
    // data must be larger than offset + 300, otherwise bang!
  }
  
  
  // same as above with explicit buffer c'tor
  {
    char   data[x][y][z]; // the complete data set
    char * target = data; // target memory address to read into...
    target += 200;          // ... is somewhere in the data space.
  
    {
      buffer b (target, 100);
      stream.read (b);
  
      b.set_data (target + 100, 200);
      stream.read (b);
  
    } // b dies here.  data are intact after that
  
    printf ("%300s", target);
  
    // data must be larger than offset + 300, otherwise bang!
  }
  
  
  ////////////////////////////////////////////////////////////////
  //
  // the next two examples perform the same reads, 
  // but switch memory management in between
  //
  ////////////////////////////////////////////////////////////////
  
  // impl managed memory, then apps managed memory
  {
    {
      char [x][y][z] data;
      char* target = data + 200;
  
      buffer b;
  
      // impl managed
      stream.read (b, 100);
      printf ("%100s", target);
  
      b.set_data (target, 200); // impl data are gone after this
  
      // apps managed
      stream.read (b);
      printf ("%200s", target);
  
    } // b dies here, apps data are ok after that, impl data are gone
  }
  
  
  // apps managed memory, then impl managed
  {
    {
      char [x][y][z] data;
      char* target = data + 200;
  
      buffer b (target);
  
      // apps managed
      stream.read (b, 100);
      printf ("%100s", target);
  
      b.set_size (-1);
  
      // impl managed
      stream.read (b, 200);
      printf ("%200s", target);
  
    } // b dies here, apps data are ok after that, impl data are gone
  }
  
  
  ////////////////////////////////////////////////////////////////
  //
  // now similar for write
  //
  ////////////////////////////////////////////////////////////////
  
  ////////////////////////////////////////////////////////////////
  //
  // general part
  //
  // all examples ignore the ssize_t return value, which should be
  // the number of bytes successfully written
  //
  ////////////////////////////////////////////////////////////////
  {
    char data[x][y][z];
    char* target = data + 200;
    buffer b;
    
    // the following four block do exactly the same, writing 
    // 100 byte (the write parameter supersedes the buffer size)
  
    // apps managed memory
    {
      b.set_data (target);
      stream.write (b, 100);
    }
  
    {
      b.set_data (target, 100);
      stream.write (b);
    }
  
    {
      b.set_data (target, 100);
      stream.write (b, 100);
    }
  
    {
      b.set_data (target, 200);
      stream.write (b, 100);
    }
  
  
    // now for impl managed memory
    {
      b.set_size (100);
      memcpy (b.get_data (), target, 100);
      stream.write (b);
    }
  
    {
      b.set_size (200);
      memcpy (b.get_data (), target, 200);
      stream.write (b, 100);
    }
  
  
    // these two MUST throw, even if there is 
    // enough memory available
  
    // app managed memory
    {
      b.set_data (target, 100);
      stream.write (b, 200); // throws BadParameter
    }
  
    // impl. managed memory
    {
      b.set_size (100);
      memcpy (b.get_data (), target, 200); // apps error
      stream.write (b, 200); // throws BadParameter
    }
  }
  
  
  ////////////////////////////////////////////////////////////////
  //
  // the next 4 examples perform two writes to a stream, 
  // first 100 bytes, then 200 bytes.
  //
  ////////////////////////////////////////////////////////////////
  
  // impl managed memory
  {
    char   data[x][y][z]; // the complete data set
    char * target = data; // target memory address to write into...
    target += offset;     // ... is actually somewhere in the data space.
  
    {
      buffer b (200);
  
      memcpy (b.get_data (), target, 100);
      stream.write (b, 100);
      
      memcpy (b.get_data (), target + 100, 200);
      stream.write (b, 200);
  
    } // b dies here, data are gone after that
  }
  
  
  // same as above, but using set_size ()
  {
    char   data[x][y][z]; // the complete data set
    char * target = data; // target memory address to write into...
    target += offset;     // ... is actually somewhere in the data space.
  
    {
      buffer b (100);
      memcpy (b.get_data (), target, 100);
      stream.write (b);
  
      b.set_size (200);
      memcpy (b.get_data (), target + 100, 200);
      stream.write (b);
  
    } // b dies here, data are gone after that
  }
  
  
  // apps managed memory
  {
    char   data[x][y][z]; // the complete data set
    char * target = data; // target memory address to write into...
    target += offset;     // ... is actually somewhere in the data space.
  
    stream.write (buffer (target,       100));
    stream.write (buffer (target + 100, 200));
  
    // data must be larger than offset + 300, otherwise bang!
  }
  
  
  // same as above with explicit buffer c'tor
  {
    char   data[x][y][z]; // the complete data set
    char * target = data; // target memory address to write into...
    target += 200;          // ... is actually somewhere in the data space.
  
    {
      buffer b (target, 100);
      stream.write (b);
  
      b.set_data (target + 100, 200);
      stream.write (b);
  
    } // b dies here.  data are intact after that
  
  
    // data must be larger than offset + 300, otherwise bang!
  }
  
  
  ////////////////////////////////////////////////////////////////
  //
  // the next two examples perform the same reads, 
  // but switch memory management in between
  //
  ////////////////////////////////////////////////////////////////
  
  // impl managed memory, then apps managed memory
  {
    {
      char [x][y][z] data;
      char* target = data + 200;
  
      buffer b (100);
  
      // impl managed
      memcpy (b.get_data (), target, 100);
      stream.write (b, 100);
  
      b.set_data (target + 100, 200); // apps managed now
                                      // impl data are gone after this
  
      // apps managed
      stream.write (b);
  
    } // b dies here, apps data are ok after that, impl data are gone
  }
  
  
  // apps managed memory, then impl managed
  {
    {
      char [x][y][z] data;
      char* target = data + 200;
  
      buffer b (target);
  
      // apps managed
      stream.write (b, 100);
  
      b.set_size (200); // impl managed now
      memcpy (b.get_data (), target + 100, 200);
  
      // impl managed
      stream.write (b);
  
    } // b dies here, apps data are ok after that, impl data are gone
  }
 \end{mycode}
 
