 
  \XComm[2]{Note that these changes to the SAGA error handling
  should be backward compatible to the original specification,
  as far as they do not correct errors.}
 
  All objects in SAGA implement the |error_handler| interface,
  which allows a user of the API to query for the latest error
  associated with a SAGA object (pull).  In languages with
  exception-handling mechanisms, such as Java, C++ and Perl, the
  language binding MAY allow exceptions to be thrown
  \XAdd[2]{instead}.  If an exception handling mechanism is
  included in a language binding, the \T{error\_handler} MUST
  NOT be included in the same binding.  Bindings for languages
  without exception handling capabilities MUST stick to the
  |error_handler| interface described here, but MAY define
  additional language-native means for error reporting.
  \XAdd[2]{This document describes error conditions in terms of
  exceptions.}
 
  For objects implementing the |error_handler| interface, each
  \XAdd{synchronous} method invocation on that object resets any
  error caused by a previous method invocation on that object.
  For asynchronous operations, the error handler interface is
  provided by the task instance performing the operation, and
  not by the object which created the task.  If an error occurs
  during object creation, then the error handler interface of
  the session the object was to be created in will report the
  error.
  
  In languages bindings where this is appropriate, some API
  methods \XAdd{MAY} return POSIX |errno| codes for errors.
  This is the case in particular for |read()|, |write()| and
  |seek()|, for |saga::file| and |saga::stream|.  The respective
  method descriptions provide explicit details of how |errno|
  error codes are utilized.  In any case, whenever numerical
  |errno| codes are used, they have to be conforming to
  POSIX.1~\cite{posix1}.
 
  \XRem[2]{Any other details of the error handling mechanisms
  will be defined in the respective language bindings, if
  required.}
 
  Each SAGA API call has an associated list of exceptions it may
  throw.  These exceptions all extend the |saga::exception|
  class described below.  \XAdd[2]{The SAGA implementation MUST NOT
  throw any other SAGA exception on that call.}

  \XAdd[1]{SAGA exceptions can be hierarchical -- for details,
  see below.}

 
 \newpage
 
 \subsubsection{Specification}
 
 \begin{myspec}
  package saga.core
  {
    enum exception_type
    {
      IncorrectURL         =  1,
      BadParameter         =  2,
      AlreadyExists        =  3,
      DoesNotExist         =  4,
      IncorrectState       =  5,
+     IncorrectType        =  6,
      PermissionDenied     =  7,
      AuthorizationFailed  =  8,
      AuthenticationFailed =  9,
      Timeout              = 10,
      NoSuccess            = 11,
      NotImplemented       = 12
    }

    class exception
    {
      CONSTRUCTOR          (in  object           obj,
                            in  string           message,
                            out exception        e);
      CONSTRUCTOR          (in  string           message,
                            out exception        e);
      DESTRUCTOR           (void);
 
+     // top level exception information
      get_message          (out string           message);
      get_object           (out object           obj);
+     get_type             (out exception_type   t);
+
+     // recursive exception information
+     get_all_exceptions   (out array<exception> elist);
+     get_all_messages     (out array<string>    mlist);
    }
 
    class incorrect_url         : extends saga::exception { }
    class bad_parameter         : extends saga::exception { }
    class already_exists        : extends saga::exception { }
    class does_not_exist        : extends saga::exception { }
    class incorrect_state       : extends saga::exception { }
+   class incorrect_type        : extends saga::exception { }
    class permission_denied     : extends saga::exception { }
    class authorization_failed  : extends saga::exception { }
    class authentication_failed : extends saga::exception { }
    class timeout               : extends saga::exception { }
    class no_success            : extends saga::exception { }
    class not_implemented       : extends saga::exception { }
 
 
    interface error_handler
    {
      has_error            (out boolean        has_error);
      get_error            (out exception      error);
    }
  }
 \end{myspec}
 

 \subsubsection{Specification Details}
 
  SAGA provides a set of well-defined exceptions (error states)
  which MUST be supported by the implementation.  As to whether
  these error states are critical, non-critical or fatal depends
  on, (a) the specific implementation (one implementation might
  be able to recover from an error while another implementation
  might not), and (b) the specific application use case (e.g.
  the error 'file does not exist' may or may not be fatal,
  depending on whether the application really needs information
  from that file).
  
  In language bindings where this is appropriate, some SAGA
  methods do not raise exceptions on certain error conditions,
  but return an error code instead.  For example, |file.read()|
  might return an error code indicating that  not enough data is
  available right now.  The error codes used in SAGA are based
  on the definitions for |errno| as defined by POSIX, and MUST
  be used in a semantically identical manner.
  
  \XAdd[7]{For try/catch blocks which cover multiple API calls,
  on multiple SAGA objects, the \T{get\_object()} method allows
  to retrieve the object which caused the exception to be
  thrown.  In general, it will not be possible, however, to
  determine the method call which caused the exception post
  mortem.  \T{get\_object()} can also be used for exceptions
  raised by asynchronous method calls (i.e. on
  \T{task::rethrow()}, to retrieve the object on which that task
  instance was created.}

  This specification defines the set of allowed exceptions for
  each method explicitly -- this set is normative: other SAGA
  exceptions MUST NOT be thrown on these methods. Also,
  implementations MUST NOT specify or use other SAGA exceptions
  than listed in this specification.
 
  Additionally, an implementation MAY throw other, non-SAGA
  exceptions, e.g. on system errors, resource shortage etc.
  \XAdd[2]{These exception SHOULD only signal local errors,
  raised by the SAGA implementation, not errors raised by the
  Grid backend.}
  SAGA implementations MUST, \XRemn{however, }translate grid
  middleware-specific exceptions and error conditions into SAGA
  exceptions whenever possible, in order to avoid middleware
  specific exception handling on applications level -- that
  would clearly contradict the intent of SAGA to be middleware
  independent.
  
  In the SAGA language bindings, exceptions are either derived
  from the base SAGA exception types, or are error codes with
  that specific name etc.   Note that the detailed description
  for \T{saga::exception} below does not list the
  \T{CONSTRUCTOR}s and \T{DESTRUCTOR}s for all exception classes
  individually, but only for the base exception class.  The
  individual exception classes MUST NOT add syntax or semantics
  to the base exception class.
  
  The string returned by  |get_message()|
  MUST be formatted as follows: 
  
   \shift |"<ExceptionName>:| |message"|
 
  where |<ExceptionName>| MUST match the literal exception
  \XRep{names}{type enum} as defined in this document, and
  |message| SHOULD be a detailed, human readable description of
  the cause of the exception.  \XAdd[7]{The error message SHOULD
  include information about the middleware binding, and
  information about the remote entities and remote operation
  which caused the exception.  It CAN contain newlines.  When
  messages from multiple errors are included in the returned
  string, then each of these messages MUST follow the format
  defined above, and the individual messages MUST be delimited
  by newlines.  Also, indentation SHOULD be used to structure
  the output for long messages.}


  \newpage
  \subsubsection*{Hierarchical SAGA Exceptions}

  \XAdd[11]{SAGA implementations may be late binding, i.e. may
  allow to interface to multiple backends at the same time, for
  a single SAGA API call.  In such implementations, more than
  one exception may be raised for a single API call.  This
  specification proposes an algorithm to determine the most
  'interesting' exception, which is to be throw by the API
  call.  SAGA implementations MAY implement other algorithms,
  but MUST document how it determines the exception to be thrown
  from the list of backend exceptions.  Further, the thrown
  exception MUST allow for inspection of the complete list of
  backend exceptions, via \T{get\_all\_exceptions()}, and
  \T{get\_all\_messages()}.  Further, the error message of the
  thrown (top level) exception MUST include information about
  the other (lower level) exceptions.}

  \XAdd[5]{In the exception list returned by
  \T{get\_all\_exceptions()}, the top level (thrown) exception
  MUST be included again, as first member of the list, to allow
  for a uniform handling of all exceptions.  To avoid infinite
  recursion, however, that copy MUST NOT have any sub-exceptions,
  i.e. the list returned by a call to \T{get\_all\_exceptions()}
  MUST be empty.  See at the end of this section for an
  extensive example.}

  
  \XRem[9]{For implementations with multiple middleware
  bindings, it can be difficult to provide detailed and
  conclusive error messaging with a single exception.  To
  support such implementations, language bindings MAY allow
  nested exceptions.  The outermost exception MUST, however,
  follow the syntax and semantics guidelines described above.
  Implementations of such bindings which only bind to a single
  backend MUST support the defined interface for nested
  exceptions as well, in order to keep the application
  independent of the specifics of the SAGA implementation, but
  will then in general not be able to return lower-level
  exceptions.}

 

 \subsubsection*{Enum \T{exception\_type}}

  The exception types available in SAGA are listed below, with a
  number of explicit examples on when exceptions should be
  thrown.  These examples are not normative, but merely
  illustrative.  \XAdd[10]{As discussed above, multiple
  exceptions may apply to a single SAGA API call, in the case of
  late binding implementations.  In that case, the
  implementation must pick one of the exceptions to be thrown as
  'top level' exception, with all other exceptions as
  subordinate 'lower level' exceptions.  In general, that top
  level exception SHOULD be that exception which is most
  interesting to the user or application.  Although we are
  fully aware of the fact that the notion of 'interesting' is
  vague, and highly context dependent, we propose the following
  mechanism to derive the top level exception -- implementations
  MAY use other schemes to determine the top level exception,
  but MUST document that mechanism:}

  \begin{enumerate}
  
   \item \XAdd[2]{NotImplemented is only allowed as top level 
         exception, if no other exception types are present.}

   \item \XAdd[5]{Exceptions from a backend which previously
         performed a successful API call on the same remote
         entity, or on the same SAGA object instance, are more
         interesting than exceptions from other backends, and
         are in particular more interesting than exceptions from
         backends which did not yet manage to perform any
         successful operation on that entity or instance.}
        
   \item \XAdd[6]{Errors which get raised early when executing
         the SAGA API call are less interesting than errors
         which occur late.  E.g. BadParameter from the FTP
         backend is less interesting than PermissionDenied from
         the WWW backend, as the WWW backend seemed to at least
         be able to handle the parameters, to access the backend
         server, and to perform authentication, whereas the FTP
         backend bailed out early, on the functions parameter
         check.}

  \end{enumerate}


  
  \XAdd[4]{In respect to item 3 above,} the list of exceptions
  \XAddn{below} is sorted, with the most specific \XAddn{(i.e.
  interesting)} exceptions listed first and least specific last.
  \XAddn{This list is advisory, i.e.  implementation MAY use a
  different sorting, which also may vary in different
  contexts.}  
  
  \XRem[6]{The most specific exception possible (i.e.
  applicable) MUST be thrown on all error conditions.  This
  means that if multiple exceptions are applicable to an error
  condition (e.g.  \T{PermissionDenied} and \T{NoSuccess} for
  opening a file with incorrect permissions), then that
  exception \XRepn{MUST}{SHOULD} be thrown which gives more
  specific information about the respective error condition:
  e.g., \T{PermissionDenied} describes the error condition
  \XRemn{much} more explicitly than a generic \T{NoSuccess}}.
 
 
  \subsubsection*{$\bullet$ \T{IncorrectURL}}\up
  
   This exception is thrown if a method is invoked with a URL
   argument that could not be handled. This error specifically
   indicates that an implementation cannot handle the specified
   protocol, or that access to the specified entity via the
   given protocol is impossible.  The exception MUST NOT be used
   to indicate any other error condition.  See also the notes to
   'The URL Problem' in Section~\ref{ssec:urlprob}.
 
    Examples:
 
    \begin{shortlist}
     \item An implementation based on gridftp might be unable to
       handle http-based URLs sensibly, and might be unable to
       translate them into gridftp based URLs internally.  
       The implementation should then throw an |IncorrectURL|
       exception if it encounters a http-based URL.
     \item A URL is well formed, but includes
       characters or path elements which are not supported by
       the SAGA implementation or the backend. Then, an
       \T{IncorrectURL} exception is thrown, with detailed
       information on why the URL could not be used.
    \end{shortlist}
 
 
  \subsubsection*{$\bullet$ \T{BadParameter}}\up
 
    This exception indicates that at least one of the parameters
    of the method call is ill-formed, invalid, out of bounds or
    otherwise not usable.  The error message MUST give specific
    information on what parameter caused the exception, and why.
 
    Examples:
 
    \begin{shortlist}
     \item a specified context type is not supported by the
           implementation
     \item a file name specified is invalid, e.g. too long, or
           contains characters which are not allowed
     \item an ivec for scattered read/write is invalid, e.g. has
           offsets which are out of bounds, or refer to
           non-allocated buffers
     \item a buffer to be written and the specified lengths are
           incompatible
     \item an enum specified is not known
     \item flags specified are incompatible (|ReadOnly| 
           and |Truncate|)
    \end{shortlist}
 
 
  \subsubsection*{$\bullet$ \T{AlreadyExists}}\up
 
    This exception indicates that an operation cannot succeed
    because an entity to be created or registered already exists
    or is already registered, and cannot be overwritten.
    Explicit flags on the method invocation may allow the
    operation to succeed, e.g. if they indicate that Overwrite
    is allowed.
 
    Examples:
 
    \begin{shortlist}
     \item a target  for a file move already exists
     \item a file    to be created already exists
     \item a name    to be added to a logical file is already 
           known
     \item a metric  to be added to a object has the same name 
           as an existing metric on that object
    \end{shortlist}
 
 
  \subsubsection*{$\bullet$ \T{DoesNotExist}}\up
 
    This exception indicates that an operation cannot succeed
    because a required entity is missing.  Explicit flags on the
    method invocation may allow the operation to succeed, e.g.
    if they indicate that Create is allowed.
 
    Examples:
 
    \begin{shortlist}
     \item a file      to be moved  does not exist
     \item a directory to be listed does not exist
     \item a name      to be deleted    is not in a replica set
     \item a metric    asked for is not known to the object
     \item a context   asked for is not known to the session
     \item a task      asked for is not in a task container
     \item a job       asked for is not known by the backend
     \item an attribute asked for is not supported
    \end{shortlist}
 
 
  \subsubsection*{$\bullet$ \T{IncorrectState}}\up
 
    This exception indicates that the object a method was called
    on is in a state where that method cannot possibly succeed.
    A change of state might allow the method to succeed with the
    same set of parameters.
 
    Examples:
 
    \begin{shortlist}
     \item calling read   on a stream which is  not connected
     \item calling run    on a task   which was canceled
     \item calling resume on a job    which is  not suspended
    \end{shortlist}
 
 
  \subsubsection*{$\bullet$ \T{IncorrectType}}\up
 
    \XAdd[6]{
    This exception indicates that a specified type does not match any
    of the available types.  This exception is in particular reserved
    for places in the SAGA API which specify function return types in a
    template like manner, such as for \T{task.get\_result()}.
    Language binding MAY replace that exception by language specific
    means of explicit/implicit type conversion, and SHOULD try to enforce type
    mismatch errors on compile time instead of linktime or runtime.}
 
    \XAdd[2]{Examples:}
 
    \begin{shortlist}
     \item \XAdd[3]{calling \T{get\_result <string> ()} on  task which actually
     encapsulates an \T{int} typed \T{file.get\_size ()} operation.}
    \end{shortlist}
 
 
  \subsubsection*{$\bullet$ \T{PermissionDenied}}\up
 
    An operation failed because the identity used for
    the operation did not have sufficient permissions to perform
    the operation successfully.  The authentication and
    authorization steps have been completed successfully.
 
    Examples:
 
    \begin{shortlist}
     \item attempt to change or set    a ReadOnly attribute
     \item attempt to change or update a ReadOnly metric
     \item calling write  on a file which is opened for read only
     \item calling read on a file which is opened for 
           write only
     \item although a user could login to a remote host via 
           GridFTP and could be mapped to a local user, the 
           write on /etc/passwd failed.
    \end{shortlist}
 
    
 
 
  \subsubsection*{$\bullet$ \T{AuthorizationFailed}}\up
 
    An operation failed because none of the available contexts
    of the used session could be used for successful
    authorization. That error indicates that the
    resource could not be accessed at all, and not that an
    operation was not available due to restricted permissions.
    The authentication step has been completed successfully.
 
    The differences between AuthorizationFailed and
    PermissionDenied are, admittedly, subtle.  Our intention
    for introducing both exceptions was to allow to
    distinguish between administrative authorization failures
    (on VO and DN level), and backend related 
    authorization failures (which can often be resolved on 
    user level).
 
    The AuthorizationFailed exception SHOULD be thrown when
    the backend does not allow the execution of the 
    requested operation at all, whereas the PermissionDenied 
    exception SHOULD be thrown if the operation was executed, 
    but failed due to insufficient privileges.
 
    Examples:
 
    \begin{shortlist}
     \item although a certificate was valid on a remote GridFTP
       server, the distinguished name could not be mapped to a
       valid local user id.  A call to file.copy() should then
       throw an AuthorizationFailed exception.
    \end{shortlist}
 
 
  \subsubsection*{$\bullet$ \T{AuthenticationFailed}}\up
  
    An operation failed because none of the available session
    contexts could successfully be used for
    authentication\XAdd[4]{, or the implementation could not
    determine which context to use for the operation}.
 
    Examples:
 
    \begin{shortlist}
     \item a remote host does not accept a X509 certificate because
       the respective CA is unknown there.  A call to
       file.copy() should then throw an AuthenticationFailed
       exception.
    \end{shortlist}
 
 
  \subsubsection*{$\bullet$ \T{Timeout}}\up
 
    This exception indicates that a remote operation did not complete
    successfully because the network communication or the remote
    service timed out.  The time waited before an
    implementation raises a |Timeout| exception depends on
    implementation and backend details, and SHOULD be documented by
    the implementation.  This exception MUST NOT be
    thrown if a timed |wait()| or similar method times out. The latter is not an error condition
    and gets indicated by the method's return value.
 
    Examples:
 
    \begin{shortlist}
     \item a remote file authorization request timed out
     \item a remote file read operation timed out
     \item a host name resolution timed out
     \item a started file transfer stalled and timed out
     \item an asynchronous file transfer 
           stalled and timed out
    \end{shortlist}
 
 
  \subsubsection*{$\bullet$ \T{NoSuccess}}\up
 
    This exception indicates that an operation failed
    semantically, e.g. the operation was not successfully
    performed.   This exception is the least specific exception
    defined in SAGA, and CAN be used for all error conditions
    which do not indicate a more specific exception specified
    above.  The error message SHOULD always contain
    some further detail, describing the circumstances which
    caused the error condition.
 
    Examples:
 
    \begin{shortlist}
     \item a once open file is not available right now
     \item a backend response cannot be parsed
     \item a remote procedure call failed due to a 
           corrupted parameter stack
     \item a file copy was interrupted mid-stream, due to 
           shortage of disk space
    \end{shortlist}
 

  \subsubsection*{$\bullet$ \T{NotImplemented}}\up
  
    If a method is specified in the SAGA API, but cannot be
    provided by a specific SAGA implementation, this exception
    MUST be thrown.  Object constructors can also throw that
    exception, if the respective object is not implemented by
    that SAGA implementation at all. See also the notes about
    compliant implementations in Section~\ref{ssec:compliance}.
 
    Examples:
 
    \begin{shortlist}
     \item An implementation based on Unicore might not be able
       to provide streams.  The |saga::stream_server| 
       constructor should throw a NotImplemented exception for 
       such an implementation.
    \end{shortlist}
 
 
 
 
 \subsubsection*{Class \T{exception}}
  
    This is the exception base class inherited by all exceptions
    thrown by a SAGA object implementation.  \XAdd[3]{Wherever
    this specification specifies the occurrence of an instance of
    this class, the reader MUST assume that this could also be
    an instance of any subclass of \T{saga::exception}, as
    specified by this document.}
 
    Note that |saga::exception| does not implement the
    |saga::object| interface.
 
 \begin{myspec}
    - CONSTRUCTOR
      Purpose:  create the exception
      Format:   CONSTRUCTOR   (in  object         obj,
                               in  string         message
                               out exception      e);
      Inputs:   obj:           the object associated with the
                               exception.
                message:       the message to be associated 
                               with the new exception
      InOuts:   -
      Outputs:  e:             the newly created exception
      PreCond:  -
      PostCond: -
      Perms:    -
      Throws:   -
      Notes:    -
 
 
    - CONSTRUCTOR
      Purpose:  create the exception,  without associating 
                a saga object instance
      Format:   CONSTRUCTOR   (in  string         message
                               out exception      e);
      Inputs:   message:       the message to be associated 
                               with the new exception
      InOuts:   -
      Outputs:  e:             the newly created exception
      PreCond:  -
      PostCond: -
      Perms:    -
      Throws:   -
      Notes:    -
 
 
    - DESTRUCTOR
      Purpose:  destroy the exception
      Format:   DESTRUCTOR    (in  exception e);
      Inputs:   e:             the exception to destroy
      InOuts:   -
      Outputs:  -
      PreCond:  -
      PostCond: -
      Perms:    -
      Throws:   -
      Notes:    -
 
 
    - get_message
      Purpose:  gets the message associated with the exception    
      Format:   get_message   (out string message);
      Inputs:   -
      InOuts:   -
      Outputs:  message:       the error message
      PreCond:  -
      PostCond: -
      Perms:    -
      Throws:   -
      Notes:    - the returned string MUST be formatted as
                  described earlier in this section.
 
 
    - get_object
      Purpose:  gets the SAGA object associated with exception
      Format:   get_object     (out object obj);
      Inputs:   -
      InOuts:   -
      Outputs:  obj:            the object associated with the
                                exception
      PreCond:  - an object was associated with the exception
                  during construction.
      PostCond: -
      Perms:    -
      Throws:   DoesNotExist
                NoSuccess
      Notes:    - the returned object is a shallow copy of the 
                  object which was used to call the method which 
                  caused the exception.
                - if the exception is raised in a task, e.g. on
                  task.rethrow(), the object is the one which the
                  task was created from.  That allows the
                  application to handle the error condition 
                  without the need to always keep track of 
                  object/task relationships.
                - an 'DoesNotExist' exception is thrown when no
                  object is associated with the exception, e.g. 
                  if an 'NotImplemented' exception was raised 
                  during the construction of an object.
+
+   - get_type
+     Purpose:  gets the type associated with the exception    
+     Format:   get_type      (out exception_type type);
+     Inputs:   -
+     InOuts:   -
+     Outputs:  type:          the error type
+     PreCond:  -
+     PostCond: -
+     Perms:    -
+     Throws:   -
+     Notes:    - 
+
+   - get_all_exceptions
+     Purpose:  gets list of lower level exceptions
+     Format:   get_all_exceptions (out array<exception> el);
+     Inputs:   -
+     InOuts:   -
+     Outputs:  el:            list of exceptions
+     PreCond:  -
+     PostCond: -
+     Perms:    -
+     Throws:   -
+     Notes:    - a copy of the exception upon which this 
+                 method is called MUST be the first element 
+                 of the list, but that copy MUST NOT return 
+                 any exceptions when get_all_exceptions()
+                 is called on it.
+
+   - get_all_messages
+     Purpose:  gets list of lower level error messages
+     Format:   get_all_messages    (out array<string> ml);
+     Inputs:   -
+     InOuts:   -
+     Outputs:  ml:            list of error messages
+     PreCond:  -
+     PostCond: -
+     Perms:    -
+     Throws:   -
+     Notes:    - a copy of the error message of the exception 
+                 upon which this method is called MUST be the 
+                 first element of the list.
 \end{myspec}
 
 
 \newpage
 
 \subsubsection*{Interface \T{error\_handler}}
 
 The \T{error\_handler} interface allows the application
 to retrieve exceptions.  An alternative approach would be to
 return an error code for all method invocations.  This,
 however, would put a significant burden on languages with
 exception handling, and would also complicate the management of
 return values.  Language bindings for languages with exception
 support will thus generally \I{not} implement the
 \T{error\_handler} interface, but use exceptions instead.
 
 Implementations which are using the interface maintain
 an internal error state for each class instance providing the
 interface.  That error state is \T{false} by default, and is
 set to \T{true} whenever an method invocation meets an error
 condition which would, according to this specification, result
 in an exception to be thrown.
 
 The error state of an object instance can be tested with
 \T{has\_error()}, and the respective exception can be retrieved
 with \T{get\_error()}.  \XRep{Any one of these calls}{The
 \T{get\_error()} call} clears the error state (i.e.  resets it
 to \T{false}).  Note that there is no other mechanism to clear
 an error state -- that means in particular that any successful
 method invocation on the object leaves the error state
 unchanged.  If two or more subsequent operations on an object
 instance fail, then only the last exception is returned on
 \T{get\_error()}.  That mechanism allows the execution of a
 number of calls, and to check if they resulted in any error
 condition, somewhat similar to \T{try}/\T{catch} statements in
 languages with exception support.  However, it must be noted
 that an exception does \I{not} cause subsequent methods to
 fail, and does not inhibit their execution. 
 
 If \T{get\_error()} is called on an instance whose error
 state is \T{false}, an \T{Incor\-rectState} exception is
 returned, which MUST state explicitly that the
 \T{get\_\-error()} method has been invoked on an object instance
 which did not encounter an error condition.
 
 \begin{myspec}
    - has_error
      Purpose:  tests if an object method caused an exception
      Format:   has_error     (out bool       has_error);
      Inputs:   -
      InOuts:   -
      Outputs:  has_error:     indicates that an exception was
                               caught.
      PreCond:  -
-     PostCond: - the internal error state is false.
+     PostCond: - the internal error state is unchanged.
      Perms:    -
      Throws:   -
      Notes:    - 
 
 
    - get_error
      Purpose:  retrieve an exception catched during a member
                method invocation.
      Format:   get_error     (out exception  e);
      Inputs:   -
      InOuts:   -
      Outputs:  e:             the caught exception
      PreCond:  - the internal error state is true.
      PostCond: - the internal error state is false.
      Perms:    -
-     Throws:   NotImplemented
-               IncorrectURL
-               BadParameter
-               AlreadyExists
-               DoesNotExist
-               IncorrectState
-               PermissionDenied
-               AuthorizationFailed
-               AuthenticationFailed
-               Timeout
-               NoSuccess
-     Notes:    - the method throws the error/exception it is
-                 reporting about.
-               - an 'IncorrectState' exception is also thrown 
-                 if the internal error state is false. 
+     Throws:   IncorrectState
+     Notes:    - an 'IncorrectState' exception is thrown 
+                 if the internal error state is false. 
 \end{myspec}
 
 
 \newpage
 
 \subsubsection{Examples}
 
 \begin{mycode}
  ////////////////////////////////////////////////////////////////
  // 
  // C++ examples for exception handling in SAGA
  // 
  ////////////////////////////////////////////////////////////////
 
  ////////////////////////////////////////////////////////////////
  // 
  // simple exception handling
  //
  int main ()
  {
    try
    {
      saga::file f ("file://remote.host.net/etc/passwd");
      f.copy ("file:///usr/tmp/passwd.bak");
    }
 
    catch ( const saga::exception::PermissionDenied & e )
    {
      std::cerr << "SAGA error: No Permissions!" << std::endl;
      return (-1);
    }
 
    catch ( const saga::exception & e )
    {
      std::cerr << "SAGA error: " 
                << e.get_message () 
                << std::endl;
      return (-1);
    }
 
    return 0;
  }

 
  ////////////////////////////////////////////////////////////////
  // 
  // recursive exception handling
  //
  int main ()
  {
    try 
    {
      saga::file f ("any://remote.host.net/etc/passwd");
      f.copy ("any:///usr/tmp/passwd.bak");
    }

    // handle a specific error condition
    catch ( const saga::permission_denied & e )
    {
      ...
    }

    // handle all error conditions
    catch ( const saga::exception & e )
    {
       std::cerr << e.what () << std::endl;
       // prints complete set of error messages:
       // DoesNotExist: ftp adaptor: /etc/passwd does not exist
       //   DoesNotExist: ftp adaptor: /etc/passwd: does not exist
       //   DoesNotExist: www adaptor: /etc/passwd: access denied

       // handle backend exceptions individually
       std::list <saga::exception> el = e.get_all_exceptions ();

       for ( int i = 0; i < el.size (); i++ )
       {
         saga::exception esub = el[i];
         std::list <saga::exception> esubl = esub.get_all_exceptions ();
         // subl MUST be empty for i==0
         // subl MAY  be empty for i!=0 

         switch ( sub.get_type () )
         {
           // handle individual exceptions
           case saga::exception::DoesNotExist:
             ...
           case saga::exception::PermissionDenied:
             ...
         }
       }


       // handle backend exception messages individually
       std::list <saga::exception> ml = e.get_all_messages ();

       for ( int i = 0; i < ml.size (); i++ )
       {
         std::cerr << ml[i] << std::endl;
       }
       // the loop above will result in
       // DoesNotExist: ftp adaptor: /etc/passwd: does not exist
       // DoesNotExist: www adaptor: /etc/passwd: access denied
    }

    return 0;
  }
         
 
  ////////////////////////////////////////////////////////////////
  // 
  // exception handling for tasks
  //
  int main ()
  {
    saga::file f ("file://remote.host.net/etc/passwd");
 
    saga::task t = f.copy <saga::task::Async> 
                           ("file:///usr/tmp/passwd.bak");
 
    t.wait ();
 
    if ( t.get_state () == saga::task::Failed )
    {
      try {
        task.rethrow ();
      }
      catch ( const saga::exception & e )
      {
        std::cout << "task failed: " 
                  << e.what () 
                  << std::endl;
      }
      return (-1);
    }
    return (0);
  }
 \end{mycode}
 
 
 
 
