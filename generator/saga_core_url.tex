 
 In many places in the SAGA API, URLs are used to reference remote
 entities.  In order to
 
  \begin{itemize}
   \item simplify the construction and the parsing of URLs on
         application level,
   \item allow for sanity checks within and outside the SAGA
         implementation,
   \item simplify and unify the signatures of SAGA calls which 
         accept URLs,
  \end{itemize}
 
 a SAGA URL class is used.  This class provides means to set and
 access the various elements of a URL.  The class parses the URL
 in conformance to RFC-3986~\cite{rfc-3986}.
 
 In respect to the URL problem (stated in
 Section~\ref{ssec:urlprob}), the class provides the method
 |translate (in string scheme)|, which allows to translate a URL
 from one scheme to another -- with all the limitations
 mentioned in Section~\ref{ssec:urlprob}.

 \XAdd[6]{Note that resolving relative URLs (or, more specific,
 relative path components in URLs) is often non-trivial.  In
 particular, such resolution may need to be deferred until the
 URL is used, as the resolution will usually depend on the
 context of usage.  If not otherwise specified in this document,
 a URL used in some object method will be considered relative to
 the object's CWD, if that is available, or otherwise to the
 application's working directory.}

 \XAdd[7]{URLs require some characters to be escaped, in order
 to allow for the URLS to be well formatted.  The setter methods
 described below MUST perform character escaping transparently.
 That may not always be possible for the \T{CONSTRUCTOR} and
 \T{set\_string()}, which will then raise a \T{BadParameter}
 exception.  The getter methods MUST return unescaped versions
 of the URL components.  However, the string returned by the
 method \T{get\_escaped()} MUST NOT contain unescaped
 characters.}

 \XAdd[3]{This specification is silent about URL encoding issues
 -- those are left to the implementation. For additional notes
 on URL usage and implementation, see
 Section~\ref{ssec:namespaces}.}
 
 \newpage 

 \begin{myspec}
  package saga.core
  {
    class url : implements   saga::core::object
             // from object  saga::core::error_handler
    {
      CONSTRUCTOR   (in  string     url       ,
                     out buffer     obj       );
      DESTRUCTOR    (in  buffer     obj       );
 
      set_string    (in  string     url       = "");
      get_string    (out string     url       );
+     get_escaped   (out string     url       );
 
      set_scheme    (in  string     scheme    = "");
      get_scheme    (out string     scheme    );
 
      set_host      (in  string     host      = "");
      get_host      (out string     host      );
 
      set_port      (in  int        port      = "");
      get_port      (out int        port      );
 
      set_fragment  (in  string     fragment  = "");
      get_fragment  (out string     fragment  );
 
      set_path      (in  string     path      = "");
      get_path      (out string     path      );
 
      set_query     (in  string     query     = "");
      get_query     (out string     query     );
 
      set_userinfo  (in  string     userinfo  = "");
      get_userinfo  (out string     userinfo  );
 
      translate     (in  session    s         ,
                     in  string     scheme    ,
                     out url        url       );
      translate     (in  string     scheme    ,
                     out url        url       );
    }
  }
 \end{myspec}
 
 
 \subsubsection{Specification Details}
 
 \subsubsection*{Class \T{url}}
 
 \begin{myspec}
    - CONSTRUCTOR
      Purpose:  create a url instance
      Format:   CONSTRUCTOR          (in  string url = "",
                                      out url    obj);
      Inputs:   url:                  initial URL to be used
      InOuts:   -
      Outputs:  obj:                  the newly created url
      PreCond:  -
      PostCond: -
      Perms:    -
-     Throws:   NotImplemented
!     Throws:   BadParameter
                NoSuccess
      Notes:    - if the implementation cannot parse the given
                  url, a 'BadParameter' exception is thrown.
+               - if the implementation cannot perform proper
+                 escaping on the url, a 'BadParameter' 
+                 exception is thrown.
                - this constructor will never throw an
                  'IncorrectURL' exception, as the
                  interpretation of the URL is not part of the
                  functionality of this class.
                - the implementation MAY change the given
                  URL as long as that does not change the
                  resource the URL is pointing to.  For
                  example, an implementation may normalize the
                  path element of the URL.
 
 
    - DESTRUCTOR
      Purpose:  destroy a url
      Format:   DESTRUCTOR           (in  url obj);
      Inputs:   obj:                  the url to destroy
      InOuts:   -
      Outputs:  -
      PreCond:  -
      PostCond: -
      Perms:    -
      Throws:   -
      Notes:    -
 
 
    - set_string
      Purpose:  set a new url
      Format:   set_string           (in  string url = "");
      Inputs:   url:                  new url
      InOuts:   -
      Outputs:  -
      PreCond:  -
      PostCond: -
      Perms:    -
-     Throws:   NotImplemented
!     Throws:   BadParameter
      Notes:    - the method is semantically equivalent to
                  destroying the url, and re-creating it with
                  the given parameter.
                - the notes for the DESTRUCTOR and the 
                  CONSTRUCTOR apply.
 
 
    - get_string
      Purpose:  retrieve the url as string
      Format:   get_string           (out string url);
      Inputs:   -
      InOuts:   -
      Outputs:  url:                  string representing the url
      PreCond:  -
      PostCond: -
      Perms:    -
-     Throws:   NotImplemented
+     Throws:   -
      Notes:    - the URL may be empty, e.g. after creating the
                  instance with an empty url parameter.
+               - the returned string is unescaped.
 
+
+   - get_escaped
+     Purpose:  retrieve the url as string with escaped
+               characters
+     Format:   get_escaped          (out string url);
+     Inputs:   -
+     InOuts:   -
+     Outputs:  url:                  string representing the url
+     PreCond:  -
+     PostCond: -
+     Perms:    -
+     Throws:   -
+     Notes:    - the URL may be empty, e.g. after creating the
+                 instance with an empty url parameter.
+               - as get_string(), but all characters are 
+                 escaped where required.
+
 
    - set_*
      Purpose:  set an url element
      Format:   set_<element>       (in  string <element> = "");
                set_scheme          (in  string scheme    = "");
                set_host            (in  string host      = "");
                set_port            (in  int    port      = "");
                set_fragment        (in  string fragment  = "");
                set_path            (in  string path      = "");
                set_query           (in  string query     = "");
                set_userinfo        (in  string userinfo  = "");
      Inputs:   <element>:           new url <element>
      InOuts:   -
      Outputs:  -
      PreCond:  -
      PostCond: - the <element> part of the URL is updated.
      Perms:    -
-     Throws:   NotImplemented
!     Throws:   BadParameter
      Notes:    - these calls allow to update the various
                  elements of the url.  
                - the given <element> is parsed, and if it is
                  either not well formed (see RFC-3986), or the
                  implementation cannot handle it, a
                  'BadParameter' exception is thrown.
                - if the given <element> is empty, it is removed
                  from the URL.  If that results in an invalid
                  URL, a 'BadParameter' exception is thrown.
                - the implementation MAY change the given
                  elements as long as that does not change the
                  resource the URL is pointing to.  For
                  example, an implementation may normalize the
                  path element.
+               - the implementation MUST perform character 
+                 escaping for the given string.
 
 
    - get_*
      Purpose:  get an url element
      Format:   get_<element>       (out string <element>);
                get_scheme          (out string scheme   );
                get_host            (out string host     );
                get_port            (out int    port     );
                get_fragment        (out string fragment );
                get_path            (out string path     );
                get_query           (out string query    );
                get_userinfo        (out string userinfo );
      Inputs:   -
      InOuts:   -
      Outputs:  <element>:           the url <element>
      PreCond:  -
      PostCond: -
      Perms:    -
-     Throws:   NotImplemented
+     Throws:   -
      Notes:    - these calls allow to retrieve the various
                  elements of the url.  
                - the returned <element> is either empty, or 
                  guaranteed to be well formed (see RFC-3986).
+               - the returned string is unescaped.
+               - if the requested value is not known, or 
+                 unspecified, and empty string is returned, 
+                 or '-1' for get_port().
 
 
+   - translate
+     Purpose:  translate an URL to a new scheme
+     Format:   translate            (in  session s, 
+                                     in  string  scheme,
+                                     out url     url);
+     Inputs:   s:                    session for authorization/
+                                     authentication
+               scheme:               the new scheme to
+                                     translate into
+     InOuts:   -
+     Outputs:  url:                  string representation of
+                                     the translated url
+     PreCond:  -
+     PostCond: -
+     Perms:    -
+     Throws:   BadParameter
+               NoSuccess
+     Notes:    - the notes from section 'The URL Problem' apply.
+               - if the scheme is not supported,  a
+                 'BadParameter' exception is thrown.
+               - if the scheme is supported, but the url
+                 cannot be translated to the scheme, a
+                 'NoSuccess' exception is thrown.
+               - if the url can be translated, but cannot be
+                 handled with the new scheme anymore, no
+                 exception is thrown.  That can only be
+                 detected if the returned string is again used
+                 in a URL constructor, or with set_string().
+               - the call does not change the URL represented
+                 by the class instance itself, but the
+                 translation is only reflected by the returned
+                 url string.
+               - the given session is used for backend
+                 communication.

    - translate
      Purpose:  translate an URL to a new scheme
      Format:   translate            (in  string scheme,
                                      out url    url);
      Inputs:   scheme:               the new scheme to
                                      translate into
      InOuts:   -
      Outputs:  url:                  string representation of
                                      the translated url
      PreCond:  -
      PostCond: -
      Perms:    -
-     Throws:   NotImplemented
!     Throws:   BadParameter
                NoSuccess
+     Notes:    - all notes from the overloaded translate() 
+                 method apply.
+               - the default session is used for backend
+                 communication.
-     Notes:    - the notes from section 'The URL Problem' apply.
-               - if the scheme is not supported,  a
-                 'BadParameter' exception is thrown.
-               - if the scheme is supported, but the url
-                 cannot be translated to the scheme, a
-                 'NoSuccess' exception is thrown.
-               - if the url can be translated, but cannot be
-                 handled with the new scheme anymore, no
-                 exception is thrown.  That can only be
-                 detected if the returned string is again used
-                 in a URL constructor, or with set_string().
-               - the call does not change the URL represented
-                 by the class instance itself, but the
-                 translation is only reflected by the returned
-                 url string.
 \end{myspec}
 
 
 \subsubsection{Examples}
 
 \begin{mycode}
  // C++ URL examples
  
  int main (int argc, char ** argv)
  {
    if ( argc < 1 )
      return -1;
 
    std::string url_string = argv[1];
 
    try 
    {
      saga::url url (url_string);
 
      cout << "url      : " << url.get_string    () << endl;
      cout << "===================================" << endl;
      cout << "scheme   : " << url.get_scheme    () << endl;
      cout << "host     : " << url.get_host      () << endl;
      cout << "port     : " << url.get_port      () << endl;
      cout << "fragment : " << url.get_fragment  () << endl;
      cout << "path     : " << url.get_path      () << endl;
      cout << "query    : " << url.get_query     () << endl;
      cout << "userinfo : " << url.get_userinfo  () << endl;
      cout << "===================================" << endl;
 
      url.set_scheme    ("ftp");
      url.set_host      ("ftp.remote.net");
      url.set_port      (1234);
      url.set_fragment  ("");
      url.set_path      ("/tmp/data");
      url.set_query     ("");
      url.set_userinfo  ("ftp:anon");
 
      cout << "===================================" << endl;
      cout << "scheme   : " << url.get_scheme    () << endl;
      cout << "host     : " << url.get_host      () << endl;
      cout << "port     : " << url.get_port      () << endl;
      cout << "fragment : " << url.get_fragment  () << endl;
      cout << "path     : " << url.get_path      () << endl;
      cout << "query    : " << url.get_query     () << endl;
      cout << "userinfo : " << url.get_userinfo  () << endl;
      cout << "===================================" << endl;
      cout << "url      : " << url.get_string    () << endl;
    }
  }
 \end{mycode}
 
