 
  Nearly all of the SAGA use cases (except for the GridRPC use
  cases) had either explicit or implicit requirements for
  submitting jobs to grid resources, and most needed 
  also to monitor and control these submitted jobs.
 
  This section describes the SAGA API for submitting jobs to a
  grid resource, either in batch mode, or in an interactive
  mode. It also describes how to control these submitted jobs
  (e.g.  to |cancel()|, |suspend()|, or |signal()| a running
  job), and how to retrieve status information for both running
  and completed jobs.
  
  This API is also intended to incorporate the work of the
  DRMAA-WG~\cite{drmaa-wg}.  Much of this specification was
  taken directly from DRMAA specification~\cite{drmaa-spec},
  with many of the differences arising from an attempt to make
  the job API consistent with the overall SAGA API
  Look-\&-Feel\footnote{We expect that SAGA-API
  implementations may be implemented using DRMAA, or may produce
  JSDL documents to be passed to underlying scheduling
  systems.}.
  
  The API covers four classes: |saga::job_description|,
  |saga::job_service|, |saga::job| and |saga::job_self|.  The
  job description class is nothing more than a container for
  a well defined set of attributes which, using
  JSDL~\cite{jsdl-wg} based keys, defines the job to be started,
  and its runtime and resource requirements.  The job
  server represents a resource management endpoint which allows
  the starting and injection of jobs. 
  
  The job class itself is central to the API, and represents an
  application instance running under the management of
  a resource manager.  The \T{job\_self} class |IS-A| job, but
  additionally implements the steering interface.  The purpose
  of this class is to represent the current SAGA application,
  which allows for a number of use cases with
  applications which actively interact with the grid
  infrastructure, for example to provide steering capabilities,
  to migrate itself, or to set new job attributes.
  
  The job class inherits the |saga::task|
  class~\ref{ssec:tasks}, and uses its methods to |run()|,
  |wait()| for, and to |cancel()| jobs.  The inheritance feature
  also allows for the management of large numbers of jobs in
  task containers.  Additional methods provided by the
  |saga::job| class relate to the |Suspended| state (which is
  not available on tasks), and provide access to the job's
  standard I/O streams, and to more detailed status information.
  In this specification, the standard I/O streams are specified
  to have |opaque| types.  The SAGA language bindings MUST
  specify a native type for I/O streams.  That type SHOULD be
  the one used as the file descriptor to the POSIX |read()| call
  in that language.
 
  \subsubsection{Job State Model}
  
    The SAGA job state diagram is shown in
    Figure~\ref{fig:job_states}.  It is an extension of the
    |saga::task| state diagram (Figure~\ref{fig:task_states}),
    and extends the state diagram  with a |'Suspended'| state, which the
    job can enter/leave using the |suspend()/resume()| calls.  
 
    \myfig{job_states}{\footnotesize The SAGA job state model extends
    the SAGA task state model with a 'Suspended' state, and additional
    transitions (See Figure~\ref{fig:states} for a legend).}
 
    SAGA implementations need to map the native backend state
    model onto the SAGA state model.  The SAGA state model
    should be simple enough to allow a straight forward mapping
    in most cases.  For some applications, access to the native
    backend state model is useful.  For that reason, an
    additional metric named |'StateDetail'| allows to query the
    native job state.  That schema follows the current state
    model of the OGSA-BES specification~\cite{bes}, which also
    has a simplified top level state model, and allows for
    additional, backend specific state details.
 
    State details in SAGA SHOULD be formatted as follows:
 
    \shift |'<model>:<state>'|
 
    with valid models being "BES", "DRMAA", or other
    implementation specific models.  For example, a state detail
    for the BES state 'StagingIn' would be rendered as
    |'BES:StagingIn'|), and would be a substate of
    \T{Running}.  If no state details are available, the metric
    is still available, but it has always an empty string value.
 
 
  \subsubsection{Job Description Attributes}
 
    SAGA implementations MUST support the |Executable|
    attribute, as that is the only required attribute for a
    |job_description|.  An implementation MUST document which
    other attributes are supported, and which are not.
    In general, a |job_description| containing an unsupported
    attribute does \I{not} cause an error on job creation or
    submission, unless noted otherwise in the attribute
    description.
    
    Attributes marked as |'not supported by JSDL'| might
    disappear in future versions of the SAGA API -- all other
    attributes are likely to be kept, at least for backward
    compatibility.  The attribute description additionally
    mentions if the attributes are supported by DRMAA
    (see~\cite{drmaa-spec}) -- that is for information purposes
    only, and supposed to support implementations on top of
    DRMAA.
 
    Several \XSpell{metrics} on the |saga::job| class (the class
    implements the \T{saga::moni\-torable} interface) reflect
    attributes from the job description.  This redundancy is
    intentional, and aims at providing information about (a)
    attributes which may change at runtime, and (b) attributes
    for jobs for which no job description is available (e.g.
    |saga::job| instances obtained by calling |get.job()|.
 
    Although JSDL~\cite{jsdl-spec} and JSDL SPMD
    extension~\cite{jsdl-spmd} based attribute names are used
    for job description, the API supports no explicit
    representation of JSDL (i.e.  JSDL compliant XML).  XML is
    deemed to be too low level to be included into the SAGA API.
    Also, the JSDL parameter sweep
    extension~\cite{jsdl-paramsweep} is not used in SAGA at the
    moment, as bulk job submission, and related the creation of
    multiple related job descriptions, is performed on
    application level in SAGA, as described in
    Section~\ref{ssec:optimization}.
    
  \subsubsection{File Transfer Specifications}
  
  The syntax of a file transfer directive for the job
  description is modeled on the LSF syntax (LSF stands for
  \I{Load Sharing Facility}, a commercial job scheduler by
  Platform Computing), and has the general syntax:
 
    \shift |local_file operator remote_file|
 
    Both the |local_file| and the |remote_file| can be URLs. If
    they are not URLs, but full or relative pathnames, then the
    |local_file| is relative to the host where the submission is
    executed, and the |remote_file| is evaluated on the
    execution host of the job.
 
    The operator is one of the following four:
 
    \begin{tabular}{lp{0.75\textwidth}}
 
      \shift |'>'| &  copies the local file to the remote file
      before the job starts. Overwrites the remote file if it
      exists.\\
 
      \shift |'>>'| & copies the local file to the remote file
      before the job starts. Appends to the remote file if it
      exists.\\
 
      \shift |'<'| & copies the remote file to the local file
      after the job finishes. Overwrites the local file if it
      exists.\\
 
      \shift |'<<'| & copies the remote file to the local file
      after the job finishes. Appends to the local file if it
      exists.\\
 
    \end{tabular}
 
  
  \subsubsection{Command Line Specification}
 
    The |run_job()| method of the |saga::job_service| class
    accepts a string parameter which constitutes a command line
    to be executed on a remote resource.  The parsing of that
    command lines follows the following rules:
 
    \begin{itemize}
      \item Elements are delimited by white space, which 
            is either a space or a tab.
      \item A string surrounded by double quotation marks is 
            interpreted as a single element, regardless of white 
            space contained within. A quoted string can be embedded 
            in an element. 
      \item A double quotation mark preceded by a backslash, |\"|, is 
            interpreted as a literal double quotation mark (|"|).
      \item Backslashes are interpreted literally, unless they 
            immediately precede a double quotation mark.
      \item The first element is used as executable name;
            all other elements are treated as job arguments.
    \end{itemize}
 
 
  
  \subsubsection{Job Identifiers}
  
    The \T{JobID} is treated as an opaque string in the
    SAGA API.  However, for the sake of interoperability of
    different SAGA implementations, and for potential extended
    use of the \T{JobID} information, the |JobID|
    SHOULD be implemented as:
 
      \shift |'[backend url]-[native id]'|
 
    For example, a job submitted to the host |remote.host.net|
    via |ssh| (whose daemon runs on port |22|), and having the
    POSIX PID |1234|, should get the job id:
 
      \shift |'[ssh://remote.host.net:22/]-[1234]'|
 
    The implementation MAY free the resources used for the job,
    and hence MAY invalidate a \T{JobID}, after a
    successful wait on the job, or after the application
    received the job status information, and job status
    details if available, at least once.
 
    A JobID may be unknown until the job enters the
    \T{Running} state, as the backend will often not assign IDs to
    jobs which are not yet running.  In such cases, the value of
    the \T{JobID} attribute SHOULD be empty.  The job MUST,
    however, retain its JobID after it enters in a final state.

    \XAdd[5]{The job attribute \T{"ServiceURL"} exposes the URL of the
    \T{job::service} instance which spawned the job.  Any new
    \T{job::service} instance created with that URL SHOULD be able to
    handle the job's jobid, and in particular SHOULD be able to
    reconnect to that job.  The tuple \T{JobID, ServiceURL} thus
    allows to create both the job service and the job instances for
    any SAGA job.}
      
 
 \subsubsection{Specification}
 
 \begin{myspec}
  package saga.job
  {
    enum state
    {
      New        =  1,  // same as in saga::task::state
      Running    =  2,  // same as in saga::task::state
      Done       =  3,  // same as in saga::task::state
      Canceled   =  4,  // same as in saga::task::state
      Failed     =  5,  // same as in saga::task::state
      Suspended  =  6
    }
 
 
    class job_description : implements   saga::object
                            implements   saga::attributes
                         // from object: saga::error_handler
    {
      CONSTRUCTOR          (out job_description obj);
      DESTRUCTOR           (in  job_description obj);
 
      // Attributes:
      //
      //   name:  Executable
      //   desc:  command to execute. 
      //   type:  String
      //   mode:  ReadWrite
      //   value: ''
      //   notes: - this is the only required attribute.  
!     //          - can be a full pathname, a pathname 
      //            relative to the 'WorkingDirectory' as 
!     //            evaluated on the execution host, or
+     //            a executable name to be searched in the
+     //            target host's PATH environment (if 
+     //            available).
      //          - available in JSDL, DRMAA
      //          - semantics as defined in JSDL
      //
      //   name:  Arguments
      //   desc:  positional parameters for the command. 
      //   mode:  ReadWrite, optional
      //   type:  Vector String
      //   value: -
      //   notes: - available in JSDL, DRMAA
      //            semantics as specified by JSDL
      // 
      //   name:  SPMDVariation
      //   desc:  SPMD job type and startup mechanism
      //   mode:  ReadWrite, optional
      //   type:  String
      //   value: -
      //   notes: - as defined in the SPMD extension of JSDL
      //   notes: - available in JSDL, SPMD extension
      //          - semantics as defined in JSDL
      //          - the SPMD JSDL extension defines the value
      //            to be an URI.  For simplicity, SAGA allows
      //            the following strings, which map into the 
      //            respective URIs: MPI, GridMPI, IntelMPI,
      //            LAM-MPI, MPICH1, MPICH2, MPICH-GM, MPICH-MX,
      //            MVAPICH, MVAPICH2, OpenMP, POE, PVM, None
      //          - the value '' (no value, default) indicates
      //            that the application is not a SPMD 
      //            application.
      //          - as JSDL, SAGA allows other arbitrary values.
      //            The implementation must clearly document
      //            which values are supported.
      //
      //   name:  TotalCPUCount
      //   desc:  total number of cpus requested for this job
      //   mode:  ReadWrite, optional
      //   type:  Int
      //   value: '1'
      //   notes: - available in JSDL, DRMAA
      //          - semantics as defined in JSDL
      // 
      //   name:  NumberOfProcesses
      //   desc:  total number of processes to be started
      //   mode:  ReadWrite, optional
      //   type:  Int
      //   value: '1'
      //   notes: - available in JSDL, SPMD extension
      //          - semantics as defined in JSDL
      // 
      //   name:  ProcessesPerHost
      //   desc:  number of processes to be started per host
      //   mode:  ReadWrite, optional
      //   type:  Int
      //   value: '1'
      //   notes: - available in JSDL, SPMD extension
      //          - semantics as defined in JSDL
      // 
      //   name:  ThreadsPerProcess
      //   desc:  number of threads to start per process
      //   mode:  ReadWrite, optional
      //   type:  Int
      //   value: '1'
      //   notes: - available in JSDL, SPMD extension
      //          - semantics as defined in JSDL
      // 
      //   name:  Environment
      //   desc:  set of environment variables for the job
      //   mode:  ReadWrite, optional
      //   type:  Vector String
      //   value: -
      //   notes: - exported into the job environment
      //          - format: 'key=value'
      //          - available in JSDL, DRMAA
      //          - semantics as specified by JSDL
      // 
      //   name:  WorkingDirectory
      //   desc:  working directory for the job
      //   mode:  ReadWrite, optional
      //   type:  String
      //   value: '.'
+     //   notes: - gets created if it does not exist
      //          - available in JSDL, DRMAA
      //          - semantics as specified by JSDL
      // 
      //   name:  Interactive
      //   desc:  run the job in interactive mode
      //   mode:  ReadWrite, optional
      //   type:  Bool
      //   value: 'False'
      //   notes: - this implies that stdio streams will stay 
      //            connected to the submitter after job 
      //            submission, and during job execution. 
      //          - if an implementation cannot handle
      //            interactive jobs, and this attribute is
      //            present, and 'True', the job creation MUST
      //            throw an 'IncorrectParameter' error with a
      //            descriptive error message.
      //          - not supported by JSDL, DRMAA
      // 
      //   name:  Input
      //   desc:  pathname of the standard input file
      //   mode:  ReadWrite, optional
      //   type:  String
      //   value: -
      //   notes: - available in JSDL, DRMAA
      //          - semantics as specified by JSDL
      //          - will not be used if 'Interactive' is 'True'
      // 
      //   name:  Output
      //   desc:  pathname of the standard output file
      //   mode:  ReadWrite, optional
      //   type:  String
      //   value: -
      //   notes: - available in JSDL, DRMAA
      //          - semantics as specified by JSDL
      //          - will not be used if 'Interactive' is 'True'
      // 
      //   name:  Error
      //   desc:  pathname of the standard error file
      //   mode:  ReadWrite, optional
      //   type:  String
      //   value: -
      //   notes: - available in JSDL, DRMAA
      //          - semantics as specified by JSDL
      //          - will not be used if 'Interactive' is 'True'
      // 
      //   name:  FileTransfer
      //   desc:  a list of file transfer directives
      //   mode:  ReadWrite, optional
      //   type:  Vector String
      //   value: -
      //   notes: - translates into jsdl:DataStaging
      //          - used to specify pre- and post-staging
      //          - staging is part of the 'Running' state
      //          - syntax similar to LSF (see earlier notes)
      //          - available in JSDL, DRMAA
      //          - semantics as specified in JSDL 
      // 
      //   name:  Cleanup
      //   desc:  defines if output files get removed after the 
      //          job finishes
      //   mode:  ReadWrite, optional
      //   type:  String
      //   value: 'Default'
      //   notes: - can have the Values 'True', 'False', and 
      //            'Default'
      //          - On 'False', output files MUST be kept 
      //            after job the finishes
      //          - On 'True', output files MUST be deleted
      //            after job the finishes
      //          - On 'Default', the behavior is defined by
      //            the implementation or the backend.
      //          - translates into 'DeleteOnTermination' elements
      //            in JSDL
      // 
      //   name:  JobStartTime
      //   desc:  time at which a job should be scheduled
      //   mode:  ReadWrite, optional
      //   type:  Int
      //   value: -
      //   notes: - Could be viewed as a desired job start 
      //            time, but that is up to the resource 
      //            manager. 
      //          - format: number of seconds since epoch
      //          - available in DRMAA
      //          - not supported by JSDL
      // 
+     //   name:  WallTimeLimit
+     //   desc:  hard limit for the total job runtime.
+     //   mode:  ReadWrite, optional
+     //   type:  Int
+     //   value: -
+     //   notes: - intended to provide hints to the scheduler. 
+     //          - available in JSDL, DRMAA
+     //          - semantics as defined in JSDL
+     // 
      //   name:  TotalCPUTime
      //   desc:  estimate total number of CPU seconds which
      //          the job will require. 
      //   mode:  ReadWrite, optional
      //   type:  Int
      //   value: -
      //   notes: - intended to provide hints to the scheduler. 
-     //            scheduling policies.
      //          - available in JSDL, DRMAA
      //          - semantics as defined in JSDL
      // 
      //   name:  TotalPhysicalMemory
      //   desc:  Estimated amount of memory the job requires
      //   mode:  ReadWrite, optional
      //   type:  Float
      //   value: -
      //   notes: - unit is in MegaByte
      //          - memory usage of the job is aggregated 
      //            across all processes of the job
      //          - available in JSDL
      //          - semantics as defined by JSDL
      // 
      //   name:  CPUArchitecture
      //   desc:  compatible processor for job submission
      //   mode:  ReadWrite, optional
!     //   type:  String
      //   value: -
      //   notes: - allowed values as specified in JSDL
      //          - available in JSDL
      //          - semantics as defined by JSDL
      // 
      //   name:  OperatingSystemType
      //   desc:  compatible operating system for job submission
      //   mode:  ReadWrite, optional
!     //   type:  String
      //   value: -
      //   notes: - allowed values as specified in JSDL
      //          - available in JSDL
      //          - semantics as defined by JSDL
      // 
      //   name:  CandidateHosts
      //   desc:  list of host names which are to be considered
      //          by the resource manager as candidate targets
      //   mode:  ReadWrite, optional
      //   type:  Vector String
      //   value: -
      //   notes: - available in JSDL
      //          - semantics as defined by JSDL
      //
      //   name:  Queue
      //   desc:  name of a queue to place the job into
      //   mode:  ReadWrite, optional
      //   type:  String
      //   value: -
      //   notes: - While SAGA itself does not define the 
      //            semantics of a "queue", many backend systems
      //            can make use of this attribute. 
      //          - not supported by JSDL
      // 
+     //   name:  JobProject
+     //   desc:  name of a account or project name
+     //   mode:  ReadWrite, optional
+     //   type:  String
+     //   value: -
+     //   notes: - While SAGA itself does not define the 
+     //            semantics of an "account" or "project", 
+     //            many backend systems can make use of
+     //            this attribute for the purpose of 
+     //            accounting.
+     //          - available in JSDL
+     //          - semantics as defined by JSDL
+     // 
      //   name:  JobContact
      //   desc:  set of endpoints describing where to report 
      //          job state transitions.
      //   mode:  ReadWrite, optional
      //   type:  Vector String
      //   value: -
      //   notes: - format: URI (e.g. fax:+123456789, 
      //            sms:+123456789, mailto:joe@doe.net). 
      //          - available in DRMAA
      //          - not supported by JSDL
    }
 
 
    class job_service : implements   saga::object
                        implements   saga::async
                     // from object  saga::error_handler
    {
      CONSTRUCTOR          (in  session         s,
                            in  url             rm = "",
                            out job_service     obj);
      DESTRUCTOR           (in  job_service     obj);
 
      create_job           (in  job_description jd,
                            out job             job);
      run_job              (in  string          commandline,
                            in  string          host = "",
                            out job             job, 
                            out opaque          stdin,
                            out opaque          stdout,
                            out opaque          stderr);
      list                 (out array<string>   job_ids);
      get_job              (in  string          job_id,
                            out job             job);
      get_self             (out job_self        job);
    }
 
 
    class job : extends      saga::task
                implements   saga::async
                implements   saga::attributes
                implements   saga::permissions
             // from task    saga::object
             // from task    saga::monitorable
             // from object  saga::error_handler
    {
      // no CONSTRUCTOR
      DESTRUCTOR           (in  job             obj);
 
      // job inspection
      get_job_description  (out job_description jd);
      get_stdin            (out opaque          stdin);
      get_stdout           (out opaque          stdout);
      get_stderr           (out opaque          stderr);
 
      // job management
      suspend              (void);
      resume               (void);
      checkpoint           (void);
      migrate              (in job_description   jd);
      signal               (in int               signum);
 
      // Attributes:
      //
      //   name:  JobID
      //   desc:  SAGA representation of the job identifier
      //   mode:  ReadOnly
      //   type:  String
      //   value: -
      //   notes: - format: as described earlier
      // 
+     //   name:  ServiceURL
+     //   desc:  URL representation of the job::service instance
+     //          managing this job
+     //   mode:  ReadOnly
+     //   type:  String
+     //   value: -
+     //   notes: - can be used for a job::service CONSTRUCTOR.
+     // 
      //   name:  ExecutionHosts
      //   desc:  list of host names or IP addresses allocated 
      //          to run this job
      //   mode:  ReadOnly, optional
      //   type:  Vector String
      //   value: -
      //   notes: - 
      // 
      //   name:  Created
      //   desc:  time stamp of the job creation in the 
      //          resource manager
      //   mode:  ReadOnly, optional
      //   type:  Time
      //   value: -
      //   notes: - can be interpreted as submission time
      // 
      //   name:  Started
      //   desc:  time stamp indicating when the job started 
      //          running
      //   mode:  ReadOnly, optional
      //   type:  Time
      //   value: -
      // 
      //   name:  Finished
      //   desc:  time stamp indicating when the job completed
      //   mode:  ReadOnly, optional
      //   type:  Time
      //   value: -
      // 
      //   name:  WorkingDirectory
      //   desc:  working directory on the execution host 
      //   mode:  ReadOnly, optional
      //   type:  String
      //   value: -
      //   notes: - can be used to determine the location of 
      //            files staged using relative file paths
      // 
      //   name:  ExitCode
      //   desc:  process exit code as collected by the wait(2)
      //          series of system calls. 
      //   mode:  ReadOnly, optional
      //   type:  Int
      //   value: -
      //   notes: - exit code is collected from the process 
      //            which was started from the 'Executable' 
      //            attribute of the job_description object.  
      //          - only available in final states, if at all
      // 
      //   name:  Termsig
      //   desc:  signal number which caused the job to exit
      //   mode:  ReadOnly, optional
      //   type:  Int
      //   value: -
      //   notes: - only available in final states, if at all
 
 
      // Metrics:
      //   name:  job.state
      //   desc:  fires on state changes of the job, and has 
      //          the literal value of the job state enum.
      //   mode:  ReadOnly
      //   unit:  1
      //   type:  Enum
      //   value: New
      //   notes: - the state metric is inherited from
      //            saga::task, but has a different set
      //            of possible values
      //          - see description of job states above
      // 
      //   name:  job.state_detail
      //   desc:  fires as a job changes its state detail
      //   mode:  ReadOnly, optional
      //   unit:  1
      //   type:  String
      //   value: -
      // 
      //   name:  job.signal
      //   desc:  fires as a job receives a signal, and has a
      //          value indicating the signal number
      //   mode:  ReadOnly, optional
      //   unit:  1
      //   type:  Int
      //   value: -
      //   notes: - no guarantees are made that any or all
      //            signals can be notified by this metric
      // 
      //   name:  job.cpu_time
      //   desc:  number of CPU seconds consumed by the job
      //   mode:  ReadOnly, optional
      //   unit:  seconds
      //   type:  Int
      //   value: -
      //   notes: - aggregated across all processes/threads
      // 
      //   name:  job.memory_use
      //   desc:  current aggregate memory usage
      //   mode:  ReadOnly, optional
      //   unit:  megabyte
      //   type:  Float
      //   value: 0.0
      //   notes: - metric becomes 'Final' after job 
      //            completion, and then shows the memory 
      //            high water mark
      // 
      //   name:  job.vmemory_use
      //   desc:  current aggregate virtual memory usage
      //   mode:  ReadOnly, optional
      //   unit:  megabyte
      //   type:  Float
      //   value: 0.0
      //   notes: - metric becomes 'Final' after job 
      //            completion, and then shows the virtual 
      //            memory high water mark
      // 
      //   name:  job.performance
      //   desc:  current performance
      //   mode:  ReadOnly, optional
      //   unit:  FLOPS
      //   type:  Float
      //   value: 0.0
      //   notes: - metric becomes 'Final' after job 
      //            completion, and then shows the performance
      //            high water mark
    }
 
 
    class job_self : extends     saga::job
                     implements  saga::steerable
                  // from job    saga::async
                  // from job    saga::attributes
                  // from job    saga::task
                  // from job    saga::object
                  // from job    saga::monitorable
                  // from job    saga::permissions
                  // from job    saga::error_handler
    {
      // no CONSTRUCTOR
      DESTRUCTOR           (in  job_self        obj);
    }
  }
 \end{myspec}
 
 
 \subsubsection{Specification Details}
 
  \subsubsection*{Enum \T{state}}
 
    The \T{state} is equivalent to the inherited
    \T{saga::task::state}, but adds the \T{Suspended} state:
 
    |Suspended|\\[1.5mm]
    \begin{tabular}{cp{110mm}}
       ~~ & This state identifies a job  instance which 
            has been suspended.  This state corresponds to the 
            BES state \T{'Suspend'}.
    \end{tabular}
 
 
  \subsubsection*{Class \T{job\_description}}
 
    This object encapsulates all the attributes which define a
    job to be run. It has no methods of its own, but implements
    the |saga::attributes| interface in order to provide access
    to the job properties, which are expressed as JSDL keywords. 
    
    The only required attribute in order to perform a valid job
    submission is the |Executable|. Given the |Executable|, a
    job can be instantiated in many existing backend
    systems without any further specification.
 
    There should be significant overlap between the
    attributes defined within SAGA and within the JSDL
    specification. This list, however, will not be complete in
    cases where the JSDL was deemed more complicated than was
    required for a simple API (e.g. the notion of JSDL
    profiles), or where an attribute was needed to
    interact with a scheduler, which was not within the stated
    scope of the JSDL working group (e.g.  |Queue|, which is
    considered a \I{site attribute}, and thus not relevant to
    the pure description of a job).
 
 
 \begin{myspec}
    - CONSTRUCTOR
      Purpose:  create the object
      Format:   CONSTRUCTOR       (out job_description obj)
      Inputs:   -
      InOuts:   -
      Outputs:  obj:               the newly created object
      PreCond:  -
      PostCond: -
      Perms:    -
      Throws:   NotImplemented
                NoSuccess
      Notes:    - a job_description is not associated with a
                  session, but can be used for job services
                  from different sessions.
 
 
    - DESTRUCTOR
      Purpose:  destroy the object
      Format:   DESTRUCTOR        (in  job_description obj)
      Inputs:   obj:               the object to destroy
      InOuts:   -
      Outputs:  -
      PreCond:  -
      PostCond: -
      Perms:    -
      Throws:   -
      Notes:    -
 \end{myspec}
 
 
  \subsubsection*{Class \T{job\_service}}
 
    The |job_service| represents a resource management backend,
    and as such allows the creation and submision of jobs, and
    to discover jobs.  The job management methods are on the job
    object itself -- this probably implies that implementations
    need to internally track what resource manager (or
    |job_service| instance) created the job.
 
 \begin{myspec}
    - CONSTRUCTOR
      Purpose:  create the object
      Format:   CONSTRUCTOR       (in  session       s,
                                   in  url           rm = "",
                                   out job_service   obj)
      Inputs:   s:                 session to associate with
                                   the object
                rm:                contact url for resource 
                                   manager
      InOuts:   -
      Outputs:  obj:               the newly created object
      PreCond:  -
      PostCond: -
      Perms:    -
      Throws:   NotImplemented
+               BadParameter
                IncorrectURL
                PermissionDenied
                AuthorizationFailed
                AuthenticationFailed
                Timeout
                NoSuccess
      Notes:    - 'rm' defaults to an empty string - in that
                  case, the implementation must perform a
                  resource discovery, or fall back to a fixed 
                  value, or find a valid  rm contact in any 
                  other way.  If that is not possible, a 
                  'BadParameter' exception MUST be thrown, and 
                  MUST indicate that a rm contact string is 
                  needed.  The expected behavior MUST be 
                  documented (i.e. if a default is available).
                - if the rm identified by the rm URL cannot be
                  contacted (i.e. does not exist), a
                  'BadParameter' exception is thrown.
 
 
    - DESTRUCTOR
      Purpose:  destroy the object
      Format:   DESTRUCTOR        (in  job_service   obj)
      Inputs:   obj:               the object to destroy
      InOuts:   -
      Outputs:  -
      PreCond:  -
      PostCond: - jobs created by that job_service instance 
                  are not affected by the destruction, and are
                  in particular not canceled.
      Perms:    -
      Throws:   - 
      Notes:    - 
 
 
    - create_job
      Purpose:  create a job instance
      Format:   create_job        (in  job_description jd,
                                   out job             job);
      Inputs:   jd:                description of job to be
                                   submitted
      InOuts:   -
      Outputs:  job:               a job object representing
                                   the submitted job instance
      PreCond:  - jd has an 'Executable' attribute.
      PostCond: - job is in 'New' state
                - jd is deep copied (no state is shared
                  after method invocation)
                - 'Owner' of the job is the id of the context
                  used for creating the job.
      Perms:    -
      Throws:   NotImplemented
                BadParameter
                PermissionDenied
                AuthorizationFailed
                AuthenticationFailed
                Timeout
                NoSuccess
      Notes:    - calling run() on the job will submit it to
                  the resource, and advance its state.
                - if the job description does not have a valid
                  'Executable' attribute, a 'BadParameter' 
                  exception is thrown.
                - if the job description contains values which
                  are outside of the allowed range, or cannot be
                  parsed, or are otherwise invalid and not
                  usable for creating a job instance, a
                  'BadParameter' exception is thrown, which MUST
                  indicate which attribute(s) caused this
                  exception, and why.
               
               
    - run_job  
      Purpose:  Run a command synchronously.
      Format:   run_job           (in  string  commandline,
                                   in  string  host = "",
                                   out job     job, 
                                   out opaque  stdin,
                                   out opaque  stdout,
                                   out opaque  stderr);
      Inputs:   commandline:       the command and arguments
                                   to be run
                host:              hostname to be used by rm for
                                   submission
      InOuts:   -
      Outputs:  stdin:             IO handle for the running
                                   job's standard input stream
                stdout:            IO handle for the running
                                   job's standard output
                stderr:            IO handle for the running
                                   job's standard error
                job:               a job object representing
                                   the submitted job instance
      PreCond:  -
      PostCond: - job is in 'Running', 'Done' or 'Failed' state.
                - 'Owner' of the job is the id of the context
                  used for creating the job.
      Perms:    - 
      Throws:   NotImplemented
                BadParameter
                PermissionDenied
                AuthorizationFailed
                AuthenticationFailed
                Timeout
                NoSuccess
      Notes:    - This is a convenience routine built on the
                  create_job method, and is intended to simplify
                  the steps of creating a job_description,
                  creating and running the job, and then
                  querying the standard I/O streams.
                - the I/O handles have to be passed to the call
                  as references, in most languages, as calls
                  often allow only one return value (Perl or
                  python being notable exceptions).  If these
                  parameters are omitted, the job is to be
                  started non-interactively, and the output I/O
                  streams may be discarded.
                - the job is guaranteed to run on the given
                  host, or not at all.
                - the method is exactly equivalent to the
                  sequence of (1) creation of a job_description
                  with 'Executable' set to the values from the
                  commandline, 'Interactive' set if I/O is
                  requested, 'CandidateHost' set to host; (2)
                  create_job() with that description; (3)
                  calling run() on that job.  This method can
                  throw any of the exceptions which can occur in
                  this sequence, with the semantics defined in
                  the detailed description of the methods used
                  in this sequence.  No other exception are to
                  be expected.
                - if 'host' is an empty string (the default),
                  the implementation MUST choose an arbitrary
                  host for execution.
                - stdin, stdout and stderr are guaranteed to
                  contain/provide the complete standard I/O
                  streams, beginning at the start of the remote
                  process.
 
 
    - list
      Purpose:  Get a list of jobs which are currently known by
                the resource manager.
      Format:   list              (out array<string>   job_ids);
      Inputs:   -
      InOuts:   -
      Outputs:  job_ids:           an array of job identifiers
      PreCond:  -
      PostCond: -
      Perms:    Query on jobs identified by the returned ids
      Throws:   NotImplemented
                PermissionDenied
                AuthorizationFailed
                AuthenticationFailed
                Timeout
                NoSuccess
      Notes:    - which jobs are viewable by the calling user 
                  context, and how long a resource manager keeps 
                  job information, are both implementation 
                  dependent.
                - a returned job_id may translate into a job 
                  (via get_job()) which is not controllable by 
                  the requesting application (e.g. it could 
                  cause an 'AuthorizationFailed' exception).
 
 
    - get_job
      Purpose:  Given a job identifier, this method returns a
                job object representing this job.
      Format:   get_job             (in  string job_id,
                                     out job    job)
      Inputs:   job_id:              job identifier as returned
                                     by the resource manager
      InOuts:   -
      Outputs:  job:                 a job object representing
                                     the job identified by
                                     job_id
      PreCond:  - job identified by job_id is managed by the
                  job_service.
      PostCond: - 
      Perms:    Query on the job.
      Throws:   NotImplemented
                BadParameter
                DoesNotExist
                PermissionDenied
                AuthorizationFailed
                AuthenticationFailed
                Timeout
                NoSuccess
      Notes:    - in general, only a job_service representing the
                  resource manager which submitted the job may be
                  able to handle the job_id, and to identify the
                  job -- however, other job_services may succeed
                  as well.
                - if the resource manager can handle the job_id,
                  but the referenced job is not alive, a 
                  'DoesNotExist' exception is thrown.
                - if the resource manager cannot parse the job_id
                  at all, a 'BadParameter' exception is thrown.
 
 
    - get_self
      Purpose:  This method returns a job object representing
                _this_ job, i.e. the calling application.
      Format:   get_self            (out job_self  self)
      Inputs:   -
      InOuts:   -
      Outputs:  self:                a job_self object 
                                     representing _this_ job.
      PreCond:  - the application is managed by the job_service.
      PostCond: - job_self is, by definition, in 'Running' 
                  state.
      Perms:    Query on the job.
      Throws:   NotImplemented
                PermissionDenied
                AuthorizationFailed
                AuthenticationFailed
                Timeout
                NoSuccess
      Notes:    - in general, only a job_service representing the
                  resource manager which started the application
                  which now calls get_self() can successfully
                  return a job_self instance.  However, other
                  job_services may succeed as well.
                - if a job_service cannot handle the calling job
                  as a job_self instance, a 'NoSuccess' exception
                  is thrown, with a descriptive error message.
 \end{myspec}
 
 
  \subsubsection*{Class \T{job}}
 
    The |job| provides the manageability interface to a job
    instance submitted to a resource manager. There are two
    general types of methods: those for retrieving job state and
    information, and those for manipulating the job.  The
    methods intended to manipulate jobs cannot make any
    guarantees about \I{how} the resource manager will
    affect an action to be taken.  The API
    implementation is designed to be agnostic of the backend
    implementation, such that any backend could be
    implemented to perform an action. For example, the
    checkpoint routine might cause an application level
    checkpoint, or might use the services of GridCPR.
 
    Job implements the |saga::attributes| interface.  If not noted
    otherwise, none of these attributes is available before the
    job is running, and none is guaranteed to have a non-empty
    value while the job is running or after the job finishes.
 
    Job also implements the monitorable interface, and thus
    allows monitoring and notification for changes of
    runtime attributes.
 
 
 \begin{myspec}
    - DESTRUCTOR
      Purpose:  destroy the object
      Format:   DESTRUCTOR        (in  job  obj)
      Inputs:   obj:               the object to destroy
      InOuts:   -
      Outputs:  -
      PreCond:  -
      PostCond: -
      Perms:    -
      Throws:   -
      Notes:    - the object destruction does not imply a 
                  call to cancel() for the job instance.
 
 
    - get_job_description
      Purpose:  Retrieve the job_description which was used to
                submit this job instance.
      Format:   get_job_description (out job_description jd);
      Inputs:   -
      InOuts:   -
      Outputs:  jd:                  a job_description object
      PreCond:  -
      PostCond: - jd is deep copied (no state is shared
                  after method invocation)
      Perms:    Query
      Throws:   NotImplemented
                DoesNotExist
                PermissionDenied
                AuthorizationFailed
                AuthenticationFailed
                Timeout
                NoSuccess
      Notes:    - There are cases when the job_description is not
                  available.  This may include cases when
                  the job was not submitted through
                  SAGA and get_job() was used to retrieve the
                  job, or when this state information has been
                  lost (e.g. the client application restarts and
                  the particular SAGA implementation did not
                  persist the information).  In that case, a
                  'DoesNotExist' exception is thrown, with a
                  descriptive error message.
 
 
    - get_stdin 
      Purpose:  retrieve input stream for a job.
      Format:   get_stdin          (out opaque stdin)
      Inputs:   -
      InOuts:   -
      Outputs:  stdin:              standard input stream for
                                    the job
      PreCond:  - the job is interactive.
      PostCond: - the jobs standard input stream is available 
                  at stdin.
      Perms:    Write (application can write to the jobs stdin).
      Throws:   NotImplemented
                BadParameter
                DoesNotExist
                IncorrectState
                PermissionDenied
                AuthorizationFailed
                AuthenticationFailed
                Timeout
                NoSuccess
      Notes:    - if the preconditions are met, but the standard
                  input stream is not available for some 
                  reason, a 'DoesNotExist' exception is thrown.
                - the stream MUST be valid until the job reaches
                  a final state.  If it is, for some reason,
                  disconnected earlier, a language typical error
                  message is thrown (e.g. EBADF could be 
                  returned on writes on that stream in C).
                - if the job is not interactive, e.g. it was
                  submitted with the 'Interactive' attribute set
                  to 'False', an 'IncorrectState' exception is
                  thrown.
                - if the job is not in 'New' state, it is not
                  guaranteed that the job did not receive other
                  data on its standard input stream before.
 
 
    - get_stdout
      Purpose:  retrieve output stream of job
      Format:   get_stdout         (out opaque stdout)
      Inputs:   -
      InOuts:   -
      Outputs:  stdout:             standard output stream for
                                    the job
      PreCond:  - the job is interactive.
      PostCond: - the jobs standard output stream is available 
                  from stdout.
      Perms:    Read (application can read the jobs stdout).
      Throws:   NotImplemented
                BadParameter
                DoesNotExist
                IncorrectState
                PermissionDenied
                AuthorizationFailed
                AuthenticationFailed
                Timeout
                NoSuccess
      Notes:    - if the preconditions are met, but the standard
                  output stream is not available for some 
                  reason, a 'DoesNotExist' exception is thrown.
                - the stream MUST be valid until the job reaches
                  a final state.  If it is, for some reason,
                  disconnected earlier, a language typical error
                  message is thrown (e.g. EBADF could be 
                  returned on reads on that stream in C).
                - if the job is not interactive, e.g. it was
                  submitted with the 'Interactive' attribute set
                  to 'False', an 'IncorrectState' exception is
                  thrown.
                - if the job is not in 'New' state, it is not
                  guaranteed that the job did write data on
                  its standard output stream before, which are
                  then not returned on the returned stream.
 
 
    - get_stderr
      Purpose:  retrieve error stream of job
      Format:   get_stderr         (out opaque stderr)
      Inputs:   -
      InOuts:   -
      Outputs:  stderr:             standard error stream for
                                    the job
      PreCond:  - the job is interactive.
      PostCond: - the jobs standard error stream is available 
                  from stderr.
      Perms:    Read (application can read the jobs stderr).
      Throws:   NotImplemented
                BadParameter
                DoesNotExist
                IncorrectState
                PermissionDenied
                AuthorizationFailed
                AuthenticationFailed
                Timeout
                NoSuccess
      Notes:    - if the preconditions are met, but the standard
                  error stream is not available for some 
                  reason, a 'DoesNotExist' exception is thrown.
                - the stream MUST be valid until the job reaches
                  a final state.  If it is, for some reason,
                  disconnected earlier, a language typical error
                  message is thrown (e.g. EBADF could be 
                  returned on reads on that stream in C).
                - if the job is not interactive, e.g. it was
                  submitted with the 'Interactive' attribute set
                  to 'False', an 'IncorrectState' exception is
                  thrown.
                - if the job is not in 'New' state, it is not
                  guaranteed that the job did write data on
                  its standard error stream before, which are
                  then not returned on the returned stream.
 
 
    Job Management Methods:
    -----------------------
 
    - suspend
      Purpose:  Ask the resource manager to perform a suspend
                operation on the running job.
      Format:   suspend       (void);
      Inputs:   -
      InOuts:   -
      Outputs:  -
      PreCond:  - the job is in 'Running' state.
      PostCond: - the job is in 'Suspended' state.
      Perms:    Exec (job can be controlled).
      Throws:   NotImplemented
                IncorrectState
                PermissionDenied
                AuthorizationFailed
                AuthenticationFailed
                Timeout
                NoSuccess
      Notes:    - if the job is not in 'Running' state, an
                  'IncorrectState' exception is thrown.
 
 
    - resume
      Purpose:  Ask the resource manager to perform a resume
                operation on a suspended job.
      Format:   resume             (void);
      Inputs:   -
      InOuts:   -
      Outputs:  -
      PreCond:  - the job is in 'Suspended' state.
      PostCond: - the job is in 'Running' state.
      Perms:    Exec (job can be controlled).
      Throws:   NotImplemented
                IncorrectState
                PermissionDenied
                AuthorizationFailed
                AuthenticationFailed
                Timeout
                NoSuccess
      Notes:    - if the job is not in 'Suspended' state, an
                  'IncorrectState' exception is thrown.
 
 
    - checkpoint
      Purpose:  Ask the resource manager to initiate a checkpoint
                operation on a running job.
      Format:   checkpoint         (void);
      Inputs:   -
      InOuts:   -
      Outputs:  -
      PreCond:  - the job is in 'Running' state.
      PostCond: - the job is in 'Running' state.
                - the job was checkpointed.
      Perms:    Exec (job can be controlled).
      Throws:   NotImplemented
                IncorrectState
                PermissionDenied
                AuthorizationFailed
                AuthenticationFailed
                Timeout
                NoSuccess
      Notes:    - The semantics of checkpoint(), and the actions
                  taken to initiate a checkpoint, are resource
                  manager specific.  In particular, the
                  implementation or backend can trigger either 
                  a system level or an application level
                - if the job is not in 'Running' state, 
                  an 'IncorrectState' exception is thrown.
 
 
    - migrate
      Purpose:  Ask the resource manager to migrate a job.
      Format:   migrate            (in job_description jd);
      Inputs:   jd:                 new job parameters to apply
                                    when the job is migrated
      InOuts:   -
      Outputs:  -
      PreCond:  - the job is in 'Running' or 'Suspended' state.
      PostCond: - the job keeps its state.
                - jd is deep copied (no state is shared
                  after method invocation)
                - the job reflects the attributes specified in
                  the job_description.
      Perms:    Exec (job can be controlled).
      Throws:   NotImplemented
                BadParameter
                IncorrectState
                AuthorizationFailed
                AuthenticationFailed
                PermissionDenied
                Timeout
                NoSuccess
      Notes:    - jd might indicate new resource 
                  requirements, for example.
                - the action of migration might change the job 
                  identifier within the resource manager.
                - ideally, the submitted job description was
                  obtained by get_job_description(), and then
                  changed by the application.  This is not a
                  requirement though.
                - if the job is not in 'Running' or 'Suspended' 
                  state, an 'IncorrectState' exception is thrown.
                - the method can call the same exceptions as
                  the submit_job() and run() methods, in
                  particular in respect to an incorrect
                  job_description.
 
 
    - signal
      Purpose:  Ask the resource manager to deliver an arbitrary
                signal to a dispatched job.
      Format:   signal               (in int signum);	
      Inputs:   signum:               signal number to be
                                      delivered
      InOuts:   -
      Outputs:  -
      PreCond:  - job is in 'Running' or 'Suspended' state.
      PostCond: - the signal was delivered to the job.
      Perms:    Exec (job can be controlled).
      Throws:   NotImplemented
                BadParameter
                IncorrectState
                PermissionDenied
                AuthorizationFailed
                AuthenticationFailed
                Timeout
                NoSuccess
      Notes:    - there is no guarantee that the signal number 
                  specified is valid for the operating system 
                  on the execution host where the job is 
                  running, or that the signal can be delivered.
                - if the signal number is not supported by the
                  backend, a 'BadParameter' exception is thrown.
                - if the job is not in 'Running' or 'Suspended'
                  state, an 'IncorrectState' exception is 
                  thrown.
 \end{myspec}
 
 
 \subsubsection*{Class \T{job\_self}}
 
    The |job_self| class |IS-A| job which represents the current
    application (i.e. the very application which owns that
    |job_self| instance).  It can only by created by calling
    |get_self()| on a job service (that call can fail though).
 
    The motivation to introduce this class is twofold: (1) it
    allows to actively handle the current application as a grid
    job (e.g. to migrate it, or to obtain its job description
    for cloning/spawning); (2) as the class implements the
    steerable interface, it is possible to add |ReadWrite|
    metrics to its instance -- that way it is possible to expose
    these metrics to other external applications, which in fact
    allows to steer the current application.
 
    A drawback of this approach is that, in order to make an
    application steerable, a |job_service| instance is needed
    which can in fact return a |job_self| instance, which means
    there must be a resource manager available which can manage
    the current application -- that however has nothing to do
    with the concept of remote steering.  Future versions of the
    SAGA API may change that, and may make |job_self| a
    singleton, independent from the |job_service| behavior.  As
    a result, that class might disappear, and might not be
    maintained for backward compatibility.
 
 \begin{myspec}
    - DESTRUCTOR
      Purpose:  destroy the object
      Format:   DESTRUCTOR        (in  job_self  obj)
      Inputs:   obj:               the object to destroy
      InOuts:   -
      Outputs:  -
      PreCond:  -
      PostCond: -
      Perms:    -
      Throws:   -
      Notes:    - the object destruction does not imply a 
                  call to cancel() for the job_self instance.
 \end{myspec}
 
 
 \subsubsection{Examples}
 
 \begin{mycode}
  Example : simple job submission and polling for finish.
 
  // -----------------------------------------------
  // c++ example
  std::list <std::string>    transfers;
  saga::job_description jobdef;
  saga::job_service     js;
 
  transfers.push_back ("infile > infile");
  transfers.push_back ("ftp://host.net/path/out << outfile");
 
  jobdef.set_attribute        ("CandidateHost", "hostname");
  jobdef.set_attribute        ("Executable",    "job.sh");
  jobdef.set_attribute        ("TotalCPUCount", "16");
  jobdef.set_vector_attribute ("FileTransfer",  transfers);
 
  saga::job job = js.create_job (jobdef);
 
  job.run ();
 
  while ( 1 )
  {
    // get job state
    saga::job::state state = job.get_state ();
 
    // get list of hosts the job is/where running on
    std::list <std::string> hostlist = job.get_attribute
                                       ("ExecutionHosts");
 
    if ( saga::job::Running == state )
    {
      std::cout << "Job is running." << std::endl;
    }
    else if ( saga::job::Suspended == state )
    {
      std::cout << "Job is suspended." << std::endl;
    }
    else if ( saga::job::Done == state )
    {
      std::cout << "Job completed successfully." << std::endl;
      exit (0);
    }
    else if ( saga::job::Canceled == state )
    {
      std::cout << "Job canceled." << std::endl;
      exit (1); 
    }
    else
    {
      // state can only be 'Failed'
      assert (saga::job::Failed == state); 
 
      std::string exitcode = job.get_attribute ("ExitCode");
 
      std::cout << "Job failed with exitcode:" 
                << exitcode 
                << std::endl;
      exit ( atoi(exitcode) );
    }
 
    sleep (1); // idle
  }
 \end{mycode}
 
