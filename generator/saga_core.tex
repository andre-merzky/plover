
\newcommand{\sagadocument}{GFD-R-P.90}
\newcommand{\sagaversion}{1.1}
\newcommand{\sagabasename}{saga_core}
\newcommand{\sagaemail}{saga-core-wg@ogf.org}

\documentclass{article}

\usepackage{ifpdf}

\ifpdf
  \usepackage[pdftex]{graphicx}
  \usepackage[pdftex]{hyperref}
  \DeclareGraphicsExtensions{.pdf, .png, .jpg}
\else
  \usepackage{graphicx}
  \usepackage[hypertex]{hyperref}
  \DeclareGraphicsExtensions{.ps, .eps}
\fi


\usepackage{srcltx}
\usepackage{fancyhdr}
\usepackage{wrapfig}
\usepackage{fancyvrb}
\usepackage{lscape}
\usepackage{color}
\usepackage{xspace}
\usepackage{ulem}

\newcommand{\F}[1]{\textbf{FIXME: #1}}
\newcommand{\B}[1]{\textbf{#1}}
\newcommand{\I}[1]{\textit{#1}}
\newcommand{\T}[1]{\texttt{#1}}
\newcommand{\U}[1]{\underline{#1}}
\newcommand{\Q}[1]{\begin{quote}\I{#1}\end{quote}}

\newcommand{\D}[2]{\footnote{\textbf{DISCUSSION (#1): #2}}}
\newcommand{\DMark}{\footnotemark}
\newcommand{\DText}[2]{\footnotetext{\textbf{DISCUSSION (#1): #2}}}

\newcommand{\BI}[1]{\B{\I{#1}}}
\newcommand{\BU}[1]{\B{\U{#1}}}

\newlength{\myflen} 

\newcommand{\XMark}[1][1]{%
  \setlength{\myflen}{1.2em}%
  \marginpar{\rule[-0.3em]{.5mm}{#1\myflen}}%
  \xspace%
}
\newcommand{\XRed}[1]{\textit{\textcolor{red}{\sout{#1}}}}
\newcommand{\XGreen}[1]{\textbf{\textcolor{green}{#1}}}
\newcommand{\XBlue}[1]{\textit{\textcolor{blue}{#1}}}

\newif \iffinal \finalfalse
\newcommand{\sagafinal}{
  \finaltrue
  \renewcommand{\XMark}[1][1]{}\xspace%
  \renewcommand{\XRed}[1]{}
  \renewcommand{\XGreen}[1]{\textcolor{black}{##1}}
  \renewcommand{\XBlue}[1]{\textcolor{black}{##1}}
}

% spelling fix
\newcommand{\XSpell}[2][1]{\XGreen{#2}\XMark[#1]}
\newcommand{\XSpelln}[1]{\XGreen{#1}}

% corrected text
\newcommand{\XCorr}[2][1]{\XBlue{#2}\XMark[#1]}
\newcommand{\XCorrn}[1]{\XBlue{#1}}

% removed text
\newcommand{\XRem}[2][1]{\XRed{#2}\XMark[#1]}
\newcommand{\XRemn}[1]{\XRed{#1}}

% added text
\newcommand{\XAdd}[2][1]{\XCorr[#1]{#2}}
\newcommand{\XAddn}[1]{\XCorrn{#1}}

% replace text
\newcommand{\XRep}[3][1]{\XRed{#2~}\XBlue{#3}\XMark[#1]}
\newcommand{\XRepn}[2]{\XRed{#1~}\XBlue{#2}}

% changeset comment
\newcommand{\XComm}[2][1]{\XGreen{#2}\XMark[#1]}
\newcommand{\XCommn}[1]{\XGreen{#1}}

\newcommand{\MUST}       {\T{MUST}\xspace}
\newcommand{\MUSTNOT}    {\T{MUST NOT}\xspace}
\newcommand{\REQUIRED}   {\T{REQUIRED}\xspace}
\newcommand{\SHALL}      {\T{SHALL}\xspace}
\newcommand{\SHALLNOT}   {\T{SHALL NOT}\xspace}
\newcommand{\SHOULD}     {\T{SHOULD}\xspace}
\newcommand{\SHOULDNOT}  {\T{SHOULD NOT}\xspace}
\newcommand{\RECOMMENDED}{\T{RECOMMENDED}\xspace}
\newcommand{\MAY}        {\T{MAY}\xspace}
\newcommand{\OPTIONAL}   {\T{OPTIONAL}\xspace}

\newcommand{\sshift}{\hspace*{1em}}
\newcommand{\sunshift}{\hspace*{-1em}}
\newcommand{\shift}{\hspace*{3em}}
\newcommand{\unshift}{\hspace*{-3em}}
\newcommand{\down}{\vspace*{1em}}
\newcommand{\downn}{\vspace*{0.5em}}
\newcommand{\up}{\vspace*{-1em}}
\newcommand{\upp}{\vspace*{-0.5em}}

\newcommand{\LF}{Look-\&-Feel\xspace}

\setlength{\parskip}{1em}
\setlength{\parindent}{0em}
\setlength{\fboxsep}{1em}

\newcommand{\mywfig}[4]{
  \begin{wrapfigure}{#1}{#2\textwidth}
    \includegraphics[width=#2\textwidth]{#3}
    \caption{\label{fig:#3} \footnotesize \B{#4}}
    \vspace*{-1em}
  \end{wrapfigure}
}

\newcommand{\myfig}[2]{
  \begin{figure}[!ht]
    \begin{center}
      \includegraphics[width=0.95\textwidth]{#1}
      \caption{\label{fig:#1} \footnotesize \B{#2}}
    \end{center}
  \end{figure}
}

\newcommand{\mysfig}[2]{
  \begin{figure}[!ht]
    \begin{center}
      \includegraphics[width=0.45\textwidth]{#1}
      \caption{\label{fig:#1} \footnotesize \B{#2}}
    \end{center}
  \end{figure}
}

\newcommand{\myxfig}[3]{
  \begin{figure}[!ht]
    \begin{center}
      \includegraphics[width=#1\textwidth]{#2}
      \caption{\label{fig:#2} \footnotesize \B{#3}}
    \end{center}
  \end{figure}
}


% \usepackage[titles]{tocloft}
%   \renewcommand{\cftbeforesecskip}{-0.0ex}
%   \renewcommand{\cftbeforesubsecskip}{-1ex}
%   \renewcommand{\cftbeforesubsubsecskip}{-1ex}
%   \renewcommand{\cftbeforetabskip}{-1ex}
%   \renewcommand{\cftbeforefigskip}{-1ex}

\newcommand{\sagatocdepth}{2}
\setcounter{tocdepth}{\sagatocdepth}

\pagestyle{fancy}
\pagenumbering{arabic}

\newcommand{\sagadate}{\today}

\newcommand{\sagaheader}{}
  \lhead{\sagadocument}
  \chead{\sagaheader}
  \rhead{\sagadate}
  \lfoot{\hrulefill\\\T{\sagaemail} \hfill \thepage}
  \cfoot{}
  \rfoot{}

\newcommand{\sagansec}[2]
{
  \section{#1}
  \label{sec:#2}
  \renewcommand{\sagaheader}{#1}
  \iffinal
    \input{\sagabasename_#2.tex.final}
  \else
    \input{\sagabasename_#2.tex}
  \fi
}

\newcommand{\sagannsec}[2]
{
  \section*{#1}
  \label{sec:#2}
  \renewcommand{\sagaheader}{#1}
  \iffinal
    \input{\sagabasename_#2.tex.final}
  \else
    \input{\sagabasename_#2.tex}
  \fi
}

\newcommand{\sagasec}[2]
{
  \newpage
  \section{#1}
  \label{sec:#2}
  \renewcommand{\sagaheader}{#1}
  \iffinal
    \input{\sagabasename_#2.tex.final}
  \else
    \input{\sagabasename_#2.tex}
  \fi
}

\newcommand{\sagassec}[2]
{
  \newpage
  \subsection{#1}
  \label{ssec:#2}
  \renewcommand{\sagaheader}{#1}
  \iffinal
    \input{\sagabasename_#2.tex.final}
  \else
    \input{\sagabasename_#2.tex}
  \fi
}

\newcommand{\sagasubpar}[1]{~\\ \underline{#1}~\\}

\newenvironment{shortlist}{
  \begin{itemize}
  \vspace*{-0.5em}
   \setlength{\itemsep}{-.1em}
}{
  \end{itemize}
  \up
}

\newenvironment{shortenum}{
  \begin{enumerate}
  \vspace*{-0.5em}
   \setlength{\itemsep}{-.1em}
}{
  \end{enumerate}
  \upp
}

\newcounter{itemlistcnt}
\newcommand{\itemlist}[2]{
  \setcounter{itemlistcnt}{0}
  \begin{list}{\B{#1-\arabic{itemlistcnt}:}}{
    \setlength{\topsep}{0em}
    \setlength{\itemsep}{0em}
    \setlength{\parsep}{0.5em}
    \usecounter{itemlistcnt}
    }
    #2
  \end{list}
}


\DefineVerbatimEnvironment{mycode}{Verbatim}
{
  label=Code Example,
  fontsize=\small,
  frame=single,
% framerule=1pt,
  framesep=1em,
  numbers=left,
  gobble=2
}


\DefineVerbatimEnvironment{myio}{Verbatim}
{
  fontsize=\small,
  frame=lines,
% framerule=1pt,
  framesep=1em
}


\DefineVerbatimEnvironment{mysmallspec}{Verbatim}
{
  fontsize=\small,
  frame=lines,
% framerule=1pt,
  framesep=1em
}

\DefineVerbatimEnvironment{myspec}{Verbatim}
{
  fontsize=\normalsize,
  frame=lines,
% framerule=1pt,
  framesep=1em
}

\newcommand{\sagaverb}[1]{
 \DefineShortVerb{#1}
}

\newcommand{\saganoverb}[1]{
 \UndefineShortVerb{#1}
}

\newcommand{\Bull}{$\bullet$\xspace}
\newcommand{\Circ}{$\circ$\xspace}

\newcommand{\sagabib}[1]{
  \renewcommand{\sagaheader}{References}
  \addcontentsline{toc}{section}{References}
  \label{sec:References}
  \bibliographystyle{abbrv}
  \bibliography{#1}
}



\sagaverb{\|}
% \sagafinal

\begin{document}

 \thispagestyle{empty}

 \sagadocument{}\hfill Tom Goodale, Cardiff\\
 SAGA-CORE-WG   \hfill Shantenu Jha, UCL\footnotemark[1]\\
           \rightline {Hartmut Kaiser, LSU}\\
           \rightline {Thilo Kielmann, VU\footnotemark[1]}\\
           \rightline {Pascal Kleijer, NEC}\\
           \rightline {Andre Merzky, VU/LSU\footnotemark[1]}\\
           \rightline {John Shalf, LBNL}\\
           \rightline {Christopher Smith, Platform}\\[1em]
           \rightline {January 15, 2008 }\\
  Version: \sagaversion \hfill {Revised \sagadate}

  \footnotetext[1]{editor}

  \hrulefill\\

  \B{\large A Simple API for Grid Applications (SAGA)}\\

  \U{Status of This Document}

  This document provides information to the grid community,
  proposing the core components for an extensible
  Simple API for Grid Applications (SAGA Core API). It is
  supposed to be used as input to the definition of language
  specific bindings for this API, and by implementors of these
  bindings.  Distribution is unlimited.

  In May 2009, a number of errata have been applied to this document.
  A complete changelog can be found in the appendix.  Most changes
  should be backward compatible with the original spec (see
  changelog).\\


  \U{Copyright Notice}

  Copyright \copyright~Open Grid Forum (2007-2009).  All Rights
  Reserved.\\

  \U{Abstract}

   This document specifies the core components for the
   Simple API for Grid Applications (SAGA Core API), a high
   level, application-oriented API for grid application
   development.  The scope of this API is derived from the
   requirements specified in GFD.71 ("A Requirements Analysis
   for a Simple API for Grid Applications").  It will in the
   future be extended by additional API extensions.

  \newpage

  \tableofcontents

  \newpage

%-----------------------------------------------------------------
% Intro, structure, disclaimer, ...
%-----------------------------------------------------------------
                                        
  \sagasec {Introduction                 }{intro}
  \sagasec {General Design Considerations}{design}

% \sagansec{SAGA API Specification       }{spec}
\sagansec{SAGA API Specification -- Look\,\&\,Feel}
                                          {nonfunctional}

  \sagassec{SAGA Error Handling          }{error}
  \sagassec{SAGA Base Object             }{object}
  \sagassec{SAGA URL Class               }{url}
  \sagassec{SAGA I/O Buffer              }{buffer}
  \sagassec{SAGA Session Management      }{session}
  \sagassec{SAGA Context Management      }{context}
  \sagassec{SAGA Permission Model        }{permission}
  \sagassec{SAGA Attribute Model         }{attributes}
  \sagassec{SAGA Monitoring Model        }{monitoring}
  \sagassec{SAGA Task Model              }{tasks}

  \sagansec{SAGA API Specification -- API Packages}
                                          {functional}
 
  \sagassec{SAGA Job Management          }{job}
  \sagassec{SAGA Name Spaces             }{namespaces}
  \sagassec{SAGA File Management         }{files}
  \sagassec{SAGA Replica Management      }{logicalfiles}
  \sagassec{SAGA Streams                 }{stream}
  \sagassec{SAGA Remote Procedure Call   }{rpc}

  \sagasec {Intellectual Property Issues }{disclaimers}

  \appendix

  \section*{Appendix}

  \sagasec {SAGA Code Examples           }{examples}
  \sagasec {Changelog                    }{changelog}
% \sagasec {Known Issues \& Feedback     }{feedback}

  \sagabib {saga_core}

\end{document}


